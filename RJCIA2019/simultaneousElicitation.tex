\documentclass[a4paper,twoside]{article}
\usepackage{rfia2000}
\usepackage[T1]{fontenc}
\usepackage{times}

\author{\begin{tabular}[t]{c@{\extracolsep{6em}}c@{\extracolsep{6em}}c}
	B. Napolitano${}^1$ & O. Cailloux${}^1$ & P. Viappiani${}^2$ \\
	\end{tabular}
{} \\
\\
${}^1$        LAMSADE, UMR 7243, CNRS and Universit\'e Paris Dauphine, PSL Research University, Paris, France   \\
${}^2$        LIP6, UMR 7606, CNRS and Sorbonne Universit\'e, Paris, France
{} \\
\\
%Mon adresse compl\`ete \\
beatrice.napolitano@dauphine.fr, olivier.cailloux@dauphine.fr, paolo.viappiani@lip6.fr\\
%{\bf Domaine principal de recherche}: RFP ou IA\\
%{\bf Papier soumis dans le cadre de la journée commune}: OUI ou NON
}

\thispagestyle{plain}
\pagestyle{plain}

\usepackage[T1]{fontenc}
\usepackage[utf8]{inputenc}
\usepackage{newunicodechar}
\usepackage{textcomp}
\newunicodechar{⇒}{\implies}
\newunicodechar{≠}{\ensuremath{\neq}}
\newunicodechar{≤}{\leq}
\newunicodechar{≥}{\geq}
%… Horizontal Ellipsis
\DeclareUnicodeCharacter{2026}{\ifmmode\dots\else\textellipsis\fi}
\newunicodechar{∧}{\land}
\newunicodechar{∨}{\lor}
\newunicodechar{∩}{\cap}
\newunicodechar{∪}{\cup}
%¬ Not Sign
\DeclareUnicodeCharacter{00AC}{\ifmmode\lnot\else\textlnot\fi}
\newunicodechar{⇔}{\Leftrightarrow}
\newcommand{\N}{ℕ}
\newunicodechar{ℕ}{\mathbb{N}}
\usepackage{algorithm, algpseudocode}
\usepackage{amsmath,amssymb,enumerate,amsthm}
\usepackage{natbib}
\usepackage[strict]{siunitx}
\usepackage{hyperref}
\usepackage{mathrsfs}

\usepackage[small]{caption}
\usepackage{graphicx}
\usepackage{amsmath}
\usepackage{booktabs}

\usepackage{bm}
\usepackage{empheq}
\usepackage{cleveref}
\usepackage{xcolor}

\newcommand{\email}[1]{\href{mailto:#1}{#1}}
\newcommand{\denselist}{\itemsep -2pt\topsep-6pt\partopsep-6pt}
\newcommand{\commentOC}[1]{\textcolor{blue}{\small$\big[$OC: #1$\big]$}}

\newcommand{\pref}{\succ}%real, connected pref, strict
\newcommand{\prefr}{{\succ}^\text{r}}%real, connected pref, strict
\newcommand{\ppreflarge}{\succeq^\text{p}}%partial pref
\newcommand{\ppref}{\succ^\text{p}}%partial pref
\newcommand{\pprefinv}{\prec^\text{p}}%partial pref
\newcommand{\nppref}{\nsucc^\text{p}}%negated partial pref
\newcommand{\linors}{\mathcal{L}(A)}
%https://tex.stackexchange.com/a/45732, works within both \set and \set*, same spacing than \mid (https://tex.stackexchange.com/a/52905).
\newcommand{\suchthat}{\;\ifnum\currentgrouptype=16 \middle\fi|\;}

%Thanks to https://tex.stackexchange.com/q/154549
\makeatletter
\newcommand{\newrelation}[2]{% #1 = control sequence, #2 = replacement text
  \@ifdefinable{#1}{%
    \def#1{%
    \@ifnextchar_{\csname\string#1\endcsname}{\mathrel{#2}}%
    }%
    \@namedef{\string#1}##1##2{\mathrel{#2_{##2}}}%
  }%
}
\makeatother

\newrelation{\pinc}{\!\parallel\!}%partial pref, complement (incomparable)
%\newrelation{\pinc}{Q^\text{p}}%partial pref, complement (incomparable)

\newcommand{\profile}{\bm{v}}%(complete) profile
\newcommand{\pprofile}{{\bm{p}}}%partial profile
\newcommand{\w}{\bm{w}}
\newcommand{\W}{\mathcal{W}}
\newcommand{\Co}{\mathcal{C}}
\newcommand{\pw}{W}%our knowledge about the weights
\newcommand{\powersetz}[1]{\mathscr{P}^*(#1)}
\newcommand{\strat}[1]{\emph{#1}}

\DeclareMathOperator{\Regret}{Regret}
\DeclareMathOperator{\SCORE}{Score}
\DeclareMathOperator{\PMR}{PMR}
\DeclareMathOperator{\MR}{MR}
\DeclareMathOperator{\MMR}{MMR}
\DeclareMathOperator*{\argmax}{argmax}
\DeclareMathOperator*{\argmin}{argmin}

\newtheorem{claim}{Claim}
\newtheorem{prop}{Proposition}
\newtheorem{corollary}{Corollary}
\newtheorem{definition}{Definition}
\newtheorem{example}{Example}

\DeclarePairedDelimiter\set{\{}{\}}
\DeclarePairedDelimiter\card{\lvert}{\rvert}
\DeclarePairedDelimiter\abs{\lvert}{\rvert}

%	\tolerance=2000
%	%Accept overfull hbox up to...
%	\hfuzz=1cm
%	%Reduces verbosity about the bad line breaks.
%	\hbadness 10000
%	%Reduces verbosity about the underful vboxes.
%	\vbadness=10000

\title{\Large\bf Simultaneous Elicitation of Committee and Voters' Preferences}

\begin{document}
\date{}
\maketitle
\thispagestyle{empty}
%\subsection*{R\'esum\'e}
%{\em
%	Le choix social traite du probl\`eme de la d\'etermination d'un choix consensuel \`a partir des pr\'ef\'erences de diff\'erents agents. Dans le cadre classique, la r\`egle de vote est fix\'ee \`a l’avance et les informations compl\`etes concernant les pr\'ef\'erences des agents sont fournies. Dans cet article, nous supposons que la r\`egle de vote et les pr\'ef\'erences des agents sont partiellement connues. Ainsi, nous fournissons un protocole interactif d’\'elicitation bas\'e sur le regret minimax et d\'eveloppons plusieurs strat\'egies d’interrogation, associant questions au comit\'e et questions aux agents, afin d’obtenir les informations les plus pertinentes et de converger rapidement vers un r\'esultat optimal ou s'approchant de l'optimal.
%}
%\subsection*{Mots Clef}
%Choix social computationnel, IA dans l'incertain, \'elicitation des pr\'ef\'erences, regret minimax.

\subsection*{Abstract}
{\em
	Social choice deals with the problem of determining a consensus choice from the preferences of different voters. In the classical setting, the voting rule is fixed beforehand and full information concerning the preferences of the voters is provided.
	Recently, the assumption of full preference information has been questioned by a number of researchers and several methods for eliciting preferences have been proposed.
	In this paper we go one step further and we assume that both the voting rule and the voters' preferences are partially specified. In this setting, we present an interactive elicitation protocol based on minimax regret and develop several query strategies that interleave questions to the chair and questions to the voters in order to attempt to acquire the most relevant information in order to quickly converge to optimal or a near-optimal alternative.
}
\subsection*{Keywords}
Computational Social Choice, uncertainty in AI, preference elicitation, minimax regret.

\section{Introduction}
In a traditional social choice setting, both the social choice function and the full preference orderings of the voters are expressed beforehand. Nevertheless, in real situations this is not always the case. When considering decisions with large sets of alternatives, requiring voters to express full preference orderings can be prohibitively costly and, perhaps, not necessary in order to determine a consensus choice. Furthermore, it is often difficult for non-expert users to formalize a voting rule on the basis of some generic preferences over a desired aggregation method. Consider as an example a situation in which the council of a city wants to build a community allotment and the decision of what to grow has to be taken. The voters are the residents of the neighborhood and the alternatives consist of all the vegetables that can grow in that particular area. An external observer, who is helping with the voting procedure, is responsible for collecting the preferences of both the residents (regarding the vegetable to grow) and the members of the council (regarding the method for aggregating the residents' preferences). Imagine that, after collecting some preferences, she notices that some alternatives are always the least preferred. In this case, it is really unlikely that these alternatives would be selected as winners by any reasonable voting rule. Thus, she may consider to ask the voters only their preferences over few strategical alternatives instead of asking them the full preference orderings. Furthermore, if she notices that a particular vegetable is the most preferred of almost the totality of the residents there is no need to know the specific voting rule. Any reasonable aggregation procedure, in fact, would select this alternative as a winner. The reasoning behind this is that it is not unrealistic to assume that the committee, that has to decide how to aggregate these preferences, is not able to define a specific procedure. \\
These observations have motivated a number of recent works considering social choice with partial preference orders \citep{Xia2008, Pini2009, Konczak05} and incremental elicitation \citep{Kalech2011, Naamani-Dery2015} of voters' preferences. Lu and Boutilier \cite{Lu2011} proposed the use of minimax regret to drive incremental vote and preferences elicitation when the social choice function is fixed and known. Furthermore, several authors \citep{Stein1994, Llamazares2013, Viappiani2018} worked on positional scoring rules with uncertain weights, assuming that the preferences of the voters are fully known. Some elicitation methods for a quite general class of rules based on weak orders have also been proposed by Cailloux and Endriss \citep{Cailloux2014}. \\
In this paper we focus on positional scoring rules, that are
a particularly common method used to aggregate rankings, and we assume that both the voters’ preferences and the social choice rule are partially specified.
We develop methods for computing the minimax-optimal alternative using positional scoring rules and we provide incremental elicitation methods to acquire relevant preference information. We then discuss several heuristics that determine queries, either to a voter or to the committee, that quickly reduce minimax regret.
While previous works have considered either partial information about the voters' preferences or a partially specified aggregation method, we do not know of any work considering both sources of uncertainty at the same time.

\section{Partial information}
\label{sec:background}
We consider a set $A$ of $m$ alternatives (products, restaurants, public projects, etc.) and a set $\set{1, …, n}$ of voters. Each voter $j$ comes from an infinite set $\N$ of potential voters and is associated to her “real” preference order $\pref_j$ which is a linear order (connex, transitive, asymmetric relation) over the alternatives.
A {\em profile} is the association of a preference to each voter and it is equivalently represented by $(\pref_1,\ldots,\pref_n)$ or by ${\profile=(v_1,\ldots,v_n) \in V}$ where $v_j(i) \in \set{1, \ldots, m}$ denotes the rank (position) of alternative $i$ in the preference order $\pref_j$ and $V$ is the set of all possible preference profiles.
A social choice function associates a profile with a set of winners; we consider {\em positional scoring rules (PSR)} which attach weights to positions according to the scoring vector $(w_1, \ldots, w_m)$. Without loss of generality, we assume $w_1=1$ and $w_m=0$.
An alternative $x$ obtains a score that depends on the rank obtained in each of the preference orders:
\begin{align*}
\label{eq:srule}
s^{\profile, \w}(x) = \sum_{j=1}^{n} w_{v_j(x)}
= \sum_{r=1}^{m} \alpha^{x}_r w_r 
\end{align*}
where $\alpha^{x}_r$ is the number of times that alternative $x$ was ranked in the $r$-th position. The winners are the alternatives with highest scores.

In this work we assume fixed, but unknown, a profile $\profile^* = (\pref^*_1, \ldots, \pref^*_n)$ and a weight vector $\w^*$.
Our knowledge at a given time of the preference of voter $j$ is encoded by a partial order over the alternatives, thus a transitive and asymmetric binary relation, denoted by $\ppref_j$; we assume that preference information is truthful, i.e. 
${a \ppref_j b ⇒ a \pref_j^* b}$.
An incomplete profile ${\pprofile = (\ppref_1, \ldots, \ppref_n)}$ maps each voter to a partial preference. A completion of $\ppref_j$ is any linear order $\pref$ that extends $\ppref_j$ and we indicate with $C(\ppref_j) = \set{{\succ} \in \linors \suchthat {\ppref_j} \subseteq {\succ}}$ the set of possible completions of $\ppref_j$, where $\linors$ is the set of linear orders on $A$.

Our knowledge regarding the weights of the positional scoring rule is represented by a set of constraints restraining the possible values that they can take. We consider a convex sequence of weights and we assume that the chair is able to specify additional preferences about how the social choice function should behave. 
Such requirements are encoded with linear constraints about the vector $\w$ and the set of these constraints is denoted by $\Co_W$. Given the set of convex weight vectors $\W$, we use $\pw \subseteq \W$ to denote the set of weight vectors compatible with the chair's preferences. We will provide an example of such constraints, used to specify information about weights, in Section \ref{sec:elicit}.

\section[Minimax regret under partial profile and weight information]{
Robust winner determination}
\label{sec:mmr}
As a decision criterion to determine a winner, we propose to use minimax regret \cite{Savage1954}. This concept has been used for robust optimization under data uncertainty \cite{Kouvelis1997} as well as in decision-making with uncertain utility values \cite{Salo2001,Boutilier2006}. Lu and Boutilier \cite{Lu2011} have adopted this criterion for winner determination in social choice with the preferences of the voters that are only partially known, while the social choice function is predetermined and known.
We use {\em maximum regret} to quantify the worst-case error, which measures, intuitively, how far an alternative is from the optimal one given current knowledge; the alternatives that minimize this error are selected as tied winners.
%, providing us with a form of robust optimization.
The maximum regret of an alternative is the highest possible difference between its score and the one achievable by any alternative in any state compatible with our current knowledge.
It is considered by assuming that an adversary can both 1) extend the partial profile $\pprofile$ into a complete one, and 2) instantiate the weights choosing among any weight vector in $\pw$.
We formalize the notion of minimax regret in multiple steps.
First of all, $\Regret^{\profile,\w}(x)$ is the “regret” of selecting $x$ as a winner instead of choosing the optimal alternative under $\profile$ and $\w$:
\[\Regret^{\profile,\w}(x) = \max_{y \in A} s^{\profile,\w}(y) - s^{\profile, \w}(x).\]
The pairwise max regret $\PMR^{\pprofile,W}(x,y)$ of $x$ relative to $y$ given partial profile $\pprofile$ and the set of weights $W$ is the worst-case loss of choosing $x$ instead of $y$ under all possible realizations of the full profile {\em and} weights. Max regret $\MR^{\pprofile,W}(x)$ is the worst-case loss of $x$ and $\MMR^{\pprofile,W}$ is the value of minimax regret obtained when recommending a minimax optimal alternative. 
\begin{align}
\PMR^{\pprofile,W}(x,y) & = \max_{\w \in W} \max_{\profile \in C(\pprofile)}s^{\profile,\w}(y) - s^{\profile, \w}(x) \\
\MR^{\pprofile,W}(x) & = \max_{y \in A} \PMR^{\pprofile,W}(x,y)\\
%& = \max_{\w \in W} \max_{\profile \in C(\profile)} \Regret(x, \profile, \w) \\
\MMR^{\pprofile,W} & = \min_{x \in A} \MR^{\pprofile,W}(x) \\
x^{*}_{\pprofile,W} & = \argmin_{x \in A}\MR^{\pprofile,W}(x) \in A^*_{\pprofile, W}
\end{align}

By picking as consensus choice an alternative associated with minimax regret $x^*_{\pprofile, W} \in A^*_{\pprofile, W} $, we can provide a recommendation that gives worst-case guarantees, giving some robustness in face of uncertainty. 
%\[x^{*}_{\pprofile,W} = \argmin_{x \in A}\MR^{\pprofile,W}(x)\]
In cases of ties in minimax regret, we can either decide to return all minimax alternatives $A^*_{\pprofile, W}$ as winners or to pick just one of them using a tie-breaking strategy.
Because the constraints in $\Co_W$ are not necessarily linear $\PMR$ is not straightforwardly optimized, but it can be computed by adapting the reasoning of Lu and Boutilier \cite{Lu2011} to the case of uncertain weights using linear programming. This optimization strategy uses the assumption of convexity.

\section{Interactive elicitation} 
\label{sec:elicit}
We propose an incremental elicitation method based on minimax regret.
At each step, the system may ask a question either to one of the voters or to the chair.
The goal is to acquire relevant information to reduce minimax regret as quickly as possible.
As termination condition of elicitation, we can check whether minimax regret is lower than a threshold or, if we wish optimality, we can perform elicitation until minimax regret drops to zero.

\paragraph{Question types}
We distinguish between questions asked to the voters and questions asked to the chair.
For the first we consider comparison queries that ask a particular voter to compare two alternatives. Then, the partial profile $\pprofile$ is augmented on the basis of the voter's answer. Thus, if voter $j$ answers a comparison query stating that alternative $a$ is preferred to $b$, then the partial order $\ppref_j$ is augmented with $a \ppref_j b$ and by transitive closure.
The query for the chair, instead, aims at refining our knowledge about the scoring rule; in particular, we assume we can acquire constraints of the type:
\[ w_{r} - w_{r+1} \geq \lambda (w_{r+1} - w_{r+2}) \] 
for $r \in \{1,\ldots,m-2\}$, relating the difference between the importance of consecutive ranks.
% $r$ and $r+1$ with the difference between ranks $r+1$ and $r+2$.

\paragraph{Elicitation strategies}
A strategy is a function (or a random process) that, given our partial knowledge so far, gives us a question that should be asked. 

The \strat{Random} strategy first decides with a probability of $1/2$ each whether it will ask a question about weights or about a preference ordering. Then it equiprobably draws a question among the set of the possible ones. This strategy is clearly non efficient but it gives us a baseline for comparisons.

The \strat{Extreme completions} strategy considers the solution of the minimax regret game, given the current knowledge, and estimates the contribution to the regret of our uncertainty about the weights and the profile. Then, it selects a question for the chair or for the voters depending on which of these two values is the highest.

The \strat{Pessimistic} strategy selects among all the possible questions the one that leads to minimal regret in the worst case. Because all our questions are binary questions, for each of them the pessimistic strategy first assumes the answer is \emph{"No"}, it updates the knowledge according to this negative answer and computes the minimax regret. Then, it does the same assuming the answer is positive. Finally, it uses some aggregation of these two MMR values (for example \emph{max} or \emph{avg}) as a score of the question. The question with the highest score is selected.

The \strat{Two phase} strategy asks a predefined, non adaptive sequence of $m-2$ questions to the chair in order to gather informations about the weights $\w$ of the scoring rule, using one question per rank except the extreme ones that we are assuming known ($w_1=1$, $w_m=0$). Then it only asks questions about the voters, using questions as defined in the \strat{Pessimistic} strategy. 

\section{Conclusions}  
\label{sec:conclusions}
In this paper we have considered a social choice setting with partial information and we have proposed the use of minimax regret both as a means of robust winner determination as well as a guide to the process of simultaneous elicitation of preferences and voting rule.

The next step is to test our strategies in order to compare the performance of our interleaved elicitation approach that mixes questions to the chair and to the voters, to a more classical approach that elicits the rule first and then the voters' preferences (or vice-versa). 

Further development of elicitation strategies, considering alternative heuristics, is also an important direction for future work together with the extension of the approach to voting rules beyond scoring rules.
\bigskip

Acknowledgements: We thank the reviewers for comments helping to improve the paper. 
%{\small
\bibliography{biblio}
\bibliographystyle{abbrv}
%\bibliographystyle{plain} 
%}
\end{document}
