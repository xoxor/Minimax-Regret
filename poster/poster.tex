%nag warns a lot about tikzposter.
%\RequirePackage[l2tabu, orthodox]{nag}
\documentclass[blockverticalspace=3cm]{tikzposter}
\input{preamble/packages}
\input{preamble/redac}
\input{preamble/math_basics}
\input{preamble/math_mine}
\input{preamble/draw}
\usepackage{tabularx}
\listfiles


\definecolor{myorange}{RGB}{250, 95, 0} 
\definecolor{myburgundy}{RGB}{150, 0, 15} 

\colorlet{backgroundcolor}{white}
\colorlet{framecolor}{orange}
\colorlet{blocktitlebgcolor}{myorange}

\newcommand{\profile}{\bm{v}}%(complete) profile
\newcommand{\pprofile}{{\bm{p}}}%partial profile
\newcommand{\w}{\bm{w}}
\newcommand{\W}{\mathcal{W}}
\newcommand{\Co}{\mathcal{C}}
\newcommand{\pw}{W}%our knowledge about the weights
\newcommand{\strat}[1]{\emph{#1}}
\newcommand{\ppref}{\succ^\text{p}}%partial pref
\newcommand{\pprefeq}{\succeq^\text{p}}%partial pref
\newcommand{\pref}{\succ}% pref
\DeclareMathOperator{\Regret}{Regret}
\DeclareMathOperator{\SCORE}{Score}
\DeclareMathOperator{\PMR}{PMR}
\DeclareMathOperator{\MR}{MR}
\DeclareMathOperator{\MMR}{MMR}

\makeatletter
\def\title#1{\gdef\@title{\scalebox{\TP@titletextscale}{%
			\begin{minipage}[t]{\linewidth}
				\centering
				#1
				\par
				\vspace{0.5em}
			\end{minipage}%
		}}}
		\makeatother
%I find these settings useful in draft mode. Should be removed for final versions.
	%Which line breaks are chosen: accept worse lines, therefore reducing risk of overfull lines. Default = 200.
%		\tolerance=2000
	%Accept overfull hbox up to...
%		\hfuzz=2cm
	%Reduces verbosity about the bad line breaks.
%		\hbadness 5000
	%Reduces verbosity about the underful vboxes.
%		\vbadness=1300

\title{Simultaneous Elicitation of Committee and \\ Voters Preferences}
\institute{$^1$ LAMSADE, Université Paris-Dauphine, Paris, France \\ $^2$ LIP6, Sorbonne Universit\'e, Paris, France}
\author{B. Napolitano$^1$, O. Cailloux$^1$ and P. Viappiani$^2$}


\begin{document}
\maketitle[titletotopverticalspace=11cm]

\begin{columns}
	\column{0.5}
		\block{Setting}{
			\begin{tikzpicture}[remember picture,overlay]
				\path (current page.north west) ++(1.5cm, -1cm) node[anchor=north west, inner sep=0] (first) {
					\includegraphics[height=6.5cm]{dauphine_psl2018.png}
				};
				\path (current page.north east) ++(-1.5cm, -1cm) node[anchor=north east, inner sep=0] {
					\includegraphics[height=7cm]{LAMSADE95.jpg}
				};
			\end{tikzpicture}%
		%
%			\begin{tikzpicture}[remember picture,overlay]
%				\path (current page.south west) ++(1.5cm, 1.5cm) node[anchor=south west, text width=27cm] {
%					Olivier Cailloux and Yves Meinard. \emph{A formal framework for deliberated judgment}. Under revision at Theory and Decision. \href{https://arxiv.org/abs/1801.05644}{arXiv:1801.05644 [cs.AI]}.
%				};
%			\end{tikzpicture}%
		%
			\textbf{Incomplete profile and uncertain positional scoring rule \\}
			\begin{tikzfigure}
				\includegraphics{set.png}
				%		\caption{.}
				%		\label{fig:b1}
			\end{tikzfigure}

			\textbf{Goals}
			\begin{itemize}
				\item Development of query strategies interleaving questions to the chair and to the voters in order to simultaneously elicit preferences and voting rule
				\item Robust winner determination
			\end{itemize}
		}
		\block{Minimax Regret}{
			
			\begin{align*}
			\Regret^{\profile,\w}(x) & = \max_{y \in A} s^{\profile,\w}(y) - s^{\profile, \w}(x) \\
			\PMR^{\pprofile,W}(x,y) & = \max_{\w \in W} \max_{\profile \in C(\pprofile)}s^{\profile,\w}(y) - s^{\profile, \w}(x) \\
			\MR^{\pprofile,W}(x) & = \max_{y \in A} \PMR^{\pprofile,W}(x,y)\\
			%& = \max_{\w \in W} \max_{\profile \in C(\profile)} \Regret(x, \profile, \w) \\
			\MMR(\pprofile,W) & = \min_{x \in A} \MR^{\pprofile,W}(x) \\
			x^{*}_{\pprofile,W} \in A^*_{\pprofile, W} & = \argmin_{x \in A}\MR^{\pprofile,W}(x)
			\end{align*}
		}
		\block{Question Types}{
			\begin{itemize}
				\item \textbf{Questions to the voters}
				\begin{itemize}
					\item Comparison queries that ask a particular agent to compare two alternatives
					\[a \pref_j b \quad ?\]
				\end{itemize}
				\item \textbf{Questions to the chair}
				\begin{itemize}
					\item Queries relating the difference between the importance of consecutive ranks $r$ and $r+1$
					\[ w_{r} - w_{r+1} \geq \lambda (w_{r+1} - w_{r+2}) \quad ? \] 
				\end{itemize}
			\end{itemize}
		}
		\block{Elicitation strategies}{
			A function that, given our partial knowledge so far, returns a question that should be asked. 
			\begin{itemize}
				\item \textbf{Random}: it decides, with a probability of $1/2$, whether to ask a question to the voters or to the chair, then it equiprobably draws a question among the set of the possible ones;
			
				\item \textbf{Extreme completions}: it asks a question to the chair or to the agents depending on which uncertainty contributes the most to the regret;
			
				\item \textbf{Pessimistic}: it selects the question that leads to minimal regret in the worst case considering, and aggregating, both possible answers to each question; 
			
				\item \textbf{Two phase}: it asks a predefined, non adaptive sequence of $m-2$ questions to the chair and then it only asks questions about the agents.
			\end{itemize}
		}
	\column{0.5}
		\block{Motivation and approach}{
			\begin{itemize}
				\item \textbf{Who?}
				\begin{itemize}
					\item Imagine to be an \emph{external observer} helping with the voting procedure
				\end{itemize}
				\item \textbf{Why?}
				\begin{itemize}
					\item Requiring voters to express \emph{full
					preference} orderings can be prohibitively \emph{costly}, especially for decisions with lots of alternatives
					\item \emph{Difficult} for non-expert users \emph{to formalize} a voting rule on the basis of some generic preferences over a desired aggregation method
				\end{itemize}
				\item \textbf{How?}
				\begin{itemize}
					\item \emph{Minimax regret}: given the current knowledge, the alternatives with the lowest worst-case regret are selected as tied winners
				\end{itemize}		
				\item \textbf{Assumptions:}
				\begin{itemize}
					\item Voters and committee have true preferences in mind
					\item The voting rule is a Positional Scoring Rule where the scoring vector $(w_1, \dots , w_m)$ is a convex sequence of weights and $w_1=1$, $w_m=0$ 
				\end{itemize}
			\end{itemize}
		}
		\block{Pairwise Max Regret Computation}{
			The computation of $\PMR^{\pprofile,W}(x,y)$ can be seen as a game in which an adversary can both extend the partial	profile into a complete one and instantiate the weights	choosing among any feasible weight vector
			\begin{itemize}
				\item \textbf{Profile Completion}\\
				For any other alternative $a$ 
				\begin{align*} 
				a \pref_j x &⇔ ¬(x \pprefeq_j a)\\
				\label{eq:comply}
				y \pref_j a &⇔ ¬(a \pprefeq_j y) ∧ ¬((x \pprefeq_j y) ∧ ¬(x \pprefeq_j a)).
				\end{align*}
				Considering the example \\
					\begin{center}
						\includegraphics{set5.png}
					\end{center}
					
				\item \textbf{Weights Choice}\\
				The vector that satisfies the constraints specified by the chair so far and maximize the PMR is chosen. \\
				In the previous example the vector {\color{red}$(1,0,0)$} is chosen.
			\end{itemize}
			}
			\block{References}{
				\small{
					\bibliography{biblio}
					\bibliographystyle{plain}
				}
			}
		
\end{columns}
%
%\block{Deliberation can change your mind}{
%	\newlength{\seplott}
%	\setlength{\seplott}{3em}
%	\begin{center}
%		Choose between L1 and L2, then choose between L3 and L4. Are you sure?
%		\framebox{
%			\begin{tikzpicture}[grow'=right, sibling distance=3cm, level distance=8cm]
%				\path node (l1) {L1} child {
%					node {1M €} edge from parent node[above] {100\%}
%				};
%				\path (l1-1.east) ++ (\seplott, 0) node (ql12) {VS};
%				\path (ql12) ++ (\seplott, 0) node[anchor=west] (l2) {L2} child {
%					node {0M €} edge from parent node[above] {1\%}
%				} child {
%					node {1M €} edge from parent node[above right] {89\%}
%				} child {
%					node {5M €} edge from parent node[below] {10\%}
%				};
%				\path (l1) ++(40cm, 0) node (l3) {L3} child {
%					node {0M €} edge from parent node[above] {89\%}
%				} child {
%					node {1M €} edge from parent node[below] {11\%}
%				};
%				\path (l3 -| l3-1.east) ++ (\seplott, 0) node (ql34) {VS};
%				\path (ql34) ++ (\seplott, 0) node[anchor=west] (l4) {L4} child {
%					node {0M €} edge from parent node[above] {90\%}
%				} child {
%					node {5M €} edge from parent node[below] {10\%}
%				};
%			\end{tikzpicture}
%		}
%	\end{center}
%	\mbox{}
%
%	\begin{minipage}{0.6\textwidth}
%%		\rule{\textwidth}{1cm}
%		\begin{itemize}
%			\item First observation (Bernouilli): don’t be content with maximizing (untransformed) expected revenue!
%			\item Second observation: $i$ could be intuitively attracted by L1 $\succ$ L2 and L3 $\succ$ L4 (Allais’s problem)
%			\item Including Savage
%			\item And might change her mind when given a reasoning pro expected utility
%			\item “There is, of course, an important sense in which preferences, being entirely subjective, cannot be in error”
%			\item … “but in a different, more subtle sense they can be.” (Savage, \emph{The Foundations of Statistics})
%			\item[⇒]Systematic decision principles might help deliberate
%		\end{itemize}
%	\end{minipage}%
%	\begin{minipage}{0.33\textwidth}%
%		\footnotesize
%%		\rule{\textwidth}{1cm}
%		\centering
%		\setlength{\seplott}{3em}
%		\begin{tikzpicture}[grow'=right, sibling distance=1.5cm, level distance=8cm]
%			\path node (l1) {L1} child {
%				node {1M €} edge from parent node[above] {89\%}
%			} child {
%				node {1M €} edge from parent node[below] {11\%}
%			};
%			\path (l1 -| l1-1.east) ++ (\seplott, 0) node (ql12) {VS};
%			\path (ql12) ++ (\seplott, 0) node[anchor=west] (l2) {L2} child {
%				node {1M €} edge from parent node[above] {89\%}
%			} child {
%				node {0M €} edge from parent node[above right] {1\%}
%			} child {
%				node {5M €} edge from parent node[below] {10\%}
%			};
%			\path (l1.south) ++(0, -4cm) node[anchor=north] (l3) {L3} child {
%				node {0M €} edge from parent node[above] {89\%}
%			} child {
%				node {1M €} edge from parent node[below] {11\%}
%			};
%			\path (l3 -| l3-1.east) ++ (\seplott, 0) node (ql34) {VS};
%			\path (ql34) ++ (\seplott, 0) node[anchor=west] (l4) {L4} child {
%				node {0M €} edge from parent node[above] {89\%}
%			} child {
%				node {0M €} edge from parent node[above right] {1\%}
%			} child {
%				node {5M €} edge from parent node[below] {10\%}
%			};
%		\end{tikzpicture}
%	\end{minipage}
%}
%
%\begin{columns}
%	\column{0.45}
%		\block{Study deliberated judgement}{
%			The proposed research program aims at the following.
%			\begin{enumerate}
%				\item Define \ac{DJ} formally
%				\begin{itemize}
%					\item Given a set of arguments
%					\item Of an individual $i$
%					\item[⇒] The position that is stable facing counter-arguments
%				\end{itemize}
%				\item Define the concept of a model of someone’s \ac{DJ}
%				\begin{itemize}
%					\item[⇒] A model articulates claims concerning $i$’s \ac{DJ} and argues for its claim
%				\end{itemize}
%				\item Define validity of a model
%				\begin{itemize}
%					\item[⇒] Correctly captures $i$’s \ac{DJ}
%				\end{itemize}
%				\item Study conditions for falsifying models using observable data only
%				\begin{itemize}
%					\item[⇒] Let models debate, use $i$ as a judge
%				\end{itemize}
%			\end{enumerate}
%		}
%	\column{0.55}
%		\block{Example of a situation and a model of it}{
%			\begin{tabularx}{40cm}{LYl}
%				\toprule
%				\text{Notation}	&\text{Here}	&Description\\
%				\midrule
%				T	&\set{\prop}	&The topic, containing propositions about which $i$ deliberates\\
%				\args	&\set{\ar, \ar[1], \ar[2], \ar[3]}	&The arguments\\
%				{\ileadsto} \subseteq \args × T	&\set{(\ar, \prop), (\ar[1], \prop)}	&Support as considered by $i$\\
%				{\ibeatse} \subseteq \args × \args	&\set{(\ar[2], \ar[1])}	&Attacks as considered by $i$ in some perspective\\
%				{\mbeats} \subseteq \args × \args	&\set{(\ar[3], \ar[2])}	&Attacks as considered by the model $\eta$\\
%				\bottomrule
%			\end{tabularx}
%			\vspace{2em}
%			\begin{center}
%			\begin{tikzpicture}
%				\path node (eq) {weather f.\ predicts so ($\ar[1]$)};
%				\path (eq.east) node[anchor=west] (eql) {$\ileadsto$};
%				\path (eql.east) node[anchor=west] (eqr) {rain tomorrow ($\prop$)};
%				\path (eq.south) ++(0, -\BDNodeSep) node[anchor=north] (wrong) {weather forecast is often wrong ($\ar[2]$)};
%				\path (wrong) edge[/Beliefs/attack] node[right] {$\ibeatse$} (eq);
%				\path (wrong.south) ++(0, -\BDNodeSep) node[anchor=north] (right) {weather forecast is more often right ($\ar[3]$)};
%				\path (right) edge[/Beliefs/attack] node[right] {$\mbeats$} (wrong);
%				
%				\path (eqr.east) node[anchor=west] (complexl) {\reflectbox{$\ileadsto$}};
%				\path (complexl.base east) node[anchor=base west] (complex) {complex arg. ($\ar$)};
%			\end{tikzpicture}
%			\end{center}
%%		\raggedleft{Example of a }
%		}
%\end{columns}
%
%\begin{columns}
%	\column{0.5}
%		\block{Application: test axioms of decision theory}{
%			\begin{itemize}
%				\item Axioms considered appropriate normatively?
%				\begin{itemize}
%					\item But some (Allais, Ellsberg) disagree
%				\end{itemize}
%				\item Proposal: build models resting on those axioms
%				\item Test models: their convincing power will give us indications about the reasonableness of the axioms for “normal” people (meaning, not scientists studying decision theory)
%			\end{itemize}
%		}
%	\column{0.5}
%		\block{Application: test conceptions of justice}{
%			\begin{itemize}
%				\item Philosophers have proposed sophisticated conceptions of justice (Rawls, Nozick, …)
%				\item Individual’s shallow intuitions about justice are observed and used to confront Rawls or others (Experimental Social Choice)
%				\item Proposal: study reactions of individuals to arguments of philosophers rather than just shallow intuitions
%				\item Move towards Reflective equilibrium (Goodman, Rawls)
%			\end{itemize}
%		}
%\end{columns}
\end{document}

