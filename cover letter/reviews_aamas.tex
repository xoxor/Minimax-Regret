\subsection*{Review 1}
\subsubsection*{Summary:}	The authors use the notion of minimax regret to guide the process of preference elicitation in an election (they elicit both the voters' preferences and the preferences of the chair regarding the scoring rule used). The authors show that the methodology of Lu and Boutilier for computing minimax regret still applies in their setting and present some experimental results.
\subsubsection*{Strengths:}	1) Minimax regret is a rather natural notion, which has not received much attention so far.
\subsubsection*{Weaknesses:}
1) The theoretical results are relatively simple extensions of the results of Lu and Boutilier
\newline 2) The experimental results seem quite preliminary.

\subsubsection*{Detailed comments:}	The paper is quite well written, clear, relevant for the conference. The biggest issues I have regard significance. The idea of using minimax regret to guide elicitation is very good, but is explored quite well by Lu and Boutilier. Here the authors extend this framework by eliciting the scoring rule used, but there is very little discussion as to why this really is such a good idea. I cannot easily think of a half-realistic setting where one would know that the scores need to be convex, but at the same time one would not know which ones exactly to use.

More specific comments come below:
\begin{enumerate}
\item Related work \\
As the paper is partially related to designing/choosing a scoring rule for a given setting---and as the problem of manipulation is mentioned---it might be worth to consider the following paper:

Dorothea Baumeister, Tobias Hogrebe:
Manipulative Design of Scoring Systems. AAMAS 2019: 1814-1816

Regarding partial preferences and manipulation, one may also look at the following one (although it is less related, for sure):

Dorothea Baumeister, Piotr Faliszewski, Jérôme Lang, Jörg Rothe:
Campaigns for lazy voters: truncated ballots. AAMAS 2012: 577-584


\item Concurrent elicitation of the scoring rule \\
I am not really convinced that the assumption that a scoring rule is not known makes sense. On the one hand, there is the problem of choosing the scoring rule manipulatively (but since the authors mention that the elections at hand are built into recommendation systems etc., this is not really an issue). The other problem is that I can hardly think of a scenario where the chair can answer the questions about the scoring rule, but somehow it is difficult for him to produce the full rule in a single step. So what do we gain?

\item Claims vs Propositions \\
I do not understand why the authors use "Claims" and not "Propositions".

\item I think that Claim 4 should say something along the lines: "There is a profile P', such that ... and P' can be computed effectively".

\item I can hardly imagine a person who faced with the profile P' from Claim 4 could meaningfully answer if a is better than b or the other way round. A piece of software could do it, though---but probably by precomputing the scoring rule to use (which removed the need for eliciation).

\item Proof of Claim 4: I do not understand notation (3,m) or (4,m-1) etc.

\item Empirical evaluation is very preliminary
I have a number of complaints regarding the evaluation process:
	\begin{enumerate}
		\item Why do the authors use IC only? Why not other statistical cultures? Why not some real-life data? After all, the experimental part is the main contribution of the paper and it is done in a fairly minimalistic way.
		
		\item Why do the authors look at very small elections only? 15 candidates and 30 voters seems completely inadequate for the motivations from the introduction.
		
		\item The conclusion that the number of questions required to reach low regret cannot be made based on Table 3. There is simply far too little data to have any sort of confidence in claims like this.	
	\end{enumerate}

\end{enumerate}

All in all, I think it is certainly a nice enough paper to be accepted as short, but is below the bar for the full acceptance.

\subsubsection*{Questions for rebuttal:}
Q1: Why did you only look at IC elections?
\newline Q2: How to use your approach in large elections? Hundreds of candidates, thousands of voters? After all, showing a strategy that can deal with large elections would meaningfully extend the work of Lu and Boutilier?
\newline Q3: Can you provide a realistic example where elicitation of the scoring rule is useful?
\subsubsection*{Score:}	
4: (poor (good attempt but too many concerns, so probably should be rejected))

\subsubsection*{Comments added after rebuttal:}
I found the example provided in the response to be convincing.

I still think that testing the heuristics on IC only is insufficient. Even if it is the hardest case, I would also like to see how the algorithm peforms in the easier ones. Can it exploit the structure when the structure is present? One cannot evaluate such issues by looking at IC alone.

\subsection*{Review 2}
\subsubsection*{Summary:}
The paper studies a voting problem wherein neither the voting rule nor the voters' preferences are specified in advance. Rather, an elicitation procedure is used to determine the election winner by asking the voters questions about their preferences and asking the chair questions about the voting rule.

Specifically, the preferences of the voters are assumed to be given by a fixed set of linear orders that are unknown to the elicitation platform. In addition, the voting rule is assumed to belong to the class of positional scoring rules with monotone weights with diminishing differences. The elicitation procedure comprises of the chair specifying the winning candidates on example profiles, and the voters responding to a series of pairwise comparisons.

At each step, the minimax regret criterion (Lu and Boutilier, IJCAI'11) is used to determine the next query, and this procedure is continued until convergence (e.g., either the regret drops below a certain threshold or a certain number of questions have been asked). The regret of an alternative, informally speaking, is the "loss" incurred for picking that alternative instead of an "optimal" one under adversarial choice of completion of the partial preferences elicited so far as well as the worst-case choice of positional weights.

The main contributions of the submission are:
1. Proposing a framework for simultaneously eliciting voters' preferences and information about the voting rule based on the minimax regret criterion.
2. Experimental comparison of a number of elicitation heuristics in terms of the number of questions asked or the regret value.

The empirical observations suggest that a modest number of questions (to the voters as well as the chair) are sufficient in arriving at a reasonable choice, and that the regret drops super-linearly as a function of the number of questions asked over the course of elicitation.
\subsubsection*{Strengths:}	The idea of modeling simultaneous uncertainty in the voting rule as well as the voters' preferences is quite interesting. It is also useful to know that the minimax regret framework of Lu and Boutilier (IJCAI'11), previously studied in the setting of a known voting rule, can be extended to the aforementioned more general problem.
\subsubsection*{Weaknesses:}	The lack of theoretical guarantees for the proposed elicitation strategy, as well as missing reasoning behind several of the experimental observations, makes this a weak paper in my opinion.

\subsubsection*{Detailed comments:}
I like the high-level question asked by the paper, namely whether it is possible to simultaneously elicit the information about the voting rule and the voters' preferences. However, the significance of the contribution is less clear. The results, which largely comprise of experiments, are at the level of proposing a heuristic that seems to work well in simulations but lacks formal theoretical guarantees.

I believe the paper will be much stronger if the authors could provide theoretical convergence guarantees for the proposed heuristics. Additionally, in the "Empirical Evaluation" section, instead of simply describing the observations, it would be useful to also formally state the hypotheses used in each experiment, and explain how the experiments support or refute those hypotheses. For example, you could describe whether (and why) one might expect "Pessimistic" strategy to work better than the "Elitist" strategy, and then discuss to what extent did the simulations confirm that hypothesis.

Some minor comments are below.
\begin{enumerate}
\item Page 1
\begin{itemize}
	\item In several places you use the phrase "permits to" when it might be clearer to say "allows for".
\end{itemize}

\item Page 2
\begin{itemize}
	\item "that showed that"-->who showed that.
\end{itemize}

\item Page 3 
\begin{itemize}
	\item "there are no necessary winner"-->there is no necessary winner.
\end{itemize}

\item Page 4 
\begin{itemize}
	\item "less good than or equal to"-->You could rather say "weakly worse".
	\item Using $\succeq^p_j(x)$ to denote the set of alternatives weakly worse than x can be confusing. Please consider using a different notation such as $A^(\succeq^p_j,x)$.
\end{itemize}

\item Page 5
\begin{itemize}
	\item Claim 4 is written in the form of a definition, and it is not clear what is actually being claimed here. Maybe you could say that "there exists a profile P' satisfying such and such properties".
	\item "each of the p+q agent"-->each of the p+q agents
	\item "worst than"-->worse than
\end{itemize}

\item Page 6 
\begin{itemize}
	\item "on order to"-->in order to
	\item "an important weight"-->a large weight
	\item "For generating the weights we first draw $m - 1$ numbers uniformly at random"-->What is the range from which you make the draws?
\end{itemize}

\item Page 7 
\begin{itemize}
	\item "In-dept evaluation"-->In-depth evaluation
	\item Figure 1, X axis label should say "Number of questions"
	\item "to be ran"-->to be run
	\item "We see that the number of questions required to reach a low regret level grows approximately linearly with the number of agents"-->Can you provide a formal regression analysis to support this claim?
\end{itemize}

\item Page 8 
\begin{itemize}
	\item "allow to quickly"-->allows to quickly
	\item "worst regret"-->worst-case regret
	\item "that allows to reproduce our experiments, and more"-->It would be useful to mention the other benefits explicitly.
	\item "halving the number of questions will only reduce the gain in regret by less than half"-->Might be clearer to say "halving the number of questions reduces the regret by more than 50$\%$".
\end{itemize}

\end{enumerate}
\subsubsection*{Questions for rebuttal:}	1. On Page 1, you write that "The quality of the recommendation increases faster than linearly with the number of questions, after an initial phase with almost no increase in quality, before slowing down and converging to an optimal recommendation." What is the reasoning behind this trend?
\newline 2. How did you decide the range of the parameter $\lambda$ to be [1,n]?
\subsubsection*{Score:}	
4: (poor (good attempt but too many concerns, so probably should be rejected))

\subsubsection*{Comments added after rebuttal:}
Thanks to the authors for responding to the questions raised in my review. My concern about the significance of the paper remains. In the absence of any formal theoretical guarantees about the performance of the elicitation heuristic, the paper will likely be of limited interest. The authors note that obtaining such guarantees could be difficult. Nevertheless, the paper could certainly benefit from a revised experimental analysis that clearly formulates the hypotheses to be tested and how the experiments support or refute those hypotheses.

\subsection*{Review 3}
\subsubsection*{Summary:}
This paper considers elicitation in the context of scoring rules when both the preferences of the voters and the weights defining the rule might be missing. The authors model the problem via regret minimization and propose several types of queries and querying strategies. The paper also presents an interesting experimental study which supports the presented approach.
\subsubsection*{Strengths:}	The problem addressed is interesting and has potential for practical applicability.
Addressing missing information in both voters and the chair’s preferences is novel. The experimental section is a good starting point.
\subsubsection*{Weaknesses:}	The discussion on the motivation behind considering uncertainty on the scoring rule’s weights is not very convincing. They claim the chair might not be able to formalize the rule…
Another point that is not clear is when, in practice a system like this would be deployed.
The application of regret minimization is somewhat incremental w.r.t. the literature.

\subsubsection*{Detailed comments:}	reference elicitation in the context of voting is certainly a relevant topic for AAMAS.
The setting considered here is not per se novel, as elicitation of voters’ and the chair’s preferences have been considered before. The originality lies in considering them both at the same time.
While this is original, the motivation of when such a setting may occur in practice is not clear. In particular, I wonder about the timing. It is assumed that voters and candidates are known, information about the voters preferences is available and yet the voting rule has not yet been decided and the elicitation process is going on at the same time. It seams unlikely to me.
As I mentioned, I am also not very convinced about a chair not knowing which rule he wants/or is using while voters are expressing preferences. A much stronger argument would be that of a third party discovering information about a voter’s preferences and the rule by being able to observe results or to tap in and acquire some information. Obviously that would mean having much less control over what information is acquired than assumed here…it might be similar to random strategy studied in the experimental section.

The content is technically not overly sophisticated, so I have no doubt about its soundness.
The presentation flows well and all parts are clearly written. The description of the experimental results is also sufficiently detailed to enable reproducibility.

A minor complaint is that Figures 1 and 2 are blurred.
\subsubsection*{Questions for rebuttal:}
1) Could testing mixed strategies yield better results than the pure ones?
\subsubsection*{Score:}	
7: (good (probably should be accepted))

\subsection*{Metareview}
The main reason for the reject recommendation is the missing novelty in this paper. It basically is a reimplementation of the ideas from Lu and Boutilier. Furthermore, in the experimental section, clear hypotheses are missing and the reasons for limiting the analysis to IC are not convincing.