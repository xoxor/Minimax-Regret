% !TEX root = resubmission.tex

Dear editors and reviewers,

We have submitted this article for the first time for IJCAI 2020. The reviewer’s evaluations of it were: from R1, Accept (good paper, I can argue for accepting it); from R2, Weak accept (marginally above the acceptance theshold, rejecting it would not be that bad); from R3, Weak reject (marginally below the acceptance threshold, accepting it would not be that bad); and from R4, Reject (Not good enough, I can argue for rejecting it). The strongest criticisms from the last two reviewers were, from R3, to improve presentation with a larger example and to expand the discussion (R3 incorrectly thought that we had room to spare within the constraints imposed by IJCAI); and from R4, that our literature review omitted important related works; that the results in the paper were incremental in nature; and that it was almost impossible to find the main contribution or take away message from the paper. The metareview pointed out that “One might view the work as somewhat incremental, but there is some substantial solid work here”; but recommended rejection because of 1) the very brief discussion of the empirical evaluation and 2) the incomplete literature review. The metareviewer also commented that “after some further work on the paper there's a reasonable chance that it could be accepted for another high quality conference.”

We extended very significantly the literature review and the experimental section, also taking advantage of two supplementary pages, and re-submitted the work for AAMAS 2021. The evaluations were: from R1 and R2, poor (4, good attempt but too many concerns, so probably should be rejected); and from R3, good (7, probably should be accepted). R1 considered our theoretical results as relatively simple extensions of the results of Lu and Boutilier and found the experimental results quite preliminary. About the first point, R1 stated to be unable to think about an example of a setting where our proposition would apply. After rebuttal, R1 stated that the example we provided in our response was convincing, but insisted that we need to test our heuristics beyond the impartial culture hypothesis. About experiments, R1 regretted that we use only impartial culture assumption; that we look at very small elections only; and that the way some of our conclusions are phrased make them appear more general than allowed by the experimental observations. R2 criticized the lack of theoretical guarantees for the proposed elicitation strategies, as well as the missing reasoning behind several of the experimental observations. After rebuttal, R2 took note of our observation that obtaining theoretical guarantees is generally considered beyond reach in a regret minimization setting; but insisted that experimental analysis should be presented by first formulating hypotheses to be tested, then showing how the experiments support or refute those hypotheses. Finally, R3, similarly to R1, found unclear which sorts of situations would permit an application of our proposed approach.
