%\typeout{IJCAI-19 Instructions for Authors}

% These are the instructions for authors for IJCAI-19.

\documentclass{article}
\pdfpagewidth=8.5in
\pdfpageheight=11in
% The file ijcai19.sty is NOT the same than previous years'
\usepackage{ijcai19}

% Use the postscript times font!
\usepackage{times}
\usepackage{soul}
\usepackage{url}
\usepackage[hidelinks]{hyperref}
\usepackage[utf8]{inputenc}
\usepackage[small]{caption}
\usepackage{graphicx}
\usepackage{amsmath}
\usepackage{booktabs}
%\usepackage{algorithm}
%\usepackage{algorithmic}
\urlstyle{same}

% the following package is optional:
%\usepackage{latexsym} 

 
\author{ID: 7094}
%\author{Beatrice Napolitano$^1$\and Olivier Cailloux$^1$	\and  Paolo Viappiani$^{2}$
%	\affiliations $^1$ LAMSADE, UMR 7243, CNRS and Universit\'e Paris Dauphine, PSL Research University, Paris, France\\ $^2$ LIP6, UMR 7606, CNRS and Sorbonne Universit\'e, Paris, France
%	\emails \{firstname.lastname\}@dauphine.fr, paolo.viappiani@lip6.fr
%}
%Restoring page numbers as missing them out violates basic human rights of the reviewers. -- Olivier
\thispagestyle{plain}
\pagestyle{plain}

\usepackage[T1]{fontenc}
\usepackage[utf8]{inputenc}
\usepackage{newunicodechar}
\usepackage{textcomp}
%\newunicodechar{⇒}{\implies}
\newunicodechar{⇒}{\Rightarrow}
\newunicodechar{≠}{\ensuremath{\neq}}
\newunicodechar{≤}{\leq}
\newunicodechar{≥}{\geq}
%… Horizontal Ellipsis
\DeclareUnicodeCharacter{2026}{\ifmmode\dots\else\textellipsis\fi}
\newunicodechar{∧}{\land}
\newunicodechar{∨}{\lor}
\newunicodechar{∩}{\cap}
\newunicodechar{∪}{\cup}
%¬ Not Sign
\DeclareUnicodeCharacter{00AC}{\ifmmode\lnot\else\textlnot\fi}
\newunicodechar{⇔}{\Leftrightarrow}
\newcommand{\N}{ℕ}
\newunicodechar{ℕ}{\mathbb{N}}
\usepackage{algorithm, algpseudocode}
\usepackage{amsmath,amssymb,enumerate,amsthm}
\usepackage{natbib}
\usepackage[strict]{siunitx}
\usepackage{hyperref}
\usepackage{mathrsfs}

\usepackage{bm}
\usepackage{empheq}
\usepackage{cleveref}
\usepackage{xcolor}

\newcommand{\email}[1]{\href{mailto:#1}{#1}}
\newcommand{\denselist}{\itemsep -2pt\topsep-6pt\partopsep-6pt}
\newcommand{\commentOC}[1]{\textcolor{blue}{\small$\big[$OC: #1$\big]$}}
\newcommand{\commentBN}[1]{\textcolor{red}{\small$\big[$BN: #1$\big]$}}

\newcommand{\pref}{\succ}%real, connected pref, strict
\newcommand{\prefeq}{\succeq}%real, connected pref, strict
\newcommand{\prefr}{{\succ}^\text{r}}%real, connected pref, strict
\newcommand{\pprefeq}{\succeq^\text{p}}%partial pref
\newcommand{\ppref}{\succ^\text{p}}%partial pref
\newcommand{\pprefinv}{\prec^\text{p}}%partial pref
\newcommand{\nppref}{\nsucc^\text{p}}%negated partial pref
\newcommand{\linors}{\mathcal{L}(A)}
%https://tex.stackexchange.com/a/45732, works within both \set and \set*, same spacing than \mid (https://tex.stackexchange.com/a/52905).
\newcommand{\suchthat}{\;\ifnum\currentgrouptype=16 \middle\fi|\;}

%Thanks to https://tex.stackexchange.com/q/154549
\makeatletter
\newcommand{\newrelation}[2]{% #1 = control sequence, #2 = replacement text
  \@ifdefinable{#1}{%
    \def#1{%
    \@ifnextchar_{\csname\string#1\endcsname}{\mathrel{#2}}%
    }%
    \@namedef{\string#1}##1##2{\mathrel{#2_{##2}}}%
  }%
}
\makeatother

\newrelation{\prefinc}{\!\parallel\!}%partial pref, complement (incomparable)
\newrelation{\pinc}{\bowtie^\text{p}}
%\newrelation{\prefinc}{Q^\text{p}}%partial pref, complement (incomparable)

\newcommand{\profile}{\bm{v}}%(complete) profile
\newcommand{\pprofile}{{\bm{p}}}%partial profile
\newcommand{\w}{\bm{w}}
\newcommand{\W}{\mathcal{W}}
\newcommand{\Co}{\mathcal{C}}
\newcommand{\pw}{W}%our knowledge about the weights
\newcommand{\powersetz}[1]{\mathscr{P}^*(#1)}
\newcommand{\strat}[1]{\emph{#1}}

\DeclareMathOperator{\Regret}{Regret}
\DeclareMathOperator{\SCORE}{Score}
\DeclareMathOperator{\PMR}{PMR}
\DeclareMathOperator{\MR}{MR}
\DeclareMathOperator{\MMR}{MMR}
\DeclareMathOperator{\leximax}{leximax}
\DeclareMathOperator*{\argmax}{argmax}
\DeclareMathOperator*{\argmin}{argmin}

\newtheorem{claim}{Claim}
\newtheorem{prop}{Proposition}
\newtheorem{corollary}{Corollary}
\newtheorem{definition}{Definition}
\newtheorem{example}{Example}

\DeclarePairedDelimiter\set{\{}{\}}
\DeclarePairedDelimiter\card{\lvert}{\rvert}
\DeclarePairedDelimiter\abs{\lvert}{\rvert}

\tolerance=2000
%Accept overfull hbox up to...
\hfuzz=1cm
%Reduces verbosity about the bad line breaks.
\hbadness 10000
%Reduces verbosity about the underful vboxes.
\vbadness=10000

\title{Simultaneous Elicitation of Scoring Rule and Agents Preferences for Robust Winner Determination}

\begin{document}
\maketitle
\begin{abstract}
Social choice deals with the problem of determining a consensus choice from the preferences of different agents (voters).
In the classical setting, the voting rule is fixed beforehand and full information concerning the preferences of the agents is provided.
Recently, the assumption of full preference information has been questioned by a number of researchers and several methods for eliciting preferences have been proposed.
In this paper we go one step further and we assume that both the voting rule and the agents preferences are partially specified. Focusing on positional scoring rules, we assume that the chair, while  not able to give a precise definition of the rule, is capable of answering simple questions  requiring to pick a winner from a specific  example profile.
Moreover, the preferences of the agents are incrementally acquired by asking comparison queries. In this setting, we propose a method  for robust approximate winner determination with minimax regret. We also provide an interactive elicitation protocol based on minimax regret and develop several query strategies that interleave questions to the chair and questions to the agents in order to attempt to acquire the most relevant information in order to quickly converge to optimal or a near-optimal alternative.
\end{abstract}

\section{Introduction}
Aggregation of preference information is a central task in many computer systems (recommender systems, search engines, etc).
In many situations, such as in group recommender systems, the goal is to find a consensus choice.
It is therefore natural to look at methods from social choice and see how they can be adapted for group decision-making in a computerized setting.

The traditional approach to social choice assumes that both the social choice function and the full preference orderings of the agents are expressed beforehand. These represent two strong hypothesis.
Requiring agents to express full preference orderings can be prohibitively costly (in terms of cognitive and communication cost).
This observation has motivated a number of recent works considering social choice with partial preference orders \citep{Xia2008, Pini2009, Konczak05} and incremental elicitation \citep{Kalech2011, Lu2011, Naamani-Dery2015} of agents’ preferences. \\ Furthermore, it is often difficult for non-expert users to formalize a voting rule on the basis of some generic preferences over a desired aggregation method. Thus, the first hypothesis should also be relaxed. 
%Also the first hypothesis should be relaxed considering that, in several situations, it may not be easy to precisely define the voting rule.Indeed, it is possible that the chair has some preferences over the desired aggregation method, but is not able to formalize the voting rule upfront.
The work of \citet{Cailloux2014} provides elicitation methods for a quite general class of rules based on weak orders.
When considering positional scoring rules, several authors \citep{Stein1994, Llamazares2013, Viappiani2018} have worked on positional scoring rules with uncertain weights, assuming that the preferences of the agents are fully known.
%it is possible to derive dominance relations (akin to stochastic dominance) that allow to eliminate some alternatives since they will be less preferred than another one for any instantiation of the weights \citep{Stein1994}.
%Among others, Llamazares and Pe{\~{n}}a have considered  the problem of dealing with underspecified weights in positional scoring rules.

In this paper we focus on positional scoring rules, that are a particularly common method used to aggregate rankings, and we assume that both the agents’ preferences and the social choice rule are partially specified. We develop methods for computing the minimax-optimal
alternative using positional scoring rules and we provide incremental elicitation methods to acquire relevant preference information. We then discuss several heuristics that determine queries, either to an agent or to the chair, that quickly reduce minimax regret. While previous works have considered either partial information about the agents’
preferences or a partially specified aggregation method, we do not know of any work considering both sources of uncertainty at the same time.

%In this paper we consider that both the agents’ preferences and the social choice rule are partially specified and we develop methods for computing the minimax-optimal alternative under these assumptions.
%changed from: We develop methods for computing the minimax-optimal alternative for scoring rules with partial agent preference information and partial information about the weights of the scoring rule.

%We also address the problem of elicitation, providing incremental elicitation methods to acquire relevant preference information. We then discuss several heuristics that attempt to simultaneously reduce the preference and voting rule uncertainty by determining queries, either to an agent or to the chair, that quickly reduce minimax regret.
%In particular, our query strategies focus simultaneously on reduction of relevant preference and voting rule uncertainty.

The paper is organized as follows.
In Section \ref{sec:background} we provide the necessary background and in Section \ref{sec:mmr} we introduce the minimax criterion, that selects a winner that minimizes the worst possible regret.
Then, in Section \ref{sec:elicit} we provide an interactive elicitation protocol based on minimal regret;  in Section \ref{sec:experiments} we present the empirical validation of our approach with simulations; and Section \ref{sec:conclusions} provides some final thoughts.

\section{Social choice with partial information}
\label{sec:background}
We now introduce some basic concepts.
We consider a set $A$ of $m$ alternatives (products, restaurants, movies, public projects, job candidates, etc.) and a set $\set{1, …, n}$ of agents (voters). Each agent $j$ comes from an infinite set $\N$ of potential agents and is associated to her “real” preference order $\pref_j \in \linors$ which is a linear order (a connected, transitive, asymmetric relation) over the alternatives.
Following the social choice nomenclature, we call {\em profile} the association of a preference to each agent, considering a subset of agents from the set $\N$, and denote a profile by $(\pref_1,\ldots,\pref_n)$.
A profile is equivalently represented by $\profile=(v_1,\ldots,v_n)$ where $v_j(i) \in \set{1, \ldots, m}$ denotes the rank (position) of alternative $i$ in the preference order $\pref_j$. 

Let $V$ be the set of possible preference profiles (the union, for any integer, $n$ of the $n$-fold cartesian product of the linear orders over the alternatives).
A social choice function $f : V \rightarrow \powersetz{A}$ associates a profile with a set of winners, where $\powersetz{A}$ represents the set of subsets of $A$ except for the emptyset. (Sets are used for tied winners.)
Among the many possible social choice functions, we consider {\em positional scoring rules (PSR)}, which attach weights to positions according to the vector $(w_1, \ldots, w_m)$ (also called the scoring vector).
An alternative obtains a score that depends on the rank obtained in each of the preference orders:
\begin{align}
\label{eq:srule}
s(x; \profile, \w) = \sum_{j=1}^{n} w_{v_j(x)}
= \sum_{r=1}^{m} \alpha^{x}_r w_r 
\end{align}
where $\alpha^{x}_r$ is the number of times that alternative $x$ was ranked in the $r$-th position.
The winners are the alternatives gighest scores.

In this work we assume fixed, but unknown to us, a profile $\profile^*$, representing the preferences $\pref_j^*$ of the agents, and a weight vector $\w^*$, representing the preferences of the chair, and we want to reason about partial preference information concerning those objects.
At a given time, our knowledge of agent $j$'s preference is encoded by a partial order over the alternatives, thus a transitive and asymmetric binary relation, denoted by $\ppref_j$. 
In this work we assume that preference information is truthful, i.e. $a \ppref_j b ⇒ a \pref_j^* b$.
%We use $\prefinc$ to denote incomparability, that is $a \prefinc_{j} b$ iff $a \nppref_j b \wedge b \nppref_j a$.
An incomplete profile $\pprofile = (\ppref_1, \ldots, \ppref_n)$ maps each agent to a partial preference.

A completion of $\ppref_j$ is any linear order $\pref$ that extends $\ppref_j$ and we indicate with $C(\ppref_j) = \set{{\succ} \in \linors \suchthat {\ppref_j} \subseteq {\succ}}$ the set of possible completions of $\ppref_i$.
Then $C(\pprofile)=C(\ppref_1)\times … \times C(\ppref_n)$ is the set of complete profiles extending $p$. Note that $\profile^* \in C(\pprofile)$.

We also assume that the weights of the scoring rule are only partially specified.
Therefore, the vector $(w_1,\ldots,w_m)$  is not known but we are given a set of constraints restraining the possible values that weights can take.
We consider a decreasing sequence of weights:
\begin{align}
1=w_{1} ≥ w_{2} ≥ \ldots ≥ w_{m}=0. \label{eq:monotone}
\end{align}
This is a natural assumption, as it is better to be ranked first than second, second than third, etc. 
Without loss of generality, we consider that $w_1=1$ and $w_m=0$. 

The weights of a scoring rule can model different preferences of the chair. 
For instance, the weights can control the inclination to favor ``extreme'' alternatives (often at either the top or the bottom of the input rankings) at the expenses of ``moderate'' alternatives (that are more consistently in the middle part of the input rankings).

An important class of scoring rule is the one composed of weights that represent a convex sequence \citep{Stein1994,Llamazares2016}, meaning that the difference between the weight of the first position and the weight of the second position is at least as great as the difference between the weights of the second and third positions, etc.
\begin{equation} 
\label{eq:convexity}
\forall r \in \{1,\ldots,m-2\}: w_r - w_{r+1} \geq w_{r+1}-w_{r+2}.
\end{equation}
The constraint above is often used when aggregating rankings in sport competitions.
We use $\W$ to denote the set of convex weight vectors.

In general it can be difficult to set the weights in an appropriate way.
We assume that in addition of basic requirements (monotonicity and convexity), the chair (the person or the organization that is supervising the voting process) is able to specify additional preferences about how the social choice function should behave.
In this work we assume that the preferences of the chair are encoded with linear constraints about the vector $\w$, relating the value of the weights of different positions, and the set of these constraints is denoted by $\Co_W$. Moreover, we use $\pw \subseteq \W$ to denote the set of weight vectors compatible with the preferences expressed by the chair about the scoring vector.

Of course it may be difficult for real decision makers to state preferences about the voting rule in such an abstract way.
These additional preferences can be elicited by asking questions about concrete profiles, for instance, by showing a complete profile of a small synthetic election and asking who should be elected in this case, as shown in Section \ref{sec:elicit}.

\section[Minimax regret under partial profile and weight information]{
Robust winner determination}
\label{sec:mmr}
In this paper, we consider a setting where both the agents' preferences and the preferences of the chair about the voting rule are incomplete.
Some authors have considered possible and necessary winners assuming a partial profile  \citep{Xia2008} or assuming an incompletely specified scoring rule \citep{Viappiani2018};
however, we note that, in typical settings, there are no necessary winner and too many possible winners.
In practice it is often useful to declare a winner given partial information.

As a decision criterion to determine a winner, we propose to use minimax regret. 
Minimax regret \citep{Savage1954} is a decision criterion that has been used for robust optimization under data uncertainty \citep{Kouvelis1997} as well as in decision-making with uncertain utility values \citep{Salo2001,Boutilier2006}.
\citet{Lu2011} have adopted minimax regret for winner determination in social choice with
the preferences of the agents that are only partially known, while the social choice function is predetermined and known.

In this work, we consider the simultaneous presence of uncertainty in agents' preferences and in weights.
Using {\em maximum regret} to quantify the worst-case error, the alternatives that minimize this error are selected as tied winners, providing us with a form of robust optimization.
Intuitively, the quality of a proposed alternative $a$ is how far $a$ is from the optimal one in the worst case, given the current knowledge.

The maximum regret is considered by assuming that an adversary can both 1) extend the partial profile $\pprofile$ into a complete profile, and 2) instantiate the weights choosing among any weight vector in $\pw$, where $\pprofile$ and $\pw$ represent our knowledge so far.
We formalize the notion of minimax regret in multiple steps.
First of all, $\Regret(x, \profile, \w)$ is the “regret” of selecting $x$ as a winner instead of choosing the optimal alternative under $\profile$ and $\w$:
\[\Regret(x, \profile, \w) = \max_{y \in A} s(y; \profile,\w) - s(x; \profile, \w).\]
The pairwise max regret $\PMR(x,y;\pprofile,W)$ of $x$ relative to $y$ given partial profile $\pprofile$ and the set of weights $W$
is the worst-case loss of choosing $x$ instead of $y$ under all possible realizations of the full profile {\em and} all possible instantiations of the weights:
\begin{align}
\PMR(x,y; \pprofile, W) & = \max_{\w \in W} \max_{\profile \in C(\pprofile)} s(y; \profile,\w) - s(x; \profile,\w).
\end{align}

Max regret $\MR(x;\pprofile,W)$ is the worst-case loss of $x$. That is the loss occurred as the result of an adversarial selection of the complete profile $\profile \in C(\pprofile)$ and of the scoring vector $\w \in W$ that together maximize the loss between $x$ and the true winner under $\profile$ and $\w$.
\begin{align}
\MR(x; \pprofile, W) & = \max_{y \in A} \PMR(x,y; \pprofile, W)\\
& = \max_{\w \in W} \max_{\profile \in C(\profile)} \Regret(x, \profile, \w).
\end{align}

Finally,  $\MMR(\pprofile,W)$ is the value of minimax regret under $\pprofile$ and $W$, obtained when recommending a minimax optimal alternative $x^*_{\pprofile, W}$:
\begin{align*}
\MMR(\pprofile,W) & = \min_{x \in A} \MR(x;\pprofile,W) \\
x^{*}_{\pprofile,W} \in A^*_{\pprofile, W} & = \argmin_{x \in A} \MR(x;\pprofile,W) 
\end{align*}
By picking as consensus choice
 an alternative associated with minimax regret, we can provide a recommendation that gives worst-case guarantees, giving some robustness in face of uncertainty (due to both not knowing the agents' preferences and the weights used in the aggregation). 
In cases of ties in minimax regret, we can either decide to return all minimax alternatives $A^*_{\pprofile, W}$ as winners or to pick just one of them using some tie-breaking strategy.

Observe that if $\MMR(\pprofile, W)\!=\!0$, then $x^{*}_{\pprofile,W}$ is a necessary co-winner; this means that for any valid completion of the profile and any feasible $w \!\in\! W$, $x^{*}_{\pprofile,W}$ obtains a highest score.

We note that our notion of regret gives some cardinal meaning to the scores: instead of just being used to select winners under the corresponding PSR, their differences are considered as representing the regret of the chair.

% Add some general remarks about using minimax regret

\paragraph{Computation of minimax regret}
In order to compute pairwise maximum regret, and therefore minimax regret, we decompose the $\PMR$ into the contributions associated to each agent by adapting the reasoning from \citet{Lu2011}.
The setting is however more challenging due to the presence of uncertainty in the weights.

Recall that, in the computation of $s(x; \profile, \w)$, $w_{v_j(x)}$ represents the score that $x$ obtains in the ranking $v_j$ (see Equation \ref{eq:srule}).
Since scoring rules are additively decomposable, we can consider separately the contribution of each agent to the total score. Thus, we can write the actual regret of choosing $x$ instead of $y$ as
\[
s(y; \profile,\w) - s(x; \profile, \w) = \sum_{j=1}^n w_{v_j(y)} - w_{v_j(x)},
\]
and we can rewrite $\PMR$ as follows:
\begin{align*}
& \PMR(x,y; \pprofile, W) = \max_{\w \in W} \max_{\profile \in C(\pprofile)} [ s(y; \profile,\w) - s(x; \profile,\w) ] \\
& =  \max_{\w \in W} \sum_{j=1}^{n} \max_{v_j \in C(\succ_j^p)} [w_{v_j(y)} - w_{v_j(x)}]. \\
\end{align*}
Note that in general the inner max depends on the weights chosen by the outer max.

We are interested in computing $\PMR(x, y; \pprofile, W)$. This represents the “worst” difference of score, thus the difference of score between $y$ and $x$ under the worst case preferences compatible with $\pprofile$ and $W$, where the worst case is the one that maximizes this difference of score.
We consider now a procedure for completing a partial profile that was first proposed by \citet{Lu2011} when considering %minimax regret with 
a fixed weight vector.
As we will show, this procedure can also be used when the weight vector is not completely known.

\begin{claim} \label{claim:completion}
There exists a completion $\hat{\profile} \in C(\pprofile)$ such that $\PMR(x,y; \pprofile, W) = \max_{\w \in W} [ s(y; \hat{\profile}, \w) - s(x; \hat{\profile}, \w) ]$ and such that the linear order $\hat{v}_{j}$ of each agent $j$ satisfies:
\begin{align} 
\label{eq:complx}
a \pref_j x &⇔ ¬(x \pprefeq_j a)\\
\label{eq:comply}
y \pref_j a &⇔ ¬(a \pprefeq_j y) ∧ ¬((x \pprefeq_j y) ∧ ¬(x \pprefeq_j a)).
\end{align} 
\end{claim}
\begin{proof}[Sketch of proof]
Consider our knowledge $\pprefeq_j$ about the preference of the agent $j$. 
The adversary's goal is to make the score of $y$ as high as possible and the score of $x$ as low as possible. 
To do this, he should complete $\ppref_j$ to $\pref_j$ by putting above $x$ as many alternatives as he can, that is, all the alternatives except those that are known to be worse than $x$ (those $a$ such that $x \pprefeq_j a$); and similarly, he should put below $y$ all the alternatives he can. Two conditions must be excluded for $a$ to go below $y$. The alternatives such that $a \pprefeq_j y$ can’t be put below $y$.
Furthermore, the first objective must take priority over the second one: when an alternative should go above $x$ according to the first objective (because $¬(x \pprefeq_j a)$), and $x$ is known to be better than $y$ (thus $x \pprefeq_j y$), then $a$ should be put above $x$ (irrespective of whether $a \pprefeq_j y$), which will move both $x$ and $y$ one rank lower than if $a$ had been put below $y$. 
This maximizes the adversary’s interests: because the weight vector is convex, the difference of scores will be lower when both alternatives are ranked lower (Equation \ref{eq:convexity}), and that difference of scores is in favor of $x$ when $x \ppref_j y$, thus to be minimized from the point of view of the adversary.
\end{proof}

Let ${\pprefeq_j}(x)$ designate the set of alternatives that are known to be considered by $j$ as less good than or equal to $x$, and ${\pprefinv_j}(y)$ be the set of alternatives known to be considered by $j$ as strictly better than $y$.
\begin{claim}
	The rank of $x$ in the PMR-maximizing linear orders of agent $j$ is $\hat{v}_{j}(x) = 1+\card{A}-\card{{\pprefeq_j}(x)}$, and the rank of $y$ is $\hat{v}_{j}(y)=1+\card{{\pprefinv_j}(y)}+\card{\beta}$, where $\card{\beta} = \card{A \setminus ({\pprefeq_j}(x) \cup {\pprefinv_j}(y))}$ if $(x \pprefeq_j y)$ and $\card{\beta} = 0$ otherwise.
\end{claim}
\begin{proof}
The rank of $x$ is directly obtained from \cref{eq:complx}. The rank of $y$ is obtained by complementing \cref{eq:comply}, obtaining $a \prefeq_j y ⇔ (a \pprefeq_j y) ∨ ((x \pprefeq_j y) ∧ ¬(x \pprefeq_j a))$, and, observing that $a \pref_j y ⇔ a ≠ y ∧ a \prefeq_j y$, obtaining that $a \pref_j y$ if and only if
\begin{equation}
\label{eq:betteryinter}
(a \neq y) ∧ [(a \pprefeq_j y) ∨ ((x \pprefeq_j y) ∧ ¬(x \pprefeq_j a))],
\end{equation} 
or equivalently, if and only if
\begin{equation}
\label{eq:bettery}
%a \pref_j y ⇔ 
(a \ppref_j y) ∨ ((x \pprefeq_j y) ∧ ¬(x \pprefeq_j a)).
\end{equation} 
Indeed, \eqref{eq:betteryinter} $⇒$ \eqref{eq:bettery}, and \eqref{eq:bettery} $⇒$ \eqref{eq:betteryinter} because $(x \pprefeq_j y) ∧ ¬(x \pprefeq_j a) ⇒ a ≠ y$ (as when $a = y$, $(x \pprefeq_j y)$ and $¬(x \pprefeq_j a)$ are opposite claims). Suffices now to rewrite \cref{eq:bettery} to let the two disjuncts designate disjoint sets:
\begin{equation}
\label{eq:betteryfinal}
a \pref_j y ⇔ 
(a \ppref_j y) ∨ ((x \pprefeq_j y) ∧ ¬(x \pprefeq_j a) ∧ ¬(a \ppref_j y)).
\end{equation} 
\end{proof}

The claim can also be understood by observing that in the case $(x \pprefeq_j y)$, $\beta$ is the number of alternatives incomparable with both $x$ and $y$.

\begin{claim}
The $\PMR$ can be written as:
\begin{align} 
\PMR(x,y; \pprofile, W)  
& = \max_{w \in W} \sum_{j=1}^n w_{\hat{v}_j(y)} - w_{\hat{v}_j(x)} = \\ 
& = \max_{w \in W} \sum_{r=1}^m (\hat{\alpha}_{r}^{y} - \hat{\alpha}_{r}^{x}) w_i. 
\end{align}
where $\hat{\alpha}_{r}^{y}$ (resp. $\hat{\alpha}_{r}^{x}$)  is the number of times $y$ (resp. $x$) is at rank $r$ in the complete profile $\hat{\profile}$.
\end{claim}
The last claim shows that PMR is linear in the weights.
Recall that the preferences of the chair are encoded with linear constraints $\Co_{W}$.
The pairwise max regret $\PMR(x,y; \pprofile,W)$ can be obtained as the solution of the following linear program defined on the variables $w_1, …, w_m$, which represent the weights attached to different positions. 
\begin{align*}
\max_{\w} & \sum_{r=1}^m (\hat{\alpha}_{r}^{y} - \hat{\alpha}_{r}^{x}) w_{r}\\
\text{ s.t. } & \text{\cref{eq:monotone}} \text{ and } \text{\cref{eq:convexity}} \text{ and } \Co_W
\end{align*}
Note that given our choice $w_{1}=1$ and $w_{m}=0$, there are only $m-2$ variables 
(we leave $w_{1}$ and $w_{m}$ in the LP just for clarity of presentation).

The max regret $\MR(x; \pprofile, W)$ is determined by considering the pairwise regret of $x$ with all other alternatives in $A$.
Optimal alternatives w.r.t. minimax regret are the ones with least max regret. 
Observe that, whenever the $\PMR$ of an alternative $x$ (against some other alternative $y$) exceeds the best MR value found so far, we don't need to further evaluate $x$. 
This idea can be exploited further by adopting a minimax-search tree \citep{Braziunas2011}.

\section{Interactive Elicitation} 
\label{sec:elicit}
We propose an incremental elicitation method based on minimax regret.
At each step, the system may ask a question either to one of the agents about her preferences or to the chair about the voting rule. 
The goal is to acquire relevant information to reduce minimax regret as quickly as possible.
As termination condition of elicitation, we can check whether minimax regret is lower than a threshold; if we wish optimality, we can perform elicitation until minimax regret drops to zero.

The remainder of this Section is structured as follows.
First of all, we discuss the different types of questions that can be asked to the agents and to the chair, and the way responses are handled.
Then, we describe different strategies to determine informative queries to ask next, with the goal of reducing $\MMR(\pprofile,W)$ quickly.

\paragraph{Question types}
We distinguish between questions asked to the agents and questions asked to the chair.
As {\em questions asked to the agents} it is natural to consider comparison queries asking to compare two alternatives.
%Another common type of queries are {\em top-k}, asking to each agent her $k$ most preferred alternatives.
The effect of a response to a question asked to an agent is the increase in our knowledge about the agents rankings, thus augmenting the partial profile $\pprofile$. 
If agent $j$ answers a comparison query stating that alternative $a$ is preferred to $b$, then the partial order $\ppref_j$ is augmented with $a \ppref_j b$ and by transitive closure.

A bit more discussion is needed about {\em questions asked to the chair}.
Such questions aim at refining our knowledge about the scoring rule; a response gives us a constraint on the weight vector $\w$.
In particular, we want to acquire constraints of the type:
\[ w_{r} - w_{r+1} \geq \lambda (w_{r+1} - w_{r+2}) \] 
for $r \in \{1,\ldots,m-2\}$, relating the difference between the importance of ranks $r$ and $r+1$ with the difference between ranks $r+1$ and $r+2$.

\paragraph{Building concrete questions for the chair}
As we assume the weights constitute the utility components of the chair, it might be reasonable to assume
%as we do here, 
that the chair is able to answer such abstract questions in our setting. However, it is important to make sure that a question can also, in principle, be asked in a more concrete way, in terms of winners of example profiles. This permits to test how the chair understands the question and to relate the preference of the chair to her choice behavior in the economic sense. 
Furthermore this will be necessary for an ordinal extension of our work where the scores would not be considered as cardinal utilities.
Thus, our task is to build a profile, given $\lambda$ and $r ≤ m-2$, in such a way that the set of (tied) winners picked by the chair reveals whether $w_{r} - w_{r+1} \geq \lambda (w_{r+1} - w_{r+2})$.
\begin{claim}
	Given a rational $\lambda = p/q$ and a rank $r$ between $1$ and $m - 2$, the profile named $P'$ in the ensuing description is such that, for any weight vector $\w \in \W$, $a \in f(P')$ iff $w_{r} - w_{r+1} ≥ \lambda (w_{r+1} - w_{r+2})$ and $b \in f(P')$ iff $w_{r} - w_{r+1} ≤ \lambda (w_{r+1} - w_{r+2})$, where $f$ is the PSR parameterized with $\w$.
\end{claim}
\begin{proof}
Observe that the question may be defined equivalently as $q \cdot w_{r} + p \cdot w_{r+2} ≥ (p + q) \cdot w_{r+1}$, where $p, q$ are natural numbers. 
As a first attempt to make this question concrete, we could define a profile $P$ containing $p+q$ agents in such a way that an alternative $a$ receives $q$ times the rank $r$ and $p$ times the rank $r+2$, and an alternative $b$ receives $p+q$ times the rank $r+1$. 
Observing that the score of $a$ in that profile is exactly the left hand side of the question, and that the score of $b$ is the right hand side, intuition suggests that observing whether the chair picks $a$ or $b$ as winner will let us determine which side is greater; equality occurring when the chair declares $a$ and $b$ as tied for the victory. 
However, we still need to complete the profile, thus, come up with other $m-2$ alternatives defined so that each agent in the resulting profile has placed exactly one alternative in each rank.
And we must ensure, doing this, that these other alternatives are not better than $a$ or $b$: if the chair picks a different alternative $c$, it tells us that the chair prefers $c$ to both $a$ and $b$, 
but it generally does not  answer the question we are interested in. 
Taking $p = q = 1$, $m=4$, $r=2$, we see that such a completion may be impossible. Luckily, this problem can be worked around, at the price of increasing the number of agents. 

First build a temporary profile $P$ of $p+q$ agents with $a$ and $b$ ranked as just described. Complete it with $m-2$ alternatives ranked in arbitrary orders so that the resulting profile has complete strict rankings for each of the $p+q$ agent. Now we will make the other alternatives as bad as desired by adding agents to $P$ that appreciate $a$ and $b$ more than the other alternatives, thus building a profile $P'$ in which, whatever the weights, $c$ may not have a better score than $a$. Observe that if we add $\delta$ agents (with $\delta$ a natural number) that put $a$ at first rank and $b$ at second rank, and $\delta$ agents that put $b$ at first rank and $a$ at second rank, we do not change the difference of the scores of $a$ and $b$, and thus, observing the chair choosing $a$ or $b$ still answers the question.
% Choose arbitrarily a matching of the remaining $m-2$ alternatives to the ranks $3$ to $m$, say, $c \mapsto 3$, $d \mapsto 4$, and so on. The first $\delta$ supplementary agents, who rank $a$ first and $b$ second, rank the other alternatives as the just defined mapping suggests. The other $\delta$ supplementary agents, who rank $b$ first and $a$ second, use the opposite ranking for the remaining alternatives, say, $c \mapsto m$, $d \mapsto m-1$, and so on. 

We can prove that, for $\delta = p+q$, and whatever the weight vector $\w \in \W$, no alternative, except possibly $b$, has a better score than $a$. 
(In fact, as the construction will exhibit, they are all worst than both $a$ and $b$, but the weaker fact is enough for our claim.)
Pick any alternative $c$ that is not $a$ or $b$. 
%Let $c_t$ denote the number of times $c$ reaches rank $t$ or better in $P$. Note that $c_m = p+q$ and $c_t ≤ p + q, 1 ≤ t ≤ m$. Define $a_t$ similarly, for the alternative $a$. By construction, $a_r = q$, $a_{r+2} = q+p = a_m$. 
Define $c'_t$ as the number of times $c$ gets rank $t$ or better in $P'$, and define $a'_t$ similarly. Now $a'_1 ≥ \delta, a'_t ≥ 2\delta \text{ for } 2 ≤ t < m, a'_{r+2} = q+p+2\delta = a'_m$. To obtain upper bounds for $c'_t$, assume $c$ is ranked first by the $p+q$ agents in $P$. By our construction, the  $2 \delta$ new agents of $P'$ give to $c$ the ranks $(3, m)$, or $(4, m-1)$, … Observe that the score of $c$ is maximal for attribution $(3, m)$, by convexity of the weights. Therefore, in the best case for $c$, $c'_1 ≤ p + q, c'_2 ≤ p+q, c'_t ≤ p+q+\delta \text{ for } 3 ≤ t < m, c'_m ≤ p+q+2\delta$. Observe that $a'_t ≥ c'_t, 1 ≤ t ≤ m$ (with strict inequality for $t=2$). Because weights are non increasing, this guarantees, with $\delta = p + q$, that the score of $a$ is not lower than the one of $c$.
\end{proof}
At this stage, we are satisfied that a procedure exists to transform our abstract questions to questions about winners of a PSR. Further studies would investigate, for example, what is lost in terms of elicitation efficiency when we are forced to restrain to realistic concrete questions, meaning, questions involving small profiles; or investigate the relationship between our understanding of scores as (cardinal) utility and (ordinal) scores as definition of a PSR.

\paragraph{Elicitation strategies}
We develop several elicitation strategies for simultaneous elicitation of agents' preferences and of the scoring rule.
While it is of course possible to first fully elicit the agents’ preferences and afterwards elicit weights, we also want to propose interleaved approaches and compare them experimentally.
Indeed, it can be beneficial to split efforts asking questions to the chair and to agents, depending on which is estimated to be more informative.
We define here the various strategies we tested experimentally. A strategy tells us, given the current partial knowledge $(\pprofile, W)$, which question should be asked next.

The \strat{Random} strategy gives a baseline for comparison and informs about the difficulty of an elicitation problem. 
This strategy first decides with a probability of $1/2$ each whether it will ask a question about weights or a question about a preference ordering (unless the profile is already known entirely). If it opted for a question about weights, it draws one rank in $2 ≤ r ≤ m-2$ equiprobably, takes the middle of the interval of values for $\lambda$ that are still possible considering our knowledge so far, and asks whether $w_r - w_{r+1} ≥ \lambda (w_{r+1} - w_{r+2})$. The intervals are initialized to $[1, n]$. If it opted for a question to agents, it draws equiprobably among the agents whose preference is not known entirely; picks an alternative $a$ randomly with equal probability among those involved in some incomparabilities in $\ppref_j$; and picks an alternative $b$ with equal probability among those incomparable with $a$ in $\ppref_j$.

The \strat{Pessimistic} strategy selects the question that leads to minimal regret in the worst case, considering both possible answers to the question. Assume that a question leads to the possible new knowledge states $(\pprofile_1, W_1)$ and $(\pprofile_2, W_2)$, depending on the answer. Then the badness of the question in the worst case is:
\[\max_{i=1,2} \MMR(\pprofile_i,W_i) \]
This badness measure gives a way of picking questions among a set of possible questions, by picking one that minimizes ths measure of minimax regret {\em a posteriori}. 
%Note that other aggregators than $\max$ can be used as well. %We use the $\max$ operator because \citet{Cailloux2014} reported that pessimistic aggregation works much better than optimistic aggregation, although our experiments in this context suggested a weak impact of that choice on the performance of the strategy.
However, to avoid the absorption property of max, we adopt the $\leximax$ criterion as an aggregator: if the maximal $\MMR$ of two questions are equal, then it considers the other $\MMR$ values, that are the one associated to the opposite answer, preferring the question with the lowest value.
 We expect this strategy to perform very well, but only for very small problem instances: its complexity is in $O(n^2 m^5)$, because we consider $O(m^2)$ questions for each agent and need for each question to compute $\MMR$ twice, whose complexity is $O(nm^3)$. For small dimensions, it estimates an upper bound to the possible quality of a strategy.

The \strat{Limited pessimistic} strategy uses the same criterion as the pessimistic strategy, but limiting it to a small set of $n+1$ candidate questions: one per agent, and one to the chair. It uses the heuristic proposed by \citet{Lu2011} to define the candidate questions to the agents. Considering agent $j$, either $x^{*} \ppref_j \bar{y}$, or $\bar{y} \ppref_j x^{*}$, or $x^{*}$ and $\bar{y}$ are incomparable in $\ppref_j$. As shown in the proof of Claim \ref{claim:completion}, in order to increase $\PMR(x^{*},\bar{y})$, an adversary should place as many alternatives as he can above $x^{*}$ in the first case, and between $\bar{y}$ and $x^{*}$ in the second. An intuitive way to reduce the pairwise max regret is, therefore, to directly ask questions involving such alternatives, hoping for an answer that would prevent the adversary to play his best game. If $x^{*}$ and $\bar{y}$ are incomparable, then we just ask the agent $j$ to compare them.
To define the candidate question to the chair, consider
%\begin{align}
$\w^\tau = \argmin_{\w \in W} s(\bar{y}; \bar{\profile}, \w) - s(x^{*}; \bar{\profile}, \w)$,
%\end{align}
the weight vector that minimize the $\PMR$ in the worst case.
We compare $\bar{\w}$ and $\w^{\tau}$ component-wise to find the position $r$ that is most valuable to ask about (the one that maximizes $\abs{\bar{\w}_r - \w^{\tau}_r}$), and choose $\lambda$ as the middle of the interval of possible values for that rank.
We then evaluate, for each candidate question, the minimax regret for the two possible answers; finally we choose the best according to $\leximax$.

Lastly, the \strat{Two phases} strategy is developed in order to investigate the effect of varying the proportion of questions asked to agents and to the chair. %The \strat{Two phases - vc} first asks $p$ questions to the agents, then $k-p$ questions to the chair, using in both cases Limited pessimistic to select the specific question. The \strat{Two phases - cv} strategy starts with $p$ questions to the chair, then asks to the agents. \commentOC{If our XPs exhibit no significant change between vc and cv (or if we don’t have time to run them), I will drop this distinction.}
It first asks $p$ questions to the chair, then $k-p$ questions to the agents, using in both cases the same method as \strat{Limited pessimistic} to select the specific question. 
Note that when asking first only questions to the chair, if the obtained knowledge approximates well the scoring vector, then in the second part of the elicitation we fall into the more classical setting of incompleteness of preferences assuming a known voting rule. 
%And vice-versa when asking first questions to the agents, the second part of the elicitation is similar to the setting of incompletely specified scoring rule. 
%The aim of this strategy is to simulate the current state of the art when considering incomplete preferences. In fact, by asking enough questions to the chair we can reduce our problem into a well studied one of incompleteness of preferences knowing the voting rule. Moreover, if we invert to whom ask questions first, we fall instead in an already studied case of uncertainty of the voting rule with a known complete profile. 

%\item {\em Volumetric} strategy: chooses an agent $i$ and a query that maximizes the number of new pairwise preferences revealed given the worst response.

\section{Empirical Evaluation} 
\label{sec:experiments}

\sisetup{round-mode=places, round-precision=1, table-figures-integer=1, table-figures-decimal=1}
\begin{table}
	\begin{center}
		\begin{tabular}{S[table-figures-integer=2, table-number-alignment = right, table-figures-decimal=0]S[table-number-alignment = right]@{ ± }S[table-number-alignment = left]S[table-number-alignment = right]@{ ± }S[table-number-alignment = left]S[table-number-alignment = right]@{ ± }S[table-number-alignment = left]S[table-number-alignment = right]@{ ± }S[table-number-alignment = left]}
			\toprule
			{k} & {Rnd} & {sd} & {Pes.} & {sd} & {L.\ pes.} & {sd} \\
			\midrule
			0 & 5.0 & 0 & 5.0 & 0 & 5.0 & 0\\
			5 & 4.958 & 0.100 & 3.741 & 0.024 & 4.350 & 0.604\\
			10 & 4.709 & 0.436 & 3.283 & 0.423 & 3.323 & 0.374\\
			15 & 4.432 & 0.467 & 2.716 &0.410 & 2.722 & 0.672\\
			20 & 3.666 & 0.522 & 1.549 & 0.430 & 2.145 & 0.716\\
			25 & 3.068 & 0.679 & 1.424 &  0.512 & 0.921 & 0.569\\
			30 & 2.631 & 0.466 & 0.394 & 0.359 & 0.470 & 0.355\\
			\bottomrule
		\end{tabular}
	\end{center}
	\caption{Minimax regret in problems of size $(5, 5)$ after $k$ questions.}
	\label{fig:55}
\end{table}

\begin{table}
	\begin{center}
		\begin{tabular}{S[table-figures-integer=2, table-number-alignment = right, table-figures-decimal=0]S[table-number-alignment = right]@{ ± }S[table-number-alignment = left]S[table-number-alignment = right]@{ ± }S[table-number-alignment = left]S[table-number-alignment = right]@{ ± }S[table-number-alignment = left]S[table-number-alignment = right]@{ ± }S[table-number-alignment = left]}
			\toprule
			{k} & {Rnd} & {sd} & {L.\ pes.} & {sd} \\
			\midrule
			0 & 20.0 & 0 & 20.0 & 0 \\
			20 & 19.985 & 0.042 & 18.770 & 0.111 \\
			40 & 19.903 & 0.080 & 17.659 & 0.290 \\
			60 & 19.764 & 0.159 & 16.951 & 0.541 \\
			80 & 19.403 & 0.221 & 15.736 & 0.367 \\
			100 & 18.947 & 0.383 & 14.621 & 0.703 \\
			\bottomrule
		\end{tabular}
	\end{center}
	\caption{Regret in problems of size $(10, 20)$ after $k$ questions.}
	\label{fig:1020}
\end{table}

We test our strategies using randomly generated datasets. 
Our first goal is to see, with a small problem size ($m = 5, n = 5$) and the (time consuming) Pessimistic strategy, if an important lowering of the maximal regret can be achieved with a reasonable number of questions. We also want to estimate how “hard” such a problem is, by using the Random strategy as a baseline. Third, we want to estimate the loss (in terms of worst regret) when switching to the faster Limited pessimistic strategy. Fourth, we want to see, on a bigger problem size ($m = 10, n = 20$), how many questions must be asked for that strategy to achieve a significant reduction of the worst regret. Finally, we want to evaluate, thanks to the Two phases strategy, the impact of varying the proportion of questions asked to the agents (with respect to questions asked to the chair) on the reduction of the worst regret.
%We also want to see how realistic our approach is for various problem sizes, by estimating the number of questions we would have to ask before being able to reduce the worst regret significantly.

We use the following protocol.
Having picked a problem size $(m, n)$, a number of questions $k$ and a strategy, we randomly generate an “oracle”, containing the true preferences of the agents (i.e. the linear orders) and the weights associated with the chair's scoring rule. We start with empty knowledge ($\pprofile = (\emptyset_j), W = \W$) about the preference orderings of the agents or the weight differences favored by the chair. We obtain the first question to be asked using the selected strategy, as described above. We use the oracle to answer the question and update our knowledge, which is thus used to obtain the next question. This is repeated until $k$ answers have been obtained. Then we compute the resulting $\MMR$. We repeat this whole experiment 10 times, for a given $(m, n, k)$, and report the average resulting $\MMR$ and the standard deviation.

The results are displayed in \cref{fig:55,fig:1020,fig:p}, where strategies are designated by Rnd for Random, Pes. for Pessimistic, L.\ pes. for Limited pessimistic and 2 ph. for Two phases. 

We observe that the Pessimistic strategy is able to reduce the regret almost completely after $30$ questions. We also see that Limited pessimistic performs almost as well as Pessimistic; this is good news since (for $m=5$ and $n=5$), the former strategy is much faster and takes only $2.6s$ for a complete elicitation session, while the latter takes $16s$.
%However, it takes $1000$ times more time than the \strat{Random} strategy ($16s$ of the Pessimistic compared to $0.014s$ of the Random). Moreover, the results of the \strat{Limited pessimistic}, that takes almost $80\%$ less of the time of the \strat{Pessimistic} ($2.6s$), are essentially the same.
%In Table \ref{fig:1020}, we can see the drop of regret using a bigger instance.
%Finally in Table \ref{fig:p}

\begin{table}
	\begin{center}
		\begin{tabular}{S[table-figures-integer=2, table-number-alignment = right, table-figures-decimal=0]S[table-number-alignment = right]@{ ± }S[table-number-alignment = left]S[table-number-alignment = right]@{ ± }S[table-number-alignment = left]S[table-number-alignment = right]@{ ± }S[table-number-alignment = left]S[table-number-alignment = right]@{ ± }S[table-number-alignment = left]}
			\toprule
			{p} & {Two ph.} & {sd} \\%& {2 ph. vc} & {sd} \\
			\midrule
			0 & 100.0 & 0 \\%&  &  \\
			20 & 18.778 & 0.125 \\%&  &  \\
			40 & 17.648 & 0.473 \\%&  &  \\
			60 & 16.460 & 0.609 \\%&  &  \\
			80 & 15.470 & 0.597 \\%&  &  \\
			100 & 14.876 & 0.460 \\%&  &  \\
			\bottomrule
		\end{tabular}
	\end{center}
	\caption{Regret in problems of size $(10, 20)$ after $100$ questions.}
	\label{fig:p}
\end{table}

\section{Conclusions}  
\label{sec:conclusions}
In this paper we have considered a social choice setting with partial information about the agent's preferences and a partially specified voting rule.
In this setting, we have proposed the use of minimax regret both as a means of robust winner determination as well as a guide to the process of simultaneous elicitation of preferences and voting rule.
Our experimental results %on randomly generated and real world data sets 
suggest that regret-based elicitation is effective and allow to quickly reduce worst regret significantly.
%Regret-based elicitation allows to determine near-optimal winners using only few information about the agents' preferences.

As part of the contribution of this work, we publish an open-source library that allows to reproduce our experiments, and more (url not displayed for anonymity reasons).

We mention some directions for future works.
Further development of elicitation strategies, considering alternative heuristics, is an important direction. 
Second, elicitation could be extended to voting rules beyond scoring rules.
Finally, an important direction of extension aims at studying how to elicit preferences while restraining to concrete and easy questions.
% Acknowledgements: We thank the reviewers for comments helping to improve the paper. 
\bibliography{biblio}
%\bibliographystyle{named}
\bibliographystyle{abbrvnat} 
\end{document}

\appendix
\section{Old table}
\begin{table}
	\begin{center}
		\begin{tabular}{S[table-figures-integer=2, table-number-alignment = right, table-figures-decimal=0]S[table-number-alignment = right]@{ ± }S[table-number-alignment = left]S[table-number-alignment = right]@{ ± }S[table-number-alignment = left]S[table-number-alignment = right]@{ ± }S[table-number-alignment = left]S[table-number-alignment = right]@{ ± }S[table-number-alignment = left]}
			\toprule
			{k} & {Rnd} & {sd} & {T. ph.} & {sd} & {Pes.} & {sd} & {Pes.} & {sd} \\
			\midrule
			0 & 4.0 & 0 & 4.0 & 0 & 4.0 & 0 & 4.0 & 0\\
			5 & 3.40 & 0.40 & 3.86 & 0.13 & 2.63 & 0.28 & 2.63 & 0.28\\
			10 & 2.59 & 0.75 & 2.63 & 0.39 & 1.29 & 0.50\\
			15 & 2.37 & 0.33 & 0.92 & 0.76 & 0.43 & 0.50\\
			20 & 1.23 & 1.04 & 0.15 & 0.23 & 0.09 & 0.27\\
			25 & 0.90 & 0.80 & 0 & 0 & 0.25 & 0.75 \\
			30 & 0.34 & 0.57 & 0.02 & 0 & 0 & 0\\
			\bottomrule
		\end{tabular}
	\end{center}
	\caption{Minimax regret in problems of size $(4, 4)$ after $k$ questions.}
	\label{fig:44}
\end{table}
See \cref{fig:44}.

\section{Bounding differences of weights}
We want to check whether we can round PMR values to (for example) $10^{-4}$. Thus, we want to bound the difference of scores (unless it is zero).

Consider $s(a) = w_2 + w_4$, $s(b) = w_3 + w_3$, $w_1 = 1, w_2 = w_3 + \epsilon, w_4 = 0$, with $w_3$ and $\epsilon$ two small positive numbers. Then convexity is satisfied and $s(a) - s(b) = \epsilon$.

Note that in this example, $w_3 ≤ \epsilon$, thus the difference of scores is indeed at least the smallest weight!

Consider $w_1 = 1, w_2 = 0.45, w_3 = 0.1, w_4 = 0$, $s(a) - s(b) = (w_2 - w_3) -3 (w_3 - w_4) = 0.35 - 0.30 = 0.05 < w_3$. 

Problem: we can make the difference of scores very small, even with $n=2$ agents and $m=3$ alternatives and while satisfying convexity. Consider $w_1 = 1, w_2 = \frac{1-\epsilon}{n}, w_3 = 0$, $s(a) - s(b) = (w_1 - w_2) - (n-1) (w_2 - w_3) = w_1 - n w_2 = \epsilon$. Choose $\epsilon < \frac{1}{n+1}$ so that $w_2 > \epsilon$ and $n ≥ 2$ to satisfy convexity. For example, $\epsilon = 1/1000, n = 2, w_2 = 999/2000$.


\section{Minimax Computation under Convex Assumption} 
Refer to \citep{Lu2011}
The goal is to choose as a winner the alternative $x^*$ whose worst case loss is minimal under all possible realizations of the full profile and all possible choices of weights. 
Assume hereinafter the selected weights sequence $\w \in W$ to be convex. 
In order to compute the minimal max regret $\MMR(\pprofile)$ under partial profile $\pprofile$ we need to compute the pairwise max regret between all pairs of alternatives $(x,y)$, where $x$ is a proposed winner and $y$ is the ``adversary'' alternative. Indeed, the construction of $\PMR(x,y,\pprofile,\w)$ can be viewed as an adversary's attempt to maximize the regret of choosing $x$ instead of $y$. 
For doing this, he can choose a completion $\profile_i \in C(\pprofile_i)$ of the partial profile and a (feasible) scoring vector $\w$ that maximize the contribution of the agent $i$ to $\PMR(x,y,\pprofile,\w)$. Let us now analyze how it could be done depending on the relation between alternatives $x$ and $y$ in $\pprofile_i$. 
\begin{itemize}
	\item $x \succ_i^\pprofile y$
	\newline If we know $x$ is preferred to $y$ and we choose $x$ as a winner, $\pprofile_i$ contribution to $\PMR(x,y,\pprofile,\w)$ must be negative. In this situation, our adversary can only try to minimize this advantage by minimizing the positional gap between the two alternatives. To achieve that, he can arbitrary place all the alternatives preferred to $x$ above $x$, together with all the ones with unknown relation to $x$. Moreover, he can place all the alternatives less preferred to $x$ and with unknown relation to $y$ below $y$. We can summarize it for each $q \in A$ as follows:
	\begin{align*}
	q \succ_i^\pprofile x \vee q \ ?_i^\pprofile \ x \ & \Rightarrow \ \uparrow_x \\
	x \succ_i^\pprofile q \wedge ( q \ ?_i^\pprofile \ y \vee y \succ_i^\pprofile q) \ & \Rightarrow \ \downarrow_y \\
	x \succ_i^\pprofile q \succ_i^\pprofile y \ & \Rightarrow \ \text{in between} \\
	\end{align*}
	It is worth noting that when the relation between $q$ and $x$ is not known in the partial profile, the adversary takes advantage by placing $q$ above $x$ only under the assumption of convex weight sequences.
	\item $y \succ_i^\pprofile x$
	\newline If $y$ is preferred to $x$ the construction proceeds similarly to the previous case, but now the adversary takes advantage by maximizing the gap between $x$ and $y$ placing as much alternatives as he can between the two. We can summarize the procedure for each $q \in A$ as follows:
	\begin{align*}
	q \succ_i^\pprofile y \ & \Rightarrow \ \uparrow_y \\
	x \succ_i^\pprofile q \ & \Rightarrow \ \downarrow_x \\
	(y \succ_i^\pprofile q \vee y \ ?_i^\pprofile \ q) \wedge (q \succ_i^\pprofile x \vee q \ ?_i^\pprofile \ x) \ & \Rightarrow \ \text{in between} \\
	\end{align*}
	\item $x \ ?_i^\pprofile \ y$
	\newline If the partial profile $\pprofile_i$ does not specify the relation between $x$ and $y$, the advantage is maximized by ordering $y$ over $x$ and maximizing the gap between them following the procedure for the case $y \succ_i^\pprofile x$.
\end{itemize}

\section{Dropping the Convex Assumption}
\subsection{Profile completion}
What if the sequence of weights is not convex? When $y \succ_i^\pprofile x$ or $x \ ?_i^\pprofile \ y$ weights do not influence the arbitrary placement of alternatives. Please remind we are working under the assumption that weights constitute a monotonic non-increasing sequence. Thus, there is no way for the adversary to take advantage from the weights distribution in order to increase the gap between $y$ and $x$ besides placing as much alternatives as he can between the two. The only case in which weights can influence the positional gap between $x$ and $y$ is when $x \succ_i^\pprofile y$ and $q \ ?_i^\pprofile \ x$. For convex sequences we place such alternatives $q$ above $x$, but it is not obvious that this is the best option for other sequences. For example, suppose $x$ and $y$ are ranked respectively in first and second position in the partial profile and we wonder where to place an alternative $q$ with unknown relation to $x$ (and thus to $y$). Suppose also that in the weight sequence the distance between the first and second positions is much lower than the one between the second and the third ones. In this case, placing $q$ above $x$ does not minimize the gap between $x$ and $y$ but we want, instead, to place $q$ below $y$.
\newline The constraints expressed by the chair may result in a set of feasible vectors such that none of them forms a convex sequence. In this case we need to analyze the particular sequence of weights in order to decide how to maximize the adversary advantage. Before going into details, let us define $A$ as the set of alternatives (if any) preferred to $x$, $B$ as the set of alternatives preferred to $y$ but not to $x$, and $U$ the set of those with unknown relation to both $x$ and $y$. The idea is to determine the positions that minimize $x$'s advantage over $y$ and then place some of the alternatives in $U$ above $x$ and some below $y$ in order to get that desired ranking. Since we cannot change the order of the alternatives in the set $B$ we know that $x$ and $y$ are separated exactly by $|B|$ positions (the adversary would not take any advantage by adding alternatives between them). So, starting from the position of $x$ in the partial completion of $\pprofile_i$ computed so far ($\hat{\profile}_i$), we find the two positions separated by $|B|$ alternatives whose weights difference is the lowest. Note that we can only add $|U|$ alternatives so we can check only until the position $\hat{\profile}_i(x)+|U|$. Algorithm \ref{alg:splittingU} shows the procedure described.

It is easy to see that we check at most $|U|$ positions. In the worst case the size of $U$ is equal to $m-2$, thus the procedure can be computed in $O(m)$ time. This cost does not affect the minimax regret computation time complexity that remains $O(nm^3)$.

\begin{algorithm}[h] 
	\caption{Placing alternatives in $U$ without Convex Assumption}
	\label{alg:splittingU} 
	\begin{algorithmic}
		\Require $x$, $y$, $\hat{\profile}_i$, $\w$, $U$, $B$
		\Ensure $\profile_i \in C(\pprofile)$
		\Statex
		\State $ j \gets 0$;
		\State $ i \gets \hat{\profile}_i(x)$;
		\State $ \mathit{posmin} \gets i$;
		\State $ \mathit{min} \gets \w(i) - \w(i+1+|B|)$;
		\While {$( j \leq |U| )$}
		\If{ $(\w(i+j)-\w(i+1+|B|+j) < \mathit{min})$ }
		\State $ \mathit{min} \gets \w(i+j) - \w(i+1+|B|+j)$;
		\State $ \mathit{posmin} \gets i+j$;
		\EndIf
		\EndWhile
		
		\State $U_{\mathit{abovex}} \gets (i-\mathit{posmin}) \text{ alternatives} \in U $;
		\State $U_{\mathit{belowy}} \gets U \setminus U_{\mathit{above}}$;
		\Statex
		\State $\profile_i \gets place(\pprofile_i,U_{\mathit{abovex}},U_{\mathit{belowy}})$;
		\Statex \Return $\profile_i$
		
	\end{algorithmic}
\end{algorithm}

\section{Minimax regret without the convex assumption}
Without the convex assumption, we cannot use $\hat{v}$ for the agents in $U^{-}$, but only for agents in $U^{+}$ and $U^{?}$.
%Let $\hat{v}_i$ be the linear order extending $\succ^{p}_i$ according to the above procedure.
Then PMR can be written as follows:
\begin{align*}
 &\PMR(x,y; \pprofile, W) =\\ 
 &\max_{\w \in W} \Big \{ \sum_{j \in U^-} [\max_{v_j \in C(\succ_j^p)} [w_{v_j(y)} \!-\! w_{v_j(x)}]] 
  \!+\!	 \sum_{j \in A^+ \cup  U^?} [w_{\hat{v}_j(y)} \!-\! w_{\hat{v}_j(x)}] \Big \} 
 \end{align*}
%Note that the second addendum inside the max do not depend on the choice of $\w$.
Consider the two addenda inside the ``max''. 
The first addendum is concerned with positioning of alternatives $x$ and $y$ for agents in $U^{-}$ for which we know that $x$ is preferred to $y$.
The second addendum is concerned with agents in $U^{+}$ and $U^{?}$.
We rewrite the second addendum as:
\begin{align*}
\sum_{j \in U^+ \cup  U^?} [w_{\hat{v}_j(y)} \!-\! w_{\hat{v}_j(x)}] 
= \sum_{i = 1}^{m} (\hat{\alpha}_{i}^{y} - \hat{\alpha}_{i}^{x}) w_{i}
\end{align*}
where $\hat{\alpha}_{i}^{x}$ is the number of times that $x$ is ranked  in position $i$  considering the profile $\hat{v}$ of agents in $U^+ \cup  U^?$.
%We compute pairwise maximum regret by considering binary variables $\{ B_{i} \}_{i=1,\ldots,m}$ to represent optimization choices related to where to position the alternatives.

We now address the agents in $A^{-}$
For a given $j$, let $\beta$ be the number of alternatives that are incomparable with $x$ and $y$:
\[ \beta_{j} = \mid \{ c : c \prefinc_{j} y \wedge c \prefinc_{j} x \} \mid \]
$x$ can be ranked between $t_{1}(j)=\mid \{ c \in A : c \ppref_{j} y \wedge x \nppref_{j} c \} \mid $ and position $t_{2}(j)=t_{1}(j)+\beta_{j}$.
The completion for agent $j$ is such that the positions of $x$ and $y$ differ of exactly $\gamma_{j} =
\mid \{ c \in A : x \ppref c \ppref y \} \mid$ positions.


We now show how to optimize $\PMR$.
In addition to variables $\{ w_{j} \}_{j=1,\ldots,m}$ (one for each position) we need to employ several additional decision variables.
We introduce two sets of binary variables $B^{+}_{i,j}$ and $B^{-}_{i,j}$  for each position $i$ and for each agent $j$.
Variable $B_{i,j}^{+}$ encodes the fact that the alternative $y$ is placed in position $i$ in the ranking of agent $j$; while  $B_{i,j}^{-}$ encodes the same thing for alternative $x$.
%We also have numerical variables to represent the weights of the scoring rule.
Since each alternative needs to be placed exactly in one place for each agent, we adopt the constraints
$\sum_{i=t_{1}(j)}^{t_{2}(j)} B_{i,j}^{+} = 1$.
Since we know that $x$ and $y$ are ranked $\gamma_{j}$ positions apart, we set the constraint:
$B_{i+\gamma_{j},j}^{-} \geq B_{i,j}^{+}$,  for $i = \{ t_{1}(j), \ldots, t_{2}(j)\}$.

% and $\sum_{i=1}^{m} B_{i,j}^{-} = 1$.
%Since the objective is to maximize pairwise regret...

The score of alternative $y$ can be written as $\sum_{i = 1}^{m} \hat{\alpha}_{i}^{y}  w_{i} + \sum_{i=1}^{n} \sum_{j=1}^{m} w_{j} B_{i,j}^{+}$.
The objective function is now:
 \[ \max \sum_{i = 1}^{m} (\hat{\alpha}_{i}^{y} - \hat{\alpha}_{i}^{x}) w_{i} +  \sum_{i=1}^{n} \sum_{j=1}^{m} (B_{i,j}^{+} - B_{i,j}^{-})  w_i \]

We use integer programming enconding tricks in order to linearize the problem.
We introduce yet another set of variables  $V_{i,j}^{+} $  and $V_{i,j}^{-}$ % the multiplicative terms by new variables.
and we enforce that $V_{i,j}^{+} = B^{+}_{i,j} w_i$ by setting constraints $V_{i,j}^{+} \leq B^{+}_{i,j}$ and $V_{i,j}^{+}  \leq w_i$.
We have similar constraints for enforcing $V_{i,j}^{-} = B^{-}_{i,j} w_i$.

We therefore obtain the following mixed integer linear program:
\newcommand{\C}{\mathcal{C}}
\begin{align}
\max & \sum_{i=1}^m  [(\hat{\alpha}_{i}^{y} - \hat{\alpha}_{i}^{x}) w_{i}] +
  \sum_{j \in A^{-}} \sum_{i=t_{1}}^{t_{2}}  [V_{i,j}^{+} - V_{i,j}^{-}]
\end{align}
\begin{align}
\text{ s.t. } &  \text{Equation } (\ref{eq:monotone}) & \\
&  \C(\w) &  \\
& \sum_{i=t_{1}}^{t_{2}} B_{i,j}^{+} = 1 & \forall j \in A^{-} \\
& B_{i+\gamma_{j},j}^{-} \geq B_{i,j}^{+} & \forall i \in \{ t_{1}, \ldots t_{2}\}, \forall j \in A^{-} \\
& V_{i,j}^{+} \leq B_{i,j}^{+}  & \forall i \in \{ t_{1}, \ldots t_{2}\}, j \in A^{-} \\
& V_{i,j}^{+} \leq w_i & \forall i \in \{ t_{1}, \ldots t_{2}\}, j \in A^{-} \\
& V_{i,j}^{-} \geq w_{i} + B_{i,j}^{-} - 1 & \forall i \in \{ t_{1}, \ldots t_{2}\}, j \in A^{-}\\
& V_{i,j}^{-} \geq 0 & \forall i \in \{ t_{1}, \ldots t_{2}\}, j \in A^{-}
\end{align} 
%We write $w \in W$ as a shourtcut to represent the constraints that the weights are chosen to be in the feasible set.
There are (at most) $nm$ binary variables and $m(n+1)$ numerical variables.
The optimization program can be solved by any suitable MILP solver, although it is not suitable to large problem instances.

% DONT KNOW IF WE HAVE TO FORMALIZE THIS AS A CLAIM
%\begin{claim}
%The $\PMR$ is computed using the above optimization problem.
%\end{claim}

\section{Considerations on weights}
\label{sec:weights}
Let us consider a monotonic non-increasing sequence of weights: $w_{1} \geq w_{2} \geq \ldots \geq w_{m}$. Without loss of generality, we can assume $w_1=1$ and $w_m=0$.

\begin{claim}
	\label{clm:wsequence}
	If the weight sequence is convex then $w_{1} > w_{2}$.
	\[\forall i \in \{1,\ldots,m\} \;\; w_i - w_{i+1} \geq w_{i+1}-w_{i+2} \Rightarrow w_{1} > w_{2} \geq \ldots \geq w_{m}\] 
\end{claim}
\begin{proof}
	By contradiction let assume $w_{1} = w_{2}$ then 
	\begin{align*}
	w_{1} - w_{2} \geq w_{2} - w_{3} &\geq \dots \geq w_{m-1} - w_{m} \\
	1 - 1 \geq 1 - w_{3} &\geq \dots \geq w_{m-1} - 0 \\
	0 \geq 1 - w_{3} &\geq \dots \geq w_{m-1}
	\end{align*}
	At this point either $0\leq w_{3}<1$ or $w_{3}=1$. In the first case 
	\[0 \ngeq 1 - w_{3}\]
	This breaks the convexity assumption so it is impossible.
	In the second case, by definition there is a $w_{i} \neq 1$ where $2 < i \leq m$. So it would be 
	\[0 \ngeq 1 - w_{i}\]
	for some $i$. Again, the convexity is not satisfied.
\end{proof}

\begin{corollary}
	\label{cor:weq}
	If the weight sequence is convex and $w_{i} = w_{i+1}$ for some $i$, then $w_{j}=0 \ \forall \
	j=i, \dots m$.
\end{corollary}


\section{Querying the chair}
Suppose our query strategy suggests us to ask the chair the following query:
\[ w_{2} - w_{3} \geq 2(w_{3} - w_{4}) \]
Then we can transform it in:
\begin{align}
\label{eqn:juryquery}
w_{2} - w_{3} &\geq 2 \cdot w_{3} - 2 \cdot w_{4} \notag \\
w_{2} + 2 \cdot w_{4} &\geq 3 \cdot w_{3} 
\end{align}
and ask the chair if they would prefer as a winner an alternative ranked first one time and third three times rather than an alternative ranked second four times.

Another way to query the chair, a more concrete one, is to present them a profile representing the situation described by the query and then deduce its answer from the selected winner. Obviously we have to be sure to choose a profile where only the alternatives we are interested in could be pick as winners. Therefore, we need a systematic way to construct a profile that reflects the situation outlined by the query for two alternatives and the others are not better than them.


Assume we have a profile of $3$ agents ranking $4$ alternatives. We can represent it by columns where each of them represents the preference ordering of one agent.
\[
\begin{array}{ccc}
v_1
& v_2
& v_3 \\
\midrule 
c
& d
& c \\
a
& c
& d \\
b
& b
& b \\
d
& a
& a \\
\end{array}
\]

Assuming anonymity, we can also write the profile expressing for each alternative $i \in A$ the number of agents placing it at a given rank.

\[
\begin{array}{ccccc}
& 1^\circ
& 2^\circ
& 3^\circ
& 4^\circ \\
\cmidrule{2-5}
a 
& 0
& 1
& 0
& 2 \\
b
& 0
& 0
& 3
& 0 \\
c
& 2
& 1
& 0
& 0 \\
d
& 0
& 1
& 1
& 1 \\
\end{array}
\]

Recalling the query (\ref{eqn:juryquery}) we are considering as an example, it is easy to see that in the current profile alternatives $a$ and $b$ represent the situation as described by the query. Therefore, if after proposing this profile to the chair the winning alternative turns out to be $a$ then we know that $w_{2} - w_{3} > 2(w_{3} - w_{4})$; if, instead, the winner is $b$ we know that $w_{2} - w_{3} < 2(w_{3} - w_{4})$ and if both alternatives are picked as winners then $w_{2} - w_{3} = 2(w_{3} - w_{4})$. 

Nevertheless, in this example it is clear that $c$ will always be preferred to other alternatives (see Claim \ref{clm:wsequence} in Section \ref{sec:weights}). To see it we can express for each alternative the number of agents placing it at a given rank or at a higher one.

\[
\begin{array}{ccccc}
& \geq 1^\circ
& \geq 2^\circ
& \geq 3^\circ
& \geq 4^\circ \\
\cmidrule{2-5}
a 
& 0
& 1
& 1
& 3 \\
b
& 0
& 0
& 3
& 3 \\
c
& 2
& 3
& 3
& 3 \\
d
& 0
& 1
& 2
& 3 \\
\end{array}
\]


\begin{claim}
	Consider a set $A$ of $m$ alternatives and let $r_k(i)$ be the number of times the alternative $i$ obtains a rank $k$ or a higher one. Consider two alternatives $ i,j \in A$, if $\forall \ k=2, \dots,m \ r_k(i)\geq r_k(j)$ and $r_1(i) > r_1(j)$ then $i$ is always preferred to $j$ for any choice of weights.
\end{claim}

\begin{proof}
	For our assumptions we know the sequence of weights is monotonic non-increasing and convex. Moreover, for the Claim \ref{clm:wsequence} in Section \ref{sec:weights}, we know that $w_1 > w_2$. Therefore, even in the worst case where $r_k(i) = r_k(j) \ forall \ k=2, \dots,m$ the sum of weights for alternative $i$ cannot be lower than the one for $j$. It is worth noting that we cannot say anything when $r_1(i) = r_1(j)$. Indeed, as said in Corollary \ref{cor:weq}, all the weights for other position but the first one can be equal, thus the number of agents ranking the alternatives at a certain position does not matter anymore.	
\end{proof}

So the strategy for querying the chair is to use two alternatives $i$ and $j$ to represent the query and complete the profile such that the other alternatives are all dominated by them. An algorithmic approach is to start from the initial profile and then add as many agents as needed that rank $i$ and $j$ as their top choices and the other alternatives afterwards. As an example let consider again the query \ref*{eqn:juryquery}. We want the committee to choose between $a$ or $b$:

\[
\begin{array}{ccccc}
& 1^\circ
& 2^\circ
& 3^\circ
& 4^\circ \\
\cmidrule{2-5}
a 
& 0
& 1
& 0
& 2 \\
b
& 0
& 0
& 3
& 0 \\
\end{array}
\]

We add agents in order to increase the number of times $a$ and $b$ are ranked at first position. Please note we must maintain these scheme, so every add of a position to the ranking of $a$ must correspond to the same in the ranking of $b$.

\[
\begin{array}{cccccc}
v_1
& v_2
& v_3 
& v_4
& v_5
& v_6 \\
\midrule 
a
& b
& c 
& d
& a
& b \\
c
& a
& d
& c
& b
& a \\
b
& d
& b
& b
& d
& c \\
d
& c
& a
& a
& c
& d \\
\end{array}
\]

\[
\begin{array}{ccccc}
& 1^\circ
& 2^\circ
& 3^\circ
& 4^\circ \\
\cmidrule{2-5}
a 
& 2
& 2
& 0
& 2 \\
b
& 2
& 1
& 3
& 0 \\
c
& 1
& 2
& 1
& 2 \\
d
& 1
& 1
& 2
& 2 \\
\end{array}
\]

\[
\begin{array}{ccccc}
& \geq 1^\circ
& \geq 2^\circ
& \geq 3^\circ
& \geq 4^\circ \\
\cmidrule{2-5}
a 
& 2
& 4
& 4
& 6 \\
b
& 2
& 3
& 6
& 6 \\
c
& 1
& 3
& 4
& 6 \\
d
& 1
& 2
& 4
& 6 \\
\end{array}
\]
\textbf{Remarks:}
\begin{itemize}
	\item We are considering $\lambda \in \mathbb{N} \setminus \{0\}$ but we could be interested in a real number. \textit{Solution}: we can multiply both sides.
\end{itemize}
\fi
\end{document}  
