\newif\ifijcai
 \ijcaitrue  %UNCOMMENT this to have the ijcai formatting / COMMENT for draft format
 
\newif\ifappendix
%\appendixtrue %COMMENT to remove appendix % UNCOMMENT to add apendix
\ifijcai
 	%\typeout{IJCAI-19 Instructions for Authors}

% These are the instructions for authors for IJCAI-19.

\documentclass{article}
\pdfpagewidth=8.5in
\pdfpageheight=11in
% The file ijcai19.sty is NOT the same than previous years'
\usepackage{ijcai19}

% Use the postscript times font!
\usepackage{times}
\usepackage{soul}
\usepackage{url}
\usepackage[hidelinks]{hyperref}
\usepackage[utf8]{inputenc}
\usepackage[small]{caption}
\usepackage{graphicx}
\usepackage{amsmath}
\usepackage{booktabs}
%\usepackage{algorithm}
%\usepackage{algorithmic}
\urlstyle{same}

% the following package is optional:
%\usepackage{latexsym} 

 
	\author{Beatrice Napolitano$^1$\and Olivier Cailloux$^1$	\and  Paolo Viappiani$^{2}$
	\affiliations $^1$ LAMSADE, UMR 7243, CNRS and Universit\'e Paris Dauphine, PSL Research University, Paris, France\\ $^2$ LIP6, UMR 7606, CNRS and Sorbonne Universit\'e, Paris, France
	\emails \{first, second\}@lamsade.fr, paolo.viappiani@lip6.fr
}
\else
	\documentclass[12pt]{article}
	\usepackage[top=2cm, bottom=2cm, left=2cm, right=2cm]{geometry}
	\geometry{a4paper}
	\usepackage{graphicx}
	%\usepackage{authblk}
	\usepackage{booktabs}
	\usepackage{algorithm, algpseudocode}
	\algrenewcommand\algorithmicrequire{\textbf{Input:}}
	\algrenewcommand\algorithmicensure{\textbf{Output:}}
	\author{Various Authors}
\fi

\usepackage{algorithm, algpseudocode}
\usepackage{amsmath,amssymb,enumerate,amsthm}
\usepackage{hyperref}
\usepackage{mathrsfs}

\usepackage{bm}
\usepackage{cleveref}
\usepackage{xcolor}


\newcommand{\email}[1]{\href{mailto:#1}{#1}}
\newcommand{\denselist}{\itemsep -2pt\topsep-6pt\partopsep-6pt}
\newcommand{\commentOC}[1]{\textcolor{blue}{\small$\big[$OC: #1$\big]$}}

%\newcommand{\rank}{\text{rank}}
\newcommand{\rank}{v}
\newcommand{\preflarge}{\boldsymbol{\succeq}^\textbf{r}}%real, complete pref
%\newcommand{\pref}{\boldsymbol{\succ}^\textbf{r}}%real, connected pref, strict
\newcommand{\pref}{\succ}%real, connected pref, strict
\newcommand{\prefr}{{\succ}^\text{r}}%real, connected pref, strict
\newcommand{\ppreflarge}{\succeq^\text{p}}%partial pref
\newcommand{\ppref}{\succ^\text{p}}%partial pref
\newcommand{\nppref}{\nsucc^\text{p}}%negated partial pref

%Thanks to https://tex.stackexchange.com/q/154549
\makeatletter
\newcommand{\newrelation}[2]{% #1 = control sequence, #2 = replacement text
  \@ifdefinable{#1}{%
    \def#1{%
    \@ifnextchar_{\csname\string#1\endcsname}{\mathrel{#2}}%
    }%
    \@namedef{\string#1}##1##2{\mathrel{#2_{##2}}}%
  }%
}
\makeatother

\newrelation{\myRgood}{R^X}
\newrelation{\pinc}{\!\parallel\!}%partial pref, complement (incomparable)
%\newrelation{\pinc}{Q^\text{p}}%partial pref, complement (incomparable)

\newcommand{\profile}{\textbf{v}}%(complete) profile
\newcommand{\pprofile}{\textbf{p}}%partial profile
\newcommand{\w}{\textbf{w}}%partial profile
\newcommand{\C}{\mathcal{C}}%partial profile


\DeclareMathOperator{\Regret}{Regret}
\DeclareMathOperator{\SCORE}{Score}
\DeclareMathOperator{\PMR}{PMR}
\DeclareMathOperator{\MR}{MR}
\DeclareMathOperator{\MMR}{MMR}


\newtheorem{claim}{Claim}
\newtheorem{prop}{Proposition}
\newtheorem{corollary}{Corollary}
\newtheorem{definition}{Definition}
\newtheorem{example}{Example}


\title{Robust Winner Determination and Simultaneous Elicitation of Scoring Rules and Preferences}
%\author{Olivier Cailloux  \and Beatrice Napolitano \and Paolo Viappiani} 



%\affil{LIP6\\\email{paolo.viappiani@lip6.fr}
%\author{Olivier Cailloux} 
%\author{Stefano Moretti}%\affil{LAMSADE}
%\author{Beatrice Napolitano}\affil{LAMSADE}

\begin{document}
\maketitle
% aggregate different opinions

%Given a full profile: use linear programming to maximize regret
%Given a partial profile: use techniques of Lu and Boutilier (2011).
\begin{abstract}
Social choice deals with the problem of determining a consensus choice from the preferences of different agents (voters).
In the classical setting, the voting rule is fixed beforehand and full preference information is provided by the voters. 
Recently, the assumption of full preference information has been questioned by a number of researchers and several methods for eliciting voters information have been proposed.

In this paper we go one step further and we assume that both the voting rule and the voters preferences are partially specified.
Focusing on positional scoring rules, we assume that the chair, while  not able to give a precise definition of the rule, is capable of answering simple questions  requiring to pick a winner from a specific  example profile.
Moreover, the preferences of the voters are incrementally acquired by asking comparison queries.

We propose a method  for robust approximate winner determination in this setting with minimax regret. 
We then provide an interactive elicitation protocol based on minimax regret
and develop several query strategies that interleave questions to the chair and questions to the voters in order to acquire the most relevant information in order to quickly converge to optimal or a near-optimal alternative.
\end{abstract}

\section{Introduction}
% indeed many works analyze the properties of different rules (including axiomatic treatments) in order to justify the choice of specific social choice functions. 
%Moreover, it is usually assumed that the preferences of the voters are known completely. 


%General motivation: need to deal with partially specified preferences and as well partially specified voting rules
Aggregation of preference information is a central task in many computer systems (recommender systems, search engines, etc).
In many situations, such as in group recommender systems, the goal is to find a consensus choice.
It is therefore natural to look at methods from social choice and see how they can be adapted for group decision-making in a computerized setting.

The traditional approach to social choice assumes that both the social function and the full preference orderings of the voters are expressed beforehand. 
As noted by several authors requiring voters to express full preference orderings can be prohibitively costly (in term of cognitive and communication cost), especially for decisions with large sets of alternatives.
This observation has motivated a number of recent works considering social choice with partial preference orders  \cite{Xia2008,Pini2009,Konczak05} and incremental elicitation \cite{Kalech2011,Lu2011,Naamani-Dery2015} of voters preferences. 

On the other hand, in several situations it may not be easy to precisely define the voting rule.
Indeed, it is possible that the chair may have some preferences over the desired aggregation method, but may still be not able to formalize the voting rule upfront.
The work of Cailloux and Endriss \cite{Cailloux2014} provides elicitation methods  for a quite general class of rules based on well orders.
When considering positional scoring rules, several authors \cite{Stein1994,Llamazares2013,Viappiani2018} have worked on  positional scoring rules with uncertain weights (assuming that the preferences of the voters are fully known).
%it is possible to derive dominance relations (akin to stochastic dominance) that allow to eliminate some alternatives since they will be less preferred than another one for any instantiation of the weights \cite{Stein1994}.
%Among others, Llamazares and Pe{\~{n}}a have considered  the problem of dealing with underspecified weights in positional scoring rules.

In this paper we consider that both preferences and the social choice rule can be only partially specified.
We develop methods for computing the minimax-optimal alternative for scoring rules with partial preference information and partial information about the weights of the scoring rule.
While previous works have considered either partial information about the voters preferences or a partially specified aggregation method, we do not know of any work considering both sources of uncertainty at the same time.

%We consider that the committee wants to express preferences for some voting rules
% Incremental elicitation of preferences is critical to easing the cognitive and communication cost for users 
%In this paper we provide a method for approximate winner determination and an incremental elicitation protocol based on minimax regret. 
%In this paper we relax the traditional assumptions.
%Incremental elicitation of preferences can help mitigating the cognitive and communication costs

In this work, we focus on positional scoring rules, that  are a particularly common method used to aggregate rankings and determine  a winner.
We also address the problem of elicitation, providing incremental elicitation methods to acquire relevant preference information; we discuss  several heuristics that allow us to determine queries that quickly allow to reduce minimax regret.
In particular, our query strategies  focus simultaneously on reduction of relevant preference and voting rule uncertainty.
% the value of certain preference information is often not worth the cost of obtaining it.
% We must engage in simultaneous elictation of the rule and of the preferences.

%In this paper we propose an interactive adaptive protocol for eliciting both the voters preferences and the voting rule.
%At each step we need to decide whether we want to ask a question to the committee or to one of the agents (and to which agent in particular).
The paper is organized as follows.
In Section \ref{sec:background} we provide the necessary background.
We introduce the minimax criterion and in Section \ref{sec:mmr}.
In Section \ref{sec:elicit} we provide an interactive elicitation protocol based on minima regret;  in Section \ref{sec:experiments} we present the empirical validation of our approach with simulations, and finally in Section \ref{sec:conclusions} we provide some final thoughts.



\section{Social choice with partial information}\label{sec:background}

We now introduce some basic concepts.
We assume that there is a set $A$ of $m$ alternatives (for instance products, restaurants, movies, public projects, job candidates, etc.) and $n$ agents (voters); each agent is associated to a preference order.
The goal of the voters is to make a consensus choice, in order to select a ``winner'' given voter rankings.

%The set of preference orders is called profile and it is denoted by $v$.
The preferences of the agents are supposed to be linear orders (connected, transitive, asymmetric relations) involving the alternatives;
$\pref_i$ denotes the ``real'' preference relation of agent $i$. 
The set $(\pref_1,\ldots,\pref_n)$ is known in the social choice literature as the {\em preference  profile}.
The profile is equivalently represented by $\profile=(v_1,\ldots,v_n)$ where $v_i(j)$ denotes the rank (position) of alternative $j$ in the preference order $\pref_i$. 
With a little abuse of notation, we will use the term profile to refer to either $\profile$ or to the preference relations, depending on the context.

Let $V$ be the set of possible preference profiles (the cartesian product of $n$ linear orders).
A social choice function $f : V \rightarrow 2^A$ associates a profile with a set of winners.
Among the many possible social choice functions, we consider {\em positional scoring rules}, which attach weights to positions according to  the vector $(w_1,\ldots,w_m)$ (also called the scoring vector).
%A scoring rule associates each alternative to a score that is given by the sum of points obtained for each voter.
An alternative obtains a score that depends on the rank obtained in each of the preference orders:
\begin{align}
s(x; \profile, \w) = \sum_{j=1}^{n} w_{v_j(x)}
= \sum_{i=1}^{m} \alpha^{x}_i w_i . \label{eq:srule}
\end{align}
where $\alpha^{x}_i$ is the number of times that alternative $x$ was ranked in the $i$-th position.
The winner is the alternative with highest score.
%We assume that the weights constitute a monotonic sequence: $w_{1} \geq w_{2} \geq \ldots \geq w_{m}$.

In this work we want to reason about partial preference information.
A partial preference  is encoded by a partial order $\ppref_j$  of voter $j$; in this work we assume that preference information is truthful, i.e. $a \ppref_j b \implies a \pref_j b$.
We use $\pinc$ to denote incomparability, that is $a \pinc_{j} b$ iff $a \nppref_j b \wedge b \nppref_j a$.
An incomplete profile is a set of partial votes  $\textbf{p}=(\ppref_1,\ldots,\ppref_n)$.

A completion of $\ppref_i$ is any linear order $\pref_i$ that extends $\ppref_i$.
Let $C(\ppref_i)$ be the set of completions of $\ppref_i$, that is the set of all complete rankings that extend $\ppref_i$.
We let $C(\textbf{p})=C(\ppref_1)\times \ldots \times C(\ppref_n)$ be the set of complete profiles extending $p$.

\medskip
We also assume that the weights of the scoring rule are only partially specified.
Therefore the vector $(w_1,\ldots,w_m)$  is not known but we are given a set of constraints restraining the possible values that weight can take.
We consider a non-increasing sequence of weights:
\begin{align}
1=w_{1} \geq w_{2} \geq \ldots \geq w_{m}=0 \label{eq:monotone}
\end{align}
this is a natural assumption, as it is better to be ranked first than second, second than third, etc. 
Without loss of generality, we assume $w_1=1$ and $w_m=0$.

The weights of a scoring rule can model different preferences of the committee. 
For instance, the weights can control the inclination to favour ``extreme'' alternatives (often at either the top or the bottom of the input rankings) at the expenses of ``moderate'' alternatives (that are more consistently in the middle part of the input rankings).

An important class of scoring rule is the one composed of weights that are a convex sequence, meaning that the difference between the weight of the first position and the weight of the second position is at least as much as the difference between the weights of the second and third positions, etc.
\begin{align} \forall i \in \{1,\ldots,n-2\} \;\; & w_i - w_{i+1} \geq w_{i+1}-w_{i+2}  \\
\iff & w_i - 2 w_{i+1} + w_{i+2} \geq 0 \label{eq:convexity}
\end{align}
The constraint above is often used when aggregating rankings in sport competitions.
We use $\mathcal{W}$ to denote the set of convex weight vectors.

We argue that in general it can be difficult to set the weights in an appropriate way.
We assume that in addition of basic requirments (monotonicity and convexity)  the chair (the person or the organization that is supervising the voting process) may be able to specify additional preferences about how the social choice function should behave.
In this work we assume that preferences of the chair are encoded with linear constraints on the vector $\w$, relating the value of the weights of different positions.
We use $\C(\w)$ to denote, in an abstract way, the preferences of the chair about the scoring vector.


Of course it may be difficult for real decision makers to state preferences about the voting rule in such an abstract way, unless perhaps in some limited cases (a specific instance of these preferences may consist in giving upper or lower bound on the value of $w_{i}$ for position $i$).
These additional preferences can be elicited by asking questions about concrete profiles, for instance, by showing a complete profile of a small synthetic election and asking who should be elected in this case.
We discuss queries asked to the chairs and strategies to select queries later in Section \ref{sec:elicit}.

\section[Minimax regret under partial profile and weight information]{
Robust winner determination}
\label{sec:mmr}

%The quality of an alternative can be quantified by considering the maximum regret with respect to an adversary that can choose the instantiation of both a complete profile (extending the known preferences of the agents) and of the scoring vectors (associated to the preferences of the committee).
In this paper, we consider a setting where both the voters' preferences and the preferences of the chair about the voting rule are incomplete.
Notice that some authors have considered possible and necessary winners assuming a partial profile  \cite{Xia2008} or assuming an incompletely specified scoring rule \cite{Viappiani2018};
however, we note that, in typical settings, there will be no necessary winner and too many possible winners.
In practice, however, it is often imperative to declare a winner (consensus choice) given partial information.

As a decision criterion to determine a winner, we propose to use minimax regret. 
Minimax regret \cite{Savage1954} is a decision criterion that has been used for robust optimization under data uncertainty \cite{Kouvelis1997} and as well in decision-making with uncertain utility values  \cite{Salo2001,Boutilier2006}.
%Minimax regret \cite{Boutilier2006} has been used for making decisions under utility uncertainty and for preference elicitation.
Lu and Boutilier \cite{Lu2011} have adopted minimax regret for winner determination in social choice with
the preferences of the voters that are only partially known, while the social choice function is known and fixed in advance.

In this work we consider the simultaneous presence of uncertainty in the agents' preferences and uncertainty in the weights.
Using {\em maximum regret} to quantify the worst-case error, the alternative that minimize this error is selected as winner, providing us with a form of robust optimization.
Intuitively, the quality of a proposed alternative $a$ is how far from optimal $a$  could be in the worst case, given the current knowledge about the voting rule and about the voters' preferences.

The maximum regret is considered by assuming an adversary can choose both 1) to extend the partial profile into a complete profile 2) can instantiate the weights choosing among any feasible weight vector in $W \subseteq \mathcal{W}$.

We formalize the notion of minimax regret in multiple steps.
First of all, $\Regret(x, \profile, \w)$ is the loss or ``actual'' regret  of selecting $x$ as a winner instead of choosing the optimal alternative under $\profile$ and $\w$:
\[\Regret(x, \profile, \w) = \max_{y \in A} s(y; \profile,\w) - s(x; \profile, \w).\]

The pairwise max regret of $x$ relative to $y$ given partial profile $\pprofile$ and the space of weights $W$
$\PMR(x,y;\pprofile,W)$ is the worst-case loss under all possible realizations of the full profile {\em and} all possible instantiations of the weights:
\begin{align}
\PMR(x,y; \pprofile, W) & = \max_{\w \in W} \max_{\profile \in C(p)} s(y; \profile,\w) - s(x; \profile,\w).
\end{align}

Max regret $\MR(x;\pprofile,W)$ is the worst-case loss of $x$. It is the loss occurred by an adversarial selection of a complete profile $\profile$ extending $\pprofile$ and a selection of $\w \in W$ to maximize the loss between $x$ and the true winner under $\profile$ and $w$.
\begin{align}
\MR(x; \pprofile, W) & = \max_{y \in A} \PMR(x,y; \pprofile, W)\\
& = \max_{\w \in W} \max_{\profile \in C(p)} \Regret(x, \profile, \w).
\end{align}

Finally,  $\MMR(\pprofile,W)$ is the value of minimax regret under $\pprofile$ and $W$, obtained when recommending a minimax optimal alternative $x^*$:
\begin{align*}
\MMR(\pprofile,W) & = \min_{x \in A} \MR(x;\pprofile,W) \\
x^{*}_{\pprofile,W} & \in \arg\min_{x \in A} \MR(x;\pprofile,W) 
\end{align*}
By picking as consensus choice\footnote{In cases of ties in minimax regret, we can either decide to return all minimax alternatives as ``winner'' or to pick just one of them using some tie-breaking strategy.} an alternative associated with minimax regret, we can provide a recommendation that gives worst-case guarantees, giving some robustness in face of uncertainty (due to both not knowing the agents' preferences and the weights used in the aggregation). 

Notice that if $MMR(\pprofile, W)=0$, then $x^{*}_{\pprofile,W}$  is a necessary co-winner; this means that for any valid completition of the profile and any feasible weight $w \in W$, $x^{*}_{\pprofile,W}$ obtains the highest score (possibly with ties).

% Add some general remarks about using minimax regret
\subsection{Computation of minimax regret}
In order to compute pairwise maximum regret and therefore minimax regret, we adapt the reasoning from \cite{Lu2011} so that we decompose the $\PMR$ into the contributions that is associated to each agent.
The settings is however more challenging due to the presence of uncertainty in the weights.


Scoring rules are additively decomposable.
%A scoring rule associates each alternative to a score that is given by the sum of points obtained for each voter.
Let $s(x; v_j,\w)=w_{v_j(x)}$ be the number of points that $x$ obtains in the ranking $v_j$ (see Equation \ref{eq:srule}), that is given by weight of position $v_j(x)$.
Obviously we have 
\[ s(x; \profile, \w) = \sum_{j=1}^n s(x; v_j,\w). \]

By exploiting the decomposition of the score in terms of votes, we write the actual regret of choosing $x$ instead of $y$ as:
\[
s(y; \profile,\w) - s(x; \profile, \w) = [\sum_{j=1}^n s(y; v_j,\w) - s(x; v_j,\w)]
\]
and  can rewrite $\PMR$ as follows:
\begin{align*}
& \PMR(x,y; \pprofile, W) = \max_{\w \in W} \max_{\profile \in C(\pprofile)} [ s(y; \profile,\w) - s(x; \profile,\w) ] = \\
& =  \max_{\w \in W} \sum_{j=1}^{n} \max_{v_j \in C(\succ_j^p)} [s(y; v_j,\w) - s(x; v_j,\w)]=\\
& =  \max_{\w \in W} \sum_{j=1}^{n} \max_{v_j \in C(\succ_j^p)} [w_{v_j(y)} - w_{v_j(x)}] \\
\end{align*}
Note that in general the inner max will depend on the weights chosen by the outer min.

We consider now a procedure for completing a partial profile that was first proposed by Lu and Boutilier \cite{Lu2011} when considering minimax regret with a fixed rule.
We will show how this procedure can be used the in the case of uncertain weight vector:
in the case of convex scoring rules we will be able to greatly simplify the problem by decoupling the two maximizations, while in the general case we will propose to solve it using a mixed integer program.

Let $U^+=\{ i \mid \ y \ppref_i x\}$ be the set of agents for which we know that $y$ is preferred to $x$ (positive contribution to pairwise regret), $U^-=\{ i \mid x \ppref_i y\}$ 
be the set of agents for wich we  know that $x$ is preferred to $y$ (negative contribution) and 
$U^?=\{ i \mid x \pinc_i y\}$ 
%$U^?=\{ i \mid x \nsucc_{i} y \wedge y \nsucc_{i} x \}$ 
the remaining case, where the preference between $x$ and $y$ is not known.
Consider how the adversary, whose goal is to maximize $\PMR$ between $x$ and $y$, will complete the partial orders:
\begin{itemize}
 \item $j \in U^+$: If we know that agent $j$ prefers $y$ to $x$ then the contribution to $\PMR$ is positive and adversary will complete the partial order of agent $j$ by placing as many alternatives as possible between $y$ and $x$ (because of monotonicity of the weights).
 The way the partial order is completed does not depend on the actual values of the weights.

 \item $j \in U^?$: If we do not know whether $x$ or $y$ is preferred by agent $j$ then the adversary will place $y$ before $x$ (since the goal is to maximize the difference in score between $y$ and $x$, and weights are monotone) in the linear order and this case reduce to the first one; therefore also in this case the completion of the partial order can be done independently of the choice of $w$.

 \item $j \in U^-$: If we know that agent $j$ prefers $x$ to $y$ then (as the contribution of this agent to $\PMR$ is negative) the adversary will place as few alternatives as possible between $x$ and $y$.

The only issue is with respect to the  alternatives that are incomparable with $x$ and $y$ according to $\ppref_i$: should they be placed better than $x$ or worse than $y$ ?
If the scoring vector is known to be convex,  than these alternatives will be placed better than $x$ in the completion; otherwise the placements of these depend on the weights $\w$.
\end{itemize}

\subsubsection{Minimax regret computation under the convex assumption}

Since weights are assumed to be convex, the adversary will place all alternatives $c$ that are such that $c \pinc_{j} y \wedge c \pinc_{j} x$ as better than $x$.
We now formalize the reasoning about the best (with respect to maximizing $\PMR$) completion with the following definition.
\begin{definition}
Given the partial profile $\pprofile$, define $\hat{\profile}$ as the completition of $\pprofile$ such that:
\begin{itemize}
\item For each $j \in U^{+}$, set   $\hat{\profile}$ as follows, for all $c \in A$:\\
if $c \pinc_{j} x$ we set $c \succ_{j}^{\hat{\profile}} x$;\\
if  $c \pinc_{j} y$ we set $y \succ_{j}^{\hat{\profile}} c$.
\item For each $j \in U^{?}$, set  $\hat{\profile}$ as follows:\\
$y \succ_{j}^{\hat{\profile}} x$;\\
for all $c \in A$, if   $c \pinc_{j} x$ we set $c \succ_{j}^{\hat{\profile}} x$;\\
for all $c \in A$, if   $c \pinc_{j} y$ we set $y \succ_{j}^{\hat{\profile}} c$.
\item For each $j \in U^{-}$,  set  $\hat{\profile}$ as follows, for all $c \in A$:\\
if $c \pinc_{j} x \wedge c \succ^{p}_{j} y$ we set $c \succ_{j}^{\hat{\profile}} x$;\\
if $c \pinc_{j} y \wedge  x \succ^{p}_{j} c$ we set $y \succ_{j}^{\hat{\profile}} c$;\\
if $c \pinc_{j} y \wedge c \pinc_{j} x$ we set $c \succ_{j}^{\hat{\profile}} x$.
\end{itemize}
\end{definition}

%Given our hypothesis, for all pairs of alternatives, there exist completions of the partial preferences $\ppref$ that maximize the PMR and that do not depend on the weights. 
%That is, we can exhibit, for each voter $j$, a completion $v_j$ such that $\forall \w, \forall v'_j \in C(\ppref_j): [w_{v_j(y)} - w_{v_j(x)}] \geq [w_{v'_j(y)} - w_{v'_j(x)}]$.

%In summary, some best completion $\pref$ satisfies two properties: (1) for all $c \pinc x$, $c \pref x$; and (2) for all $c \pinc y$, if not $c \pref x \pref y$ then $y \pref c$. 
%These two properties are intuitively understandable by observing that maximizing the PMR requires placing $x$ as low as possible, hence the first property; and $y$ as high as possible, hence the second property, with the exception that sometimes it is not possible to satisfy both goals and noting that the first goal is more important by convexity of the weights. 
%A proof that this reasoning is correct is in \cref{sec:prfCompl}.

The space of feasible weights $W$ is encoded with linear constraints that model the preferences of the committee.
Therefore the pairwise maximum regret can be computed with a linear program.


By solving a sequence of pairwise max regret computations is possible to determine the minimax regret solution.
Each PMR is computed using a linear program.
The decision variables correspond to the weights attached to the different positions.
Constraints are on the feasible parameters (the ``weights'' of the scoring rules); contraints represent the preferences of the chair about the scoring rule.

Let $\hat{\profile}$ be the profile computed according to the above procedure, and  $\hat{v}_j$ the $j$-th element (the linear order extending the partial order $p_j$ of the $j$-th voter).
\begin{prop}
Assume convex weights.
The PMR can be written as:
\begin{align} 
\PMR(x,y; \pprofile, W) =  
\max_{w \in W} s(y; \hat{\profile}, \w) - s(y; \hat{\profile}, \w) =\\
=\max_{w \in W} \sum_{j=1}^n w_{\hat{v}_i(y)} - w_{\hat{v}_i(x)} =
\max_{w \in W} \sum_{i=1}^m (\hat{\alpha}_{i}^{y} - \hat{\alpha}_{i}^{x}) w_i 
\end{align}
where the coefficient $ (\hat{\alpha}_{i}^{y} - \hat{\alpha}_{i}^{x})$ is the number of times $y$ is in position $i$ minus the number of times $x$ is the in position $i$ in the rankings of the completed profile $\hat{v}$:
$\hat{\alpha}_{i}^{x} = \sum_{j=1}^{n}  I[\hat{v}_{j}(x)=i]$ where $I$ is the indicator function.
\end{prop}
%\[ \alpha_{i} = \sum_{j=1}^{n}  I[\hat{v}_{j}(y)=i] - I[\hat{v}_{j}(x)=i]\]
Then, what we have to to compute pairwise max regret $\PMR(x,y; \pprofile,W)$ is just to compute the solution of the following linear program defined on variables\footnote{In fact, given our assumptions, $w_{1}=1$ and $w_{m}=0$, therefore there are only $m-2$ variables. We leave $w_{1}$ and $w_{m}$ in the linear program just for ease of presentation.} $w_{1},\ldots,w_{m}$:
\begin{align*}
\max_{\w} & \sum_{i=1}^m (\hat{\alpha}_{i}^{y} - \hat{\alpha}_{i}^{x}) w_{i}\\
\text{ s.t. } & \text{Equation } (\ref{eq:monotone})\\
& \text{Equation } (\ref{eq:convexity})\\
& \C(\w)
\end{align*}
where $\C(w)$ are the linear constraints on $\w$ encoding the chair's preferences.
The result of the optimization is the value of the pairwise max regret $\PMR$.

\subsubsection{Minimax regret without the convex assumption}
{\em (Still here? Or move it somewhere else?) }

Without the convex assumption, we cannot use $\hat{v}$ for the agents in $U^{-}$, but only for agents in $U^{+}$ and $U^{?}$.
%Let $\hat{v}_i$ be the linear order extending $\succ^{p}_i$ according to the above procedure.
Then PMR can be written as follows:
\begin{align*}
 &\PMR(x,y; \pprofile, W) =\\ 
 &\max_{\w \in W} \Big \{ \sum_{j \in U^-} [\max_{v_j \in C(\succ_j^p)} [w_{v_j(y)} \!-\! w_{v_j(x)}]] 
  \!+\!	 \sum_{j \in A^+ \cup  U^?} [w_{\hat{v}_j(y)} \!-\! w_{\hat{v}_j(x)}] \Big \} 
 \end{align*}
%Note that the second addendum inside the max do not depend on the choice of $\w$.
Consider the two addenda inside the ``max''. 
The first addendum is concerned with positioning of alternatives $x$ and $y$ for voters in $U^{-}$ for which we know that $x$ is preferred to $y$.
The second addendum is concerned with voters in $U^{+}$ and $U^{?}$.
We rewrite the second addendum as:
\begin{align*}
\sum_{j \in U^+ \cup  U^?} [w_{\hat{v}_j(y)} \!-\! w_{\hat{v}_j(x)}] 
= \sum_{i = 1}^{m} (\hat{\alpha}_{i}^{y} - \hat{\alpha}_{i}^{x}) w_{i}
\end{align*}
where $\hat{\alpha}_{i}^{x}$ is the number of times that $x$ is ranked  in position $i$  considering the profile $\hat{v}$ of voters in $U^+ \cup  U^?$.
%We compute pairwise maximum regret by considering binary variables $\{ B_{i} \}_{i=1,\ldots,m}$ to represent optimization choices related to where to position the alternatives.

We now address the agents in $A^{-}$
For a given $j$, let $\beta$ be the number of alternatives that are incomparable with $x$ and $y$:
\[ \beta_{j} = \mid \{ c : c \pinc_{j} y \wedge c \pinc_{j} x \} \mid \]
$x$ can be ranked between $t_{1}(j)=\mid \{ c \in A : c \ppref_{j} y \wedge x \nppref_{j} c \} \mid $ and position $t_{2}(j)=t_{1}(j)+\beta_{j}$.
The completion for agent $j$ is such that the positions of $x$ and $y$ differ of exactly $\gamma_{j} =
\mid \{ c \in A : x \ppref c \ppref y \} \mid$ positions.


We now show how to optimize $\PMR$.
In addition to variables $\{ w_{j} \}_{j=1,\ldots,m}$ (one for each position) we need to employ several additional decision variables.
We introduce two sets of binary variables $B^{+}_{i,j}$ and $B^{-}_{i,j}$  for each position $i$ and for each voter $j$.
Variable $B_{i,j}^{+}$ encodes the fact that the alternative $y$ is placed in position $i$ in the ranking of agent $j$; while  $B_{i,j}^{-}$ encodes the same thing for alternative $x$.
%We also have numerical variables to represent the weights of the scoring rule.
Since each alternative needs to be placed exactly in one place for each agent, we adopt the constraints
$\sum_{i=t_{1}(j)}^{t_{2}(j)} B_{i,j}^{+} = 1$.
Since we know that $x$ and $y$ are ranked $\gamma{j}$ positions apart, we set the constraint:
$B_{i+\gamma{j},j}^{-} \geq B_{i,j}^{+}$,  for $i = \{ t_{1}(j), \ldots, t_{2}(j)\}$.

% and $\sum_{i=1}^{m} B_{i,j}^{-} = 1$.
%Since the objective is to maximize pairwise regret...

The score of alternative $y$ can be written as $\sum_{i = 1}^{m} \hat{\alpha}_{i}^{y}  w_{i} + \sum_{i=1}^{n} \sum_{j=1}^{m} w_{j} B_{i,j}^{+}$.
The objective function is now:
 \[ \max \sum_{i = 1}^{m} (\hat{\alpha}_{i}^{y} - \hat{\alpha}_{i}^{x}) w_{i} +  \sum_{i=1}^{n} \sum_{j=1}^{m} (B_{i,j}^{+} - B_{i,j}^{-})  w_i \]

We use integer programming enconding tricks in order to linearize the problem.
We introduce yet another set of variables  $V_{i,j}^{+} $  and $V_{i,j}^{-}$ % the multiplicative terms by new variables.
and we enforce that $V_{i,j}^{+} = B^{+}_{i,j} w_i$ by setting constraints $V_{i,j}^{+} \leq B^{+}_{i,j}$ and $V_{i,j}^{+}  \leq w_i$.
We have similar constraints for enforcing $V_{i,j}^{-} = B^{-}_{i,j} w_i$.

We therefore obtain the following mixed integer linear program:
\begin{align}
\max & \sum_{i=1}^m  [(\hat{\alpha}_{i}^{y} - \hat{\alpha}_{i}^{x}) w_{i}] +
  \sum_{j \in A^{-}} \sum_{i=t_{1}}^{t_{2}}  [V_{i,j}^{+} - V_{i,j}^{-}]
\end{align}
\begin{align}
\text{ s.t. } &  \text{Equation } (\ref{eq:monotone}) & \\
&  \C(\w) &  \\
& \sum_{i=t_{1}}^{t_{2}} B_{i,j}^{+} = 1 & \forall j \in A^{-} \\
& B_{i+\gamma_{j},j}^{-} \geq B_{i,j}^{+} & \forall i \in \{ t_{1}, \ldots t_{2}\}, \forall j \in A^{-} \\
& V_{i,j}^{+} \leq B_{i,j}^{+}  & \forall i \in \{ t_{1}, \ldots t_{2}\}, j \in A^{-} \\
& V_{i,j}^{+} \leq w_i & \forall i \in \{ t_{1}, \ldots t_{2}\}, j \in A^{-} \\
& V_{i,j}^{-} \geq w_{i} + B_{i,j}^{-} - 1 & \forall i \in \{ t_{1}, \ldots t_{2}\}, j \in A^{-}\\
& V_{i,j}^{-} \geq 0 & \forall i \in \{ t_{1}, \ldots t_{2}\}, j \in A^{-}
\end{align} 
%We write $w \in W$ as a shourtcut to represent the constraints that the weights are chosen to be in the feasible set.
There are (at most) $nm$ binary variables and $m(n+1)$ numerical variables.
The optimization program can be solved by any suitable MILP solver, although it is not suitable to large problem instances.

% DONT KNOW IF WE HAVE TO FORMALIZE THIS AS A CLAIM
%\begin{claim}
%The $\PMR$ is computed using the above optimization problem.
%\end{claim}

\section{Interactive Elicitation} \label{sec:elicit}

We propose an incremental elicitation method based on minimax regret.
At each step, the system may either ask a question to one of the agents, or ask a question to the chair about the voting rule. 
The goal is to acquire relevant information to reduce minimax regret as quick as possible.
%Minimax regret can be used to guide the elicitation process.
As termination condition of elicitation, we can check whether minimax regret is lower than a threshold; if we wish optimality, we can perform elicitation until minimax regret drops to zero.

The remainder of this Section is structured as follows.
First of all, we discuss the different types of queries that can be asked to the voters and to the chair, and the way responses are handled.
Then, we describe different strategies to determine informative queries to ask next, with the goal of reducing $\MMR(\pprofile,W)$ quickly.


\paragraph{Query types}
We distinguish between queries asked to the voters and questions asked to the chair.
As {\em questions asked to the voters}, it is natural to consider comparison queries asking to compare two alternatives.
Another common type or queries are {\em top-k}, asking to each voter his $k$ most alternatives.
The effect of a response to a query asked to the voters, is to augment our knowledge about the voters rankings, thus augmenting the partial profile $\pprofile$. 
If voter $i$ answers a comparison query stating that alternative $a$ is preferred to $b$, then the partial order $\succ^{p}_{i}$ is augmented with $a \succ^{p}_{i} b$ and by transitive closure.

A bit more discussion is need about {\em questions asked to the chair}.
A query asked to the chair has the goal of refining our knowledge about the scoring rule;
a response to a query gives us a constraint on the weighting vectors $w$.

In particular, we want to acquire constraints of the type:
\[ w_{j} - w_{j+1} \geq \lambda (w_{j+1} - w_{j+2}) \] 
for $j \in \{1,\ldots,m-2\}$, relating the difference between the importance of positions $i$ and $i+1$ with the difference between positions $i+1$ and $i+2$.
While perhaps in some cases the decision maker may be able to state preferences of this kind, we argue that a more concrete way is better.
Notice that the constraint above is equivalent to 
$w_{i} - (\lambda+1) w_{i+1} + \lambda w_{i+2} \geq 0$
and to $w_{i} + \lambda w_{i+2} \geq  (\lambda+1) w_{i+1}$.
%We conjecture that preferences of this kind are relatively easy to state
%(``do you think that the difference between the importance of position $i$ and $i+1$ is at least the difference between positions $i+1$ and $i+2$ ?'').
We present them a profile representing the situation described by the query and then deduce its answer from the selected winner. 
%We argue that it may be cognitively easier to ask to pick a winner from a profile, rather than answering an abstract question.
Therefore we generate profiles specifically in such a way that user answers gives us the kind of constraint that we want to acquire.
We provide a systematic way to construct such profiles.  %that reflects the situation outlined by the query for two alternatives and the others are not better than them.

We consider a profile where a fictious alternative $a$ is $1$ time in position $i$, and $\lambda$ times 
in position $i+2$ and another fictitious alternative, $b$, is ranked $\lambda+1$ times in position $i+1$.
To have a valid profile, we need to include other alternatives.
We add other alternatives in such a way that they are dominated by $a$ or $b$. 
(if the chair picks an alternative that is either $a$ or $b$ this is a signale that the model is incorrect).

\vspace{1cm}
{\em {\bf TODO:} complete section and give a precise algorithm for completing the profile}
\vspace{1cm}

\begin{example}
Suppose our query strategy suggests us to ask the committee the following query:
\[ w_{2} - w_{3} \geq 2(w_{3} - w_{4}) \]
In order to impose this constraints,
It is straightforward to see that we want the  chair of the committee to choose between $a$ or $b$ with the following rank distribution:
\[
\begin{array}{ccccc}
& 1^\circ
& 2^\circ
& 3^\circ
& 4^\circ \\
\cmidrule{2-5}
a & 0& 1& 0& 2 \\
b& 0& 0& 3& 0 \\
\end{array}
\]
In order to obtain a valid profile, we need to include a number of other alternatives in such a way that the are dominated by $a$ and $b$.
The following profile is exactly what we want,
%We add voters in order to increase the number of times $a$ and $b$ are ranked at first position. Please note we must maintain these scheme, so every add of a position to the ranking of $a$ must correspond to the same in the ranking of $b$.

\[
\begin{array}{cccccc}
v_1& v_2& v_3 & v_4& v_5& v_6 \\
\midrule 
a& b& c & d& a& b \\
c& a& d& c& b& a \\
b& d& b& b& d& c \\
d& c& a& a& c& d \\
\end{array}
\]
giving the following rank distribution (number of times each alternatives is ranked first, second, etc): 
\[
\begin{array}{ccccc}
& 1^\circ& 2^\circ& 3^\circ& 4^\circ \\
\cmidrule{2-5}
a & 2& 2& 0& 2 \\
b& 2& 1& 3& 0 \\
c& 1& 2& 1& 2 \\
d& 1& 1& 2& 2 \\
\end{array}
\]
Indeed alternative $c$ and $d$ are dominated and they cannot win for any instantiation of the weights; this can be seen by comparing the cumulative ranks (see  \cite{Stein1994}).
\[
\begin{array}{ccccc}
& \geq 1^\circ
& \geq 2^\circ
& \geq 3^\circ
& \geq 4^\circ \\
\cmidrule{2-5}
a & 2 & 4& 4& 6 \\
b & 2 & 3& 6& 6 \\
c& 1& 3& 4& 6 \\
d& 1& 2& 4& 6 \\
\end{array}
\]
\end{example}



\paragraph{Elicitation strategies}

We develop some elicitation strategies for simultaneous elicitation of voters' preferences and of the scoring rule.
Starting from some initial partial knowledge, our goal is to learn both the scoring rule function and the agents' preferences.
While it is of course possible doing full elicitation of the weights and afterwards elicit the agents' preferences (or the other way around) we propose an interleaved approach.
In our interactive  protocol for simultaneously eliciting the preferences of the chair about the voting rule and the voters' preferences about the alternatives.
Indeed, it can be beneficial to interleave questions asked to the committee and questions asked to voters, depending on which is estimated to be more informative.

Answers given by the committee about the scoring rule refine our knowledge of the weights $w_1,\ldots,w_n$, while
answers given by one of the voters refine our knowledge about the voters preferences.

At each step we need to decide whether we want to ask a question to the committee or to one of the agents (and to which agent in particular). We assume that we can ask comparison queries to the agents and questions comparing the differences of weights to the committee. 

In the experiments we want to compare the interleaved approach with a baseline challenger, a method  that elicits the preferences of the voters first and then the voting rule (or the other way around)

%Type of questions that we can ask to the chair:	bound queries (Is this alternative among your top-k most preferred items?), comparison queries (do you prefer alternative x or alternative y?).

% we discuss the different forms of queries and develop several query strategies 
%\paragraph{Query methods}
We consider different strategies to determine the next question to ask given the current information.
%{\em Some ideas: decompose the regret into two components, one due to $\w$ and one due to $\pprofile$, and ask a question to the chair / or to one of the agents depending on which is highest}

%We consider strategies of the following form
%{\em if} condition {\em then} ask comparison query {\em else} ask a committee query
%Define a score associated to each potential query that we may ask.

\begin{itemize}
\item {\em Current solution strategy}: consider the solution of the minimax regret game, 
 $(x,y,\profile^{a},\w^{a})$, where $x$ is the minimax regret optimal alternative, $y$ the adversarial choice, $\w^{a}$ the weights, and $\profile^{a}$ the profile completed by the adversary.

Ask a question to compare $x$ with $y$ to an agent.

% Determine the part of ``regret'' due to uncertainty in $w$
% Determine the part of ``regret' due to uncertainty in $p$
 
 \item One possibility is to consider the MMR {\em a posteriori}. Assume that the different answers to a query induce the possible sets to be $(\pprofile_1,W_1)$ and $(\pprofile_2,W_2)$, then the score according to worst-case maximum regret is:
\[\SCORE(x)= \max_{i=1,2} \MMR(\pprofile_i,W_i) \]
In this case, the query with least value is chosen.

What is the complexity of this strategy? Let's define $Q$ as the set of all possible queries we can address to the agents and to the committee. The worst case is represented by the situation in which we do not have any information about both the agents' preferences and the weights constraints. In this case we can address $\binom{m}{2}=\frac{m(m-1)}{2}$ questions to each agent. Moreover, we need to ask at least $m$ questions to the committee. Then, each question induces two possible pairs of $(\pprofile,W)$ for each of which we compute the $\MMR$, whose worst case complexity is $O(nm^3)$. Thus, the cardinality of $Q$ in the worst case is $\approx n \cdot \frac{m(m-1)}{2} + \lambda m$, and the worst case time complexity of the strategy is $O(n^2m^5)$.

\item Two phase method.
Ask a predefined (non adaptive) sequence of questions in order to learn the weights $w$ of the scoring rule).
Then use minimax regret as in Lu and Boutilier.

%\item Non-interleaved regret method: comparison with a method based on minimax regret, but not interleaved. 

\item Non-interleaved non-regret strategies (useful as benchmarks).	
	\begin{itemize}
		\item Random strategy: randomly chooses an agent $i$ and a comparison query such that $x \pinc_i y$.
		\item {\em Volumetric} strategy: chooses an agent $i$ and a query that maximizes the number of new pairwise preferences revealed given the worst response.
	\end{itemize}
\end{itemize}

% unless $\MMR=0$ we ask a question that reduce uncertainty

\section{Empirical Evaluation} \label{sec:experiments}

{\em {\bf TODO:} update with actual experiments}

\begin{figure}[t]
\begin{center}PLACEHOLDER\end{center}
\vspace{3.5cm}
\caption{Minimax regret reduction in our experiments (synthetic dataset).}
\end{figure}

\begin{figure}[t]
\begin{center}PLACEHOLDER\end{center}
\vspace{3.5cm}
\caption{Minimax regret reduction in our experiments (real dataset).}
\end{figure}

% We can stop elicitation when minimax regret is sufficiently small.
We test the elicitation protocols in randomly generated datasets and real datasets. % (sushi and iris).
We evaluate the proposed method using the following simulated protocol.
Our goal is to 
 to verify our intuition that an interleaved elicitation approach (mixing questions to the chair and questions asked to the voters) is more efficient than performing the elicitation of the rule first and than eliciting the voters preferences.
First of all, we randomly generate the true preferences of the users (i.e. the linear orders) and the weights associated with the committee's scoring rule.
For each of the elicitation strategy, we simulate an elicitation session.
%Queries are posed to simulated users, that answer accor
%We compare the performance  of the different strategies of Section \ref{sec:elicit} with respect to decrement in max regret and with respect to the real loss.
We measure the effectiveness of our strategies of Section \ref{sec:elicit} by examining regret reduction as a function of the number of queries.

%For each of the elicitation strategy, we simulate the elicitation by asking the queries selected by the strategy.
We performed tests with different values of $m$ (number of alternatives) and $n$ (number of agents); test with different population sizes, different number of alternatives, etc.

We experimented with both settings that assume a convex sequences of weights, and others without this assumption.

{\em {\bf TODO:} say something about computation times}

%Comparison between our interleaved strategy and a strategy, still based on minimax regret  but that is not interleaved; also compare to some heuristic baselines.

\section{Conclusions}  \label{sec:conclusions}

% integrated preference elicitation methodology
In this paper we have considered a  social choice setting with partial  information about the voter's  
preferences and as well a partially specified voting rule.
We have proposed the use of minimax regret as a means of robust winner determination in this setting, and as well as to guide the process of simultaneous elicitation of preferences and voting rule.
Our experimental results on randomly generated and real world data sets show that regret-based elicitation is effective and allow to find a near-optimal consensus choice in a limited number of steps.
%The detection of optimal or near-optimal product recommendation is generally possible with little concept and utility information
%Regret-based elicitation allows to determine near-optimal winners using only few information about the voters' preferences.
We mention some  important directions for future works.
First of all, further development of elicitation strategies, considering alternative heuristics, is an important direction. 
Second, we are interested in extending elicitation of voting rules going beyond scoring rules.
A third direction is that of considering probabilistic methods for elicitation.
Finally, we are interested in studying the effect of strategic agents, that may not report their true preferences.
% distribution-free heuristics

% Acknowledgements: We thank the reviewers for comments helping to improve the paper. 
%{\small
\bibliography{biblio}
\bibliographystyle{named}
%\bibliographystyle{plain} 
%}

\pagebreak
\ifappendix
\appendix
\section{Minimax Computation under Convex Assumption} 

Refer to \cite{Lu2011}
The goal is to choose as a winner the alternative $x^*$ whose worst case loss is minimal under all possible realizations of the full profile and all possible choices of weights. 
Assume hereinafter the selected weights sequence $\w \in W$ to be convex. 
In order to compute the minimal max regret $\MMR(\pprofile)$ under partial profile $\pprofile$ we need to compute the pairwise max regret between all pairs of alternatives $(x,y)$, where $x$ is a proposed winner and $y$ is the ``adversary'' alternative. Indeed, the construction of $\PMR(x,y,\pprofile,\w)$ can be viewed as an adversary's attempt to maximize the regret of choosing $x$ instead of $y$. 
For doing this, he can choose a completion $\profile_i \in C(\pprofile_i)$ of the partial profile and a (feasible) scoring vector $\w$ that maximize the contribution of the voter $i$ to $\PMR(x,y,\pprofile,\w)$. Let us now analyze how it could be done depending on the relation between alternatives $x$ and $y$ in $\pprofile_i$. 
\begin{itemize}
	\item $x \succ_i^\pprofile y$
	\newline If we know $x$ is preferred to $y$ and we choose $x$ as a winner, $\pprofile_i$ contribution to $\PMR(x,y,\pprofile,\w)$ must be negative. In this situation, our adversary can only try to minimize this advantage by minimizing the positional gap between the two alternatives. To achieve that, he can arbitrary place all the alternatives preferred to $x$ above $x$, together with all the ones with unknown relation to $x$. Moreover, he can place all the alternatives less preferred to $x$ and with unknown relation to $y$ below $y$. We can summarize it for each $q \in A$ as follows:
	\begin{align*}
	q \succ_i^\pprofile x \vee q \ ?_i^\pprofile \ x \ & \Rightarrow \ \uparrow_x \\
	x \succ_i^\pprofile q \wedge ( q \ ?_i^\pprofile \ y \vee y \succ_i^\pprofile q) \ & \Rightarrow \ \downarrow_y \\
	x \succ_i^\pprofile q \succ_i^\pprofile y \ & \Rightarrow \ \text{in between} \\
	\end{align*}
	It is worth noting that when the relation between $q$ and $x$ is not known in the partial profile, the adversary takes advantage by placing $q$ above $x$ only under the assumption of convex weight sequences.
	\item $y \succ_i^\pprofile x$
	\newline If $y$ is preferred to $x$ the construction proceeds similarly to the previous case, but now the adversary takes advantage by maximizing the gap between $x$ and $y$ placing as much alternatives as he can between the two. We can summarize the procedure for each $q \in A$ as follows:
	\begin{align*}
	q \succ_i^\pprofile y \ & \Rightarrow \ \uparrow_y \\
	x \succ_i^\pprofile q \ & \Rightarrow \ \downarrow_x \\
	(y \succ_i^\pprofile q \vee y \ ?_i^\pprofile \ q) \wedge (q \succ_i^\pprofile x \vee q \ ?_i^\pprofile \ x) \ & \Rightarrow \ \text{in between} \\
	\end{align*}
	\item $x \ ?_i^\pprofile \ y$
	\newline If the partial profile $\pprofile_i$ does not specify the relation between $x$ and $y$, the advantage is maximized by ordering $y$ over $x$ and maximizing the gap between them following the procedure for the case $y \succ_i^\pprofile x$.
\end{itemize}

\section{Dropping the Convex Assumption}
\subsection{Profile completion}
What if the sequence of weights is not convex? When $y \succ_i^\pprofile x$ or $x \ ?_i^\pprofile \ y$ weights do not influence the arbitrary placement of alternatives. Please remind we are working under the assumption that weights constitute a monotonic non-increasing sequence. Thus, there is no way for the adversary to take advantage from the weights distribution in order to increase the gap between $y$ and $x$ besides placing as much alternatives as he can between the two. The only case in which weights can influence the positional gap between $x$ and $y$ is when $x \succ_i^\pprofile y$ and $q \ ?_i^\pprofile \ x$. For convex sequences we place such alternatives $q$ above $x$, but it is not obvious that this is the best option for other sequences. For example, suppose $x$ and $y$ are ranked respectively in first and second position in the partial profile and we wonder where to place an alternative $q$ with unknown relation to $x$ (and thus to $y$). Suppose also that in the weight sequence the distance between the first and second positions is much lower than the one between the second and the third ones. In this case, placing $q$ above $x$ does not minimize the gap between $x$ and $y$ but we want, instead, to place $q$ below $y$.
\newline The constraints expressed by the chair may result in a set of feasible vectors such that none of them forms a convex sequence. In this case we need to analyze the particular sequence of weights in order to decide how to maximize the adversary advantage. Before going into details, let us define $A$ as the set of alternatives (if any) preferred to $x$, $B$ as the set of alternatives preferred to $y$ but not to $x$, and $U$ the set of those with unknown relation to both $x$ and $y$. The idea is to determine the positions that minimize $x$'s advantage over $y$ and then place some of the alternatives in $U$ above $x$ and some below $y$ in order to get that desired ranking. Since we cannot change the order of the alternatives in the set $B$ we know that $x$ and $y$ are separated exactly by $|B|$ positions (the adversary would not take any advantage by adding alternatives between them). So, starting from the position of $x$ in the partial completion of $\pprofile_i$ computed so far ($\hat{\profile}_i$), we find the two positions separated by $|B|$ alternatives whose weights difference is the lowest. Note that we can only add $|U|$ alternatives so we can check only until the position $\hat{\profile}_i(x)+|U|$. Algorithm \ref{alg:splittingU} shows the procedure described.

It is easy to see that we check at most $|U|$ positions. In the worst case the size of $U$ is equal to $m-2$, thus the procedure can be computed in $O(m)$ time. This cost does not affect the minimax regret computation time complexity that remains $O(nm^3)$.

\begin{algorithm}[h] 
	\caption{Placing alternatives in $U$ without Convex Assumption}
	\label{alg:splittingU} 
	\begin{algorithmic}
		\Require $x$, $y$, $\hat{\profile}_i$, $\w$, $U$, $B$
		\Ensure $\profile_i \in C(\pprofile)$
		\Statex
		\State $ j \gets 0$;
		\State $ i \gets \hat{\profile}_i(x)$;
		\State $ \mathit{posmin} \gets i$;
		\State $ \mathit{min} \gets \w(i) - \w(i+1+|B|)$;
		\While {$( j \leq |U| )$}
		\If{ $(\w(i+j)-\w(i+1+|B|+j) < \mathit{min})$ }
		\State $ \mathit{min} \gets \w(i+j) - \w(i+1+|B|+j)$;
		\State $ \mathit{posmin} \gets i+j$;
		\EndIf
		\EndWhile
		
		\State $U_{\mathit{abovex}} \gets (i-\mathit{posmin}) \text{ alternatives} \in U $;
		\State $U_{\mathit{belowy}} \gets U \setminus U_{\mathit{above}}$;
		\Statex
		\State $\profile_i \gets place(\pprofile_i,U_{\mathit{abovex}},U_{\mathit{belowy}})$;
		\Statex \Return $\profile_i$
		
	\end{algorithmic}
\end{algorithm}

%\subsection{Profile completion, again}
% %{\bf \em TODO: still have to find out exactly how to do it}
% 
% We introduce a set integer variables $I^i$, one for each position $i$.
% Let $U^j$ be, for the agents in $U^-$, the alternatives that, in the partial order $\succ^p_j$ of some agent $j \in U^-$, are incomparable with $x$ and $y$.
% \[ \sum_{l=1}^n I^j_l w_l \]
% Variable $I^j$ associated with the constraint $0 \leq I \leq |U^j|$.
% 
% We provide the following MIP (mixed integer linear program).
% 
%% Use linearization techniques to handle the multiplications...
%
%
%%\section{Even more fun}
%%We can consider different types of questions: asking to compare a pair of alternatives, asking about top-k alternatives.
%

\section{Proof of completion}
\label{sec:prfCompl}
Given $w$ and ${\pref} \in C(\ppref)$, let $w[\pref]$ denote $w_{|\pref(y)|} - w_{|\pref(x)|}$. \commentOC{Changing the notation to slightly reduce the number of symbols, to be discussed.}

Given ${\pref} \in C(\ppref)$, define $\pref^1$ as $\pref$ except that the elements in $C = \{c \pinc x \land x \pref c\}$ come just above $x$ in $\pref^1$, respecting in $\pref^1$ their internal ordering in $\pref$. We show first that $\pref^1$ is an element of $C(\ppref)$; and second that $\forall w: w[\pref^1] \geq w[\pref]$. 

First, it is an element of $C(\ppref)$ as it does not contradict $\ppref$. To show this, recall first that $\pref$ does not contradict $\ppref$. Left to show is that $\forall c \in C, d \notin C: x \pref d \pref c \Rightarrow d \pinc c$ (thus any element $d$ whose relation with $c$ has changed from $\pref$ to $\pref^1$ were incomparable with $c$ in $\ppref$). Consider any such pair $c, d$. As $d \notin C$, $\lnot(d \pinc x)$ or $d \pref x$. From the hypothesis, $d \pref x$ is excluded. 
As $\pref$ extends $\ppref$, $\lnot(d \ppref x)$ and $\lnot (c \ppref d)$. Because $\lnot (d \pinc x)$, $x \ppref d$. Thus, $\lnot (d \ppref c)$, otherwise by transitivity $x \ppref c$. The conclusion now follows from $\lnot (c \ppref d)$ and $\lnot (d \ppref c)$.

Second, in $\pref^1$, compared to $\pref$, $x$ is $|C|$ positions lower, and $y$ is at most $|C|$ positions lower, depending on where $y$ was in $\pref$. Hence, by convexity of the weights, $w[\pref^1] \geq w[\pref]$.

Now define $\pref^2$ as $\pref^1$ except that the elements from $C' = \{c \pinc y \land \lnot (c \pref^1 x \pref^1 y) \land c \pref^1 y\}$ move just below $y$, respecting in $\pref^2$ their internal ordering in $\pref^1$. Observe that this move keeps property (1), that was satisfied by construction in $\pref^1$, intact, and satisfies property (2). The rest of the reasoning is similar to the previous move, as the situations are symmetric. (To be checked?) Hence, ${\pref^2} \in C(\pref)$ and it satisfies both properties.

Given any $x, y, w$, as $C(\pref)$ has an element maximizing $w[.]$ (by finiteness of the set), the above reasoning shows that some element in $C(\pref)$ satisfies both properties and maximizes $w[.]$. And any element in $C(\pref)$ satisfying both properties attribute the same positions to $x$ and $y$, thus, maximizes $w[.]$.

\section{Considerations on weights}
\label{sec:weights}
Let us consider a monotonic non-increasing sequence of weights: $w_{1} \geq w_{2} \geq \ldots \geq w_{m}$. Without loss of generality, we can assume $w_1=1$ and $w_m=0$.

\begin{claim}
	\label{clm:wsequence}
	If the weight sequence is convex then $w_{1} > w_{2}$.
	\[\forall i \in \{1,\ldots,m\} \;\; w_i - w_{i+1} \geq w_{i+1}-w_{i+2} \Rightarrow w_{1} > w_{2} \geq \ldots \geq w_{m}\] 
\end{claim}
\begin{proof}
	By contradiction let assume $w_{1} = w_{2}$ then 
	\begin{align*}
	w_{1} - w_{2} \geq w_{2} - w_{3} &\geq \dots \geq w_{m-1} - w_{m} \\
	1 - 1 \geq 1 - w_{3} &\geq \dots \geq w_{m-1} - 0 \\
	0 \geq 1 - w_{3} &\geq \dots \geq w_{m-1}
	\end{align*}
	At this point either $0\leq w_{3}<1$ or $w_{3}=1$. In the first case 
	\[0 \ngeq 1 - w_{3}\]
	This breaks the convexity assumption so it is impossible.
	In the second case, by definition there is a $w_{i} \neq 1$ where $2 < i \leq m$. So it would be 
	\[0 \ngeq 1 - w_{i}\]
	for some $i$. Again, the convexity is not satisfied.
\end{proof}

\begin{corollary}
	\label{cor:weq}
	If the weight sequence is convex and $w_{i} = w_{i+1}$ for some $i$, then $w_{j}=0 \ \forall \
	j=i, \dots m$.
\end{corollary}


\section{Querying the committee}
Suppose our query strategy suggests us to ask the committee the following query:
\[ w_{2} - w_{3} \geq 2(w_{3} - w_{4}) \]
Then we can transform it in:
\begin{align}
\label{eqn:juryquery}
w_{2} - w_{3} &\geq 2 \cdot w_{3} - 2 \cdot w_{4} \notag \\
w_{2} + 2 \cdot w_{4} &\geq 3 \cdot w_{3} 
\end{align}
and ask the committee if they would prefer as a winner an alternative ranked first one time and third three times rather than an alternative ranked second four times.

Another way to query the committee, a more concrete one, is to present them a profile representing the situation described by the query and then deduce its answer from the selected winner. Obviously we have to be sure to choose a profile where only the alternatives we are interested in could be pick as winners. Therefore, we need a systematic way to construct a profile that reflects the situation outlined by the query for two alternatives and the others are not better than them.


Assume we have a profile of $3$ voters ranking $4$ alternatives. We can represent it by columns where each of them represents the preference ordering of one voter.
\[
\begin{array}{ccc}
v_1
& v_2
& v_3 \\
\midrule 
c
& d
& c \\
a
& c
& d \\
b
& b
& b \\
d
& a
& a \\
\end{array}
\]

Assuming anonymity, we can also write the profile expressing for each alternative $i \in A$ the number of voters placing it at a given rank.

\[
\begin{array}{ccccc}
& 1^\circ
& 2^\circ
& 3^\circ
& 4^\circ \\
\cmidrule{2-5}
a 
& 0
& 1
& 0
& 2 \\
b
& 0
& 0
& 3
& 0 \\
c
& 2
& 1
& 0
& 0 \\
d
& 0
& 1
& 1
& 1 \\
\end{array}
\]

Recalling the query (\ref{eqn:juryquery}) we are considering as an example, it is easy to see that in the current profile alternatives $a$ and $b$ represent the situation as described by the query. Therefore, if after proposing this profile to the committee the winning alternative turns out to be $a$ then we know that $w_{2} - w_{3} > 2(w_{3} - w_{4})$; if, instead, the winner is $b$ we know that $w_{2} - w_{3} < 2(w_{3} - w_{4})$ and if both alternatives are picked as winners then $w_{2} - w_{3} = 2(w_{3} - w_{4})$. 

Nevertheless, in this example it is clear that $c$ will always be preferred to other alternatives (see Claim \ref{clm:wsequence} in Section \ref{sec:weights}). To see it we can express for each alternative the number of voters placing it at a given rank or at a higher one.

\[
\begin{array}{ccccc}
& \geq 1^\circ
& \geq 2^\circ
& \geq 3^\circ
& \geq 4^\circ \\
\cmidrule{2-5}
a 
& 0
& 1
& 1
& 3 \\
b
& 0
& 0
& 3
& 3 \\
c
& 2
& 3
& 3
& 3 \\
d
& 0
& 1
& 2
& 3 \\
\end{array}
\]


\begin{claim}
	Consider a set $A$ of $m$ alternatives and let $r_k(i)$ be the number of times the alternative $i$ obtains a rank $k$ or a higher one. Consider two alternatives $ i,j \in A$, if $\forall \ k=2, \dots,m \ r_k(i)\geq r_k(j)$ and $r_1(i) > r_1(j)$ then $i$ is always preferred to $j$ for any choice of weights.
\end{claim}

\begin{proof}
	For our assumptions we know the sequence of weights is monotonic non-increasing and convex. Moreover, for the Claim \ref{clm:wsequence} in Section \ref{sec:weights}, we know that $w_1 > w_2$. Therefore, even in the worst case where $r_k(i) = r_k(j) \ forall \ k=2, \dots,m$ the sum of weights for alternative $i$ cannot be lower than the one for $j$. It is worth noting that we cannot say anything when $r_1(i) = r_1(j)$. Indeed, as said in Corollary \ref{cor:weq}, all the weights for other position but the first one can be equal, thus the number of voters ranking the alternatives at a certain position does not matter anymore.	
\end{proof}

So the strategy for querying the committee is to use two alternatives $i$ and $j$ to represent the query and complete the profile such that the other alternatives are all dominated by them. An algorithmic approach is to start from the initial profile and then add as many voters as needed that rank $i$ and $j$ as their top choices and the other alternatives afterwards. As an example let consider again the query \ref*{eqn:juryquery}. We want the committee to choose between $a$ or $b$:

\[
\begin{array}{ccccc}
& 1^\circ
& 2^\circ
& 3^\circ
& 4^\circ \\
\cmidrule{2-5}
a 
& 0
& 1
& 0
& 2 \\
b
& 0
& 0
& 3
& 0 \\
\end{array}
\]

We add voters in order to increase the number of times $a$ and $b$ are ranked at first position. Please note we must maintain these scheme, so every add of a position to the ranking of $a$ must correspond to the same in the ranking of $b$.

\[
\begin{array}{cccccc}
v_1
& v_2
& v_3 
& v_4
& v_5
& v_6 \\
\midrule 
a
& b
& c 
& d
& a
& b \\
c
& a
& d
& c
& b
& a \\
b
& d
& b
& b
& d
& c \\
d
& c
& a
& a
& c
& d \\
\end{array}
\]

\[
\begin{array}{ccccc}
& 1^\circ
& 2^\circ
& 3^\circ
& 4^\circ \\
\cmidrule{2-5}
a 
& 2
& 2
& 0
& 2 \\
b
& 2
& 1
& 3
& 0 \\
c
& 1
& 2
& 1
& 2 \\
d
& 1
& 1
& 2
& 2 \\
\end{array}
\]

\[
\begin{array}{ccccc}
& \geq 1^\circ
& \geq 2^\circ
& \geq 3^\circ
& \geq 4^\circ \\
\cmidrule{2-5}
a 
& 2
& 4
& 4
& 6 \\
b
& 2
& 3
& 6
& 6 \\
c
& 1
& 3
& 4
& 6 \\
d
& 1
& 2
& 4
& 6 \\
\end{array}
\]
\textbf{Remarks:}
\begin{itemize}
	\item We are considering $\lambda \in \mathbb{N} \setminus \{0\}$ but we could be interested in a real number. \textit{Solution}: we can multiply both sides.
\end{itemize}
\fi
\end{document}  
