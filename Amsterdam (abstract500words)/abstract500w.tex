\documentclass[a4paper]{article}
%\usepackage[a4paper, total={5.5in, 7.7in}]{geometry}
%\usepackage{rfia2000}
\usepackage[T1]{fontenc}
%\usepackage[english]{babel}
\usepackage{times}
\usepackage{authblk}

%%\author{ID: 2696}
%\author{\begin{tabular}[t]{c@{\extracolsep{3em}}c@{\extracolsep{3em}}c}
%	B. Napolitano${}^1$ & O. Cailloux${}^1$ & P. Viappiani${}^2$ \\
%	\end{tabular}
%{} \\
%\\
%${}^1$        LAMSADE, UMR 7243, CNRS and Universit\'e Paris Dauphine, PSL Research University, Paris, France   \\
%${}^2$        LIP6, UMR 7606, CNRS and Sorbonne Universit\'e, Paris, France
%{} \\
%\\
%%Mon adresse compl\`ete \\
%\{firstname.lastname\}@dauphine.fr, paolo.viappiani@lip6.fr\\
%%{\bf Domaine principal de recherche}: RFP ou IA\\
%%{\bf Papier soumis dans le cadre de la journée commune}: OUI ou NON
%}
\author[1]{Beatrice Napolitano}
\author[1]{Olivier Cailloux}
\author[2]{Paolo Viappiani}
\affil[1]{LAMSADE, UMR 7243, CNRS and Universit\'e Paris Dauphine, PSL Research University, Paris, France}
\affil[2]{LIP6, UMR 7606, CNRS and Sorbonne Universit\'e, Paris, France}
\affil[ ]{ }
\affil[ ]{\{firstname.lastname\}@dauphine.fr, paolo.viappiani@lip6.fr}
\renewcommand\Authands{ and }

%\author{%
%	Beatrice Napolitano \affmark[1], Olivier Cailloux\affmark[1] and Paolo Viappiani\affmark[2]\\
%	\affaddr{\affmark[1]LAMSADE, UMR 7243, CNRS and Universit\'e Paris Dauphine, PSL Research University, Paris, France}\\
%	\affaddr{\affmark[2]LIP6, UMR 7606, CNRS and Sorbonne Universit\'e, Paris, France}\\
%	\email{\{firstname.lastname\}@dauphine.fr, paolo.viappiani@lip6.fr}\\
%}

\thispagestyle{plain}
\pagestyle{plain}

\usepackage[T1]{fontenc}
\usepackage[utf8]{inputenc}
\usepackage{newunicodechar}
\usepackage{textcomp}
\newunicodechar{⇒}{\implies}
\newunicodechar{≠}{\ensuremath{\neq}}
\newunicodechar{≤}{\leq}
\newunicodechar{≥}{\geq}
%… Horizontal Ellipsis
\DeclareUnicodeCharacter{2026}{\ifmmode\dots\else\textellipsis\fi}
\newunicodechar{∧}{\land}
\newunicodechar{∨}{\lor}
\newunicodechar{∩}{\cap}
\newunicodechar{∪}{\cup}
%¬ Not Sign
\DeclareUnicodeCharacter{00AC}{\ifmmode\lnot\else\textlnot\fi}
\newunicodechar{⇔}{\Leftrightarrow}
\newcommand{\N}{ℕ}
\newunicodechar{ℕ}{\mathbb{N}}
\usepackage{algorithm, algpseudocode}
\usepackage{amsmath,amssymb,enumerate,amsthm}
\usepackage{natbib}
\usepackage[strict]{siunitx}
\usepackage{hyperref}
\usepackage{mathrsfs}

\usepackage[small]{caption}
\usepackage{graphicx}
\usepackage{amsmath}
\usepackage{booktabs}

\usepackage{bm}
\usepackage{empheq}
\usepackage{cleveref}
\usepackage{xcolor}

\newcommand{\email}[1]{\href{mailto:#1}{#1}}
\newcommand*{\affaddr}[1]{#1} % No op here. Customize it for different styles.
\newcommand*{\affmark}[1][*]{\textsuperscript{#1}}
\newcommand{\denselist}{\itemsep -2pt\topsep-6pt\partopsep-6pt}
\newcommand{\commentOC}[1]{\textcolor{blue}{\small$\big[$OC: #1$\big]$}}

\newcommand{\pref}{\succ}%real, connected pref, strict
\newcommand{\prefr}{{\succ}^\text{r}}%real, connected pref, strict
\newcommand{\ppreflarge}{\succeq^\text{p}}%partial pref
\newcommand{\ppref}{\succ^\text{p}}%partial pref
\newcommand{\pprefinv}{\prec^\text{p}}%partial pref
\newcommand{\nppref}{\nsucc^\text{p}}%negated partial pref
\newcommand{\linors}{\mathcal{L}(A)}
%https://tex.stackexchange.com/a/45732, works within both \set and \set*, same spacing than \mid (https://tex.stackexchange.com/a/52905).
\newcommand{\suchthat}{\;\ifnum\currentgrouptype=16 \middle\fi|\;}

%Thanks to https://tex.stackexchange.com/q/154549
\makeatletter
\newcommand{\newrelation}[2]{% #1 = control sequence, #2 = replacement text
  \@ifdefinable{#1}{%
    \def#1{%
    \@ifnextchar_{\csname\string#1\endcsname}{\mathrel{#2}}%
    }%
    \@namedef{\string#1}##1##2{\mathrel{#2_{##2}}}%
  }%
}
\makeatother

\newrelation{\pinc}{\!\parallel\!}%partial pref, complement (incomparable)
%\newrelation{\pinc}{Q^\text{p}}%partial pref, complement (incomparable)

\newcommand{\profile}{\bm{v}}%(complete) profile
\newcommand{\pprofile}{{\bm{p}}}%partial profile
\newcommand{\w}{\bm{w}}
\newcommand{\W}{\mathcal{W}}
\newcommand{\Co}{\mathcal{C}}
\newcommand{\pw}{W}%our knowledge about the weights
\newcommand{\powersetz}[1]{\mathscr{P}^*(#1)}
\newcommand{\strat}[1]{\emph{#1}}

\DeclareMathOperator{\Regret}{Regret}
\DeclareMathOperator{\SCORE}{Score}
\DeclareMathOperator{\PMR}{PMR}
\DeclareMathOperator{\MR}{MR}
\DeclareMathOperator{\MMR}{MMR}
\DeclareMathOperator*{\argmax}{argmax}
\DeclareMathOperator*{\argmin}{argmin}

\newtheorem{claim}{Claim}
\newtheorem{prop}{Proposition}
\newtheorem{corollary}{Corollary}
\newtheorem{definition}{Definition}
\newtheorem{example}{Example}

\DeclarePairedDelimiter\set{\{}{\}}
\DeclarePairedDelimiter\card{\lvert}{\rvert}
\DeclarePairedDelimiter\abs{\lvert}{\rvert}

%	\tolerance=2000
%	%Accept overfull hbox up to...
%	\hfuzz=1cm
%	%Reduces verbosity about the bad line breaks.
%	\hbadness 10000
%	%Reduces verbosity about the underful vboxes.
%	\vbadness=10000



\begin{document}
\date{}
\title{\Large\bf Simultaneous Elicitation of Committee and Voters Preferences}

\maketitle

\section*{Abstract}
Social choice deals with the problem of aggregating the preferences of different agents over several alternatives determining the ones that represent a consensus choice. In a traditional social choice setting both the social choice function and the full preference orderings of the agents are expressed beforehand. Nevertheless, in real situations this is not always the case. When considering decisions with large sets of alternatives, requiring agents to express full preference orderings can be prohibitively costly and, perhaps, not necessary in order to determine a consensus choice.

Consider as an example a situation in which the council of a city wants to build a community allotment and the decision of what to grow has to be taken. The agents, or voters, are the residents of the neighborhood and the alternatives consist of all the fruits and vegetables that can grow in that particular area. Imagine now to be an external observer (someone helping with the voting procedure for example) who does not know anything about both residents preferences and the aggregation method chosen by the council. If, after collecting some preferences, we notice that a vegetable is ranked very low by a large number of residents then we are not interested in knowing how a particular agent, we do not know anything about, ranks this vegetable. This is because it is really unlikely that it would be selected as a winner by any reasonable voting rule. A clever approach would be, instead, to ask that agent her preferences regarding alternatives for which, considering the information collected so far, even a single vote can be decisive for them to be selected as consensus choices. Furthermore, following the same reasoning, if we notice that a particular vegetable is the most preferred of almost the totality of the residents we do not need to know the specific voting rule. Any reasonable aggregation procedure, in fact, would select this alternative as a winner. The reason behind this consideration is that is not unrealistic to assume that the committee that has to decide how to aggregate these preferences is not able to define a specific procedure. Indeed, it is often difficult for non-expert users to formalize a voting rule on the basis of some generic preferences over a desired aggregation method.

These observations have motivated a number of recent works considering social choice with partial preference orders \citep{Xia2008, Pini2009, Konczak05} and incremental elicitation \citep{Kalech2011, Lu2011, Naamani-Dery2015} of agents preferences. Furthermore, several authors \citep{Stein1994, Llamazares2013, Viappiani2018} have worked on positional scoring rules with uncertain weights, assuming that the preferences of the agents are fully known. Some elicitation methods for a quite general class of rules based on weak orders have also been proposed by Cailloux and Endriss \citep{Cailloux2014}.

In this work we go one step further and we assume that both the voting rule and the agents preferences are partially specified. We consider positional scoring rules and develop methods for computing the minimax-optimal alternative using the notion of regret. The main idea is to use this procedure both as a means of robust winner determination as well as a guide to the process of simultaneous elicitation of preferences and voting rule. The latter leads to the development of query strategies that interleave questions to the chair and questions to the agents in order to attempt to acquire the most relevant information as quickly as possible.
While previous works have considered either partial information about the agents preferences or a partially specified aggregation method, we do not know of any work considering both sources of uncertainty at the same time.

%, providing incremental elicitation methods that aim to acquire relevant preference information. We then discuss several heuristics that determine queries, either to an agent or to the committee, that quickly reduce minimax regret.
\subsection*{Keywords}
Computational Social Choice, uncertainty in AI, preference elicitation, minimax regret.

\bibliography{biblio}
\bibliographystyle{apalike}
\end{document}  