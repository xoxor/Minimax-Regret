%%%%%%%%%%%%%%%%%%%%%%%%%%%%%%%%%%%%%%%%%%%%%%%%%%%%%%%%%%%%%%%%%%%%%%%%

%%% LaTeX Template for AAMAS-2021 (based on sample-sigconf.tex)
%%% Prepared by Natasha Alechina and Ulle Endriss (version 2020-08-06)

%%%%%%%%%%%%%%%%%%%%%%%%%%%%%%%%%%%%%%%%%%%%%%%%%%%%%%%%%%%%%%%%%%%%%%%%

%%% Start your document with the \documentclass command.
%%% Use the first variant below for the final paper.
%%% Use the second variant below for submission.

\PassOptionsToPackage{hypertexnames=false, pdfusetitle, linkbordercolor={1 1 1}, citebordercolor={1 1 1}, urlbordercolor={1 1 1}}{hyperref}
%\documentclass[sigconf]{aamas} 
\documentclass[sigconf, anonymous]{aamas} 

%%% Load required packages here (note that many are included already).
\input{preamble/packages}
%TODO consider removing anyfontsize and mathrsfs when submitting. They are needed for the \mathscr command at Olivier’s box (probably due to some outdated package) and for removing a warning about rsfs font size.
\usepackage{anyfontsize}
\usepackage{mathrsfs}
%\input{preamble/redac}
%\input{preamble/math_basics}
\newunicodechar{⇒}{\Rightarrow}
\newunicodechar{≠}{\ensuremath{\neq}}
\newunicodechar{≤}{\leq}
\newunicodechar{≥}{\geq}
%… Horizontal Ellipsis
\DeclareUnicodeCharacter{2026}{\ifmmode\dots\else\textellipsis\fi}
\newunicodechar{∧}{\land}
\newunicodechar{∨}{\lor}
\newunicodechar{∩}{\cap}
\newunicodechar{∪}{\cup}
%¬ Not Sign
\DeclareUnicodeCharacter{00AC}{\ifmmode\lnot\else\textlnot\fi}
\newunicodechar{⇔}{\Leftrightarrow}
\newcommand{\N}{ℕ}
\newunicodechar{ℕ}{\mathbb{N}}

\newcommand{\pref}{\succ}%real, connected pref, strict
\newcommand{\prefeq}{\succeq}%real, connected pref, strict
\newcommand{\prefr}{{\succ}^\text{r}}%real, connected pref, strict
\newcommand{\pprefeq}{\succeq^\text{p}}%partial pref
\newcommand{\ppref}{\succ^\text{p}}%partial pref
\newcommand{\pprefinv}{\prec^\text{p}}%partial pref
\newcommand{\nppref}{\nsucc^\text{p}}%negated partial pref
\newcommand{\linors}{\mathcal{L}(A)}
%https://tex.stackexchange.com/a/45732, works within both \set and \set*, same spacing than \mid (https://tex.stackexchange.com/a/52905).
\newcommand{\suchthat}{\;\ifnum\currentgrouptype=16 \middle\fi|\;}

%Thanks to https://tex.stackexchange.com/q/154549
\makeatletter
\newcommand{\newrelation}[2]{% #1 = control sequence, #2 = replacement text
	\@ifdefinable{#1}{%
		\def#1{%
			\@ifnextchar_{\csname\string#1\endcsname}{\mathrel{#2}}%
		}%
		\@namedef{\string#1}##1##2{\mathrel{#2_{##2}}}%
	}%
}
\makeatother

\newrelation{\prefinc}{\!\parallel\!}%partial pref, complement (incomparable)
\newrelation{\pinc}{\bowtie^\text{p}}
%\newrelation{\prefinc}{Q^\text{p}}%partial pref, complement (incomparable)

\newcommand{\profile}{\bm{v}}%(complete) profile
\newcommand{\pprofile}{{\bm{p}}}%partial profile
\newcommand{\w}{\bm{w}}
\newcommand{\W}{\mathcal{W}}
\newcommand{\Co}{\mathcal{C}}
\newcommand{\pw}{W}%our knowledge about the weights
\newcommand{\powersetz}[1]{\mathscr{P}^*(#1)}
\newcommand{\strat}[1]{\emph{#1}}

\newcommand{\intvl}[1]{\llbracket #1 \rrbracket}

\DeclareMathOperator{\Regret}{Regret}
\DeclareMathOperator{\SCORE}{Score}
\DeclareMathOperator{\PMR}{PMR}
\DeclareMathOperator{\MaxR}{MR}
\DeclareMathOperator{\MMR}{MMR}
\DeclareMathOperator{\leximax}{leximax}
\DeclareMathOperator*{\argmax}{argmax}
\DeclareMathOperator*{\argmin}{argmin}

\DeclareMathDelimiter{(}{\mathopen} {operators}{"28}{largesymbols}{"00}
\DeclareMathDelimiter{)}{\mathclose}{operators}{"29}{largesymbols}{"01}

\newtheorem{claim}{Claim}
\newtheorem{prop}{Proposition}
\newtheorem{corollary}{Corollary}
\newtheorem{definition}{Definition}
\newtheorem{example}{Example}

\DeclarePairedDelimiter\set{\{}{\}}
\DeclarePairedDelimiter\card{\lvert}{\rvert}
\DeclarePairedDelimiter\abs{\lvert}{\rvert}

\newtheorem{proof}{Proof}
%for appendix
\usepackage{import}
\usepackage{algorithm, algpseudocode}

\usepackage{balance} % for balancing columns on the final page

\newcommand{\commentOC}[1]{\textcolor{blue}{\small$\big[$OC: #1$\big]$}}
\newcommand{\commentBN}[1]{\textcolor{red}{\small$\big[$BN: #1$\big]$}}

%TODO remove this before submission, check vboxes
\vbadness=10000

%%%%%%%%%%%%%%%%%%%%%%%%%%%%%%%%%%%%%%%%%%%%%%%%%%%%%%%%%%%%%%%%%%%%%%%%

%%% AAMAS-2021 copyright block (do not change!)

\setcopyright{ifaamas} 
\acmConference[AAMAS '21]{Proc.\@ of the 20th International Conference on Autonomous Agents and Multiagent Systems (AAMAS 2021)}{May 3--7, 2021}{London, UK}{U.~Endriss, A.~Now\'{e}, F.~Dignum, A.~Lomuscio (eds.)}
\copyrightyear{2021}
\acmYear{2021}
\acmDOI{}
\acmPrice{}
\acmISBN{}

%%%%%%%%%%%%%%%%%%%%%%%%%%%%%%%%%%%%%%%%%%%%%%%%%%%%%%%%%%%%%%%%%%%%%%%%

%%% Use this command to specify your EasyChair submission number.
%%% In anonymous mode, it will be printed on the first page.

\acmSubmissionID{554}

%%% Use this command to specify the title of your paper.

\title{Simultaneous Elicitation of Scoring Rule and Agent Preferences for Robust Winner Determination}

%%% Provide names, affiliations, and email addresses for all authors.

\author{Beatrice Napolitano}
\affiliation{
	\institution{Université Paris-Dauphine, Université PSL, CNRS, LAMSADE}
	\city{75016 Paris}
	\state{France}}
\email{beatrice.napolitano@dauphine.fr}

\author{Olivier Cailloux}
\affiliation{
	\institution{Université Paris-Dauphine, Université PSL, CNRS, LAMSADE}
	\city{75016 Paris}
	\state{France}}
\email{olivier.cailloux@dauphine.fr}

\author{Paolo Viappiani}
\affiliation{
	\institution{LIP6, UMR 7606, CNRS and Sorbonne Universit\'e}
	\city{Paris}
	\state{France}}
\email{paolo.viappiani@lip6.fr}

%%% Use this environment to specify a short abstract for your paper.

\begin{abstract}
Social choice deals with the problem of determining a consensus choice from the preferences of different agents.
In the classical setting, the voting rule is fixed beforehand and full information concerning the preferences of the agents is provided.
This assumption of full preference information has recently been questioned by a number of researchers.
Methods for eliciting the voting rule when the preferences of the agents are completely known have been proposed, as well as techniques that consider the opposite scenario. %, viewing the voting rule as representing the preference of the chair. 

This article goes one step further by tackling one important new case, namely, when both the voting rule and the agents preferences are partially known. It also permits to obtain a progressively refined recommendation without requiring to achieve full knowledge of either sort of information.
%However, there may be cases where both of them are only partially known.
%This paper considers both sources of uncertainty at the same time, and proposes a method able to recommend a consensus choice without requiring full knowledge of either.
This is useful because providing such information is often costly: the chair may find it hard to specify a rule completely; agents may need time to form reflective preferences.

Focusing on positional scoring rules, we assume that the chair, while not able to give a precise definition of the rule, is capable of answering simple questions requiring to pick a winner from a specific example profile. In addition, the preferences of the agents are incrementally acquired by asking comparison queries. 
In this setting, we propose a method for robust approximate winner determination with minimax regret. We also provide an interactive elicitation protocol based on minimax regret and develop several query strategies that question the chair and the agents in order to acquire the most relevant information for quickly converging to a near-optimal alternative. Finally, we find out experimentally that questioning agents is more useful than questioning the chair in our setting.
\end{abstract}

%%% The code below was generated by the tool at http://dl.acm.org/ccs.cfm.
%%% Please replace this example with code appropriate for your own paper.
%TODO remove this before submission. This only serves to suppress the warning, but nobody seems to want this.
\settopmatter{printccs=false}
\ccsdesc[500]{Computing methodologies~Knowledge representation and reasoning}

%%% Use this command to specify a few keywords describing your work.
%%% Keywords should be separated by commas.

\keywords{Uncertainty in AI, Computational Social Choice, Preference Elicitation}

%%%%%%%%%%%%%%%%%%%%%%%%%%%%%%%%%%%%%%%%%%%%%%%%%%%%%%%%%%%%%%%%%%%%%%%%

%%% Include any author-defined commands here.
         
\newcommand{\BibTeX}{\rm B\kern-.05em{\sc i\kern-.025em b}\kern-.08em\TeX}

%%%%%%%%%%%%%%%%%%%%%%%%%%%%%%%%%%%%%%%%%%%%%%%%%%%%%%%%%%%%%%%%%%%%%%%%

\begin{document}

%%% The following commands remove the headers in your paper. For final 
%%% papers, these will be inserted during the pagination process.

\pagestyle{fancy}
\fancyhead{}

%%% The next command prints the information defined in the preamble.

\maketitle 

%%%%%%%%%%%%%%%%%%%%%%%%%%%%%%%%%%%%%%%%%%%%%%%%%%%%%%%%%%%%%%%%%%%%%%%%

\section{Introduction}
Aggregation of preference information is a central task in many computer systems (recommender systems, search engines, etc).
In many situations, such as in group recommender systems, the goal is to find a consensus choice.
It is therefore natural to look at methods from social choice and see how they can be adapted for group decision-making in a computerized setting.

The traditional approach to social choice assumes that both the social choice function and the full preference orderings of the agents (voters) are expressed beforehand. These represent two strong hypothesis.
Requiring agents to express full preference orderings can be prohibitively costly (in terms of cognitive and communication cost).
This observation has motivated a number of recent works considering social choice with partial preference orders \citep{Xia2008, Pini2009, Konczak05} and incremental elicitation \citep{Kalech2011, Lu2011, Naamani-Dery2015,Benabbou2016} of agent preferences. \\ Furthermore, it is often difficult for non-expert users to formalize a voting rule on the basis of some generic preferences over a desired aggregation method. Thus, the first hypothesis should also be relaxed. 
%Also the first hypothesis should be relaxed considering that, in several situations, it may not be easy to precisely define the voting rule.Indeed, it is possible that the chair has some preferences over the desired aggregation method, but is not able to formalize the voting rule upfront.

%it is possible to derive dominance relations (akin to stochastic dominance) that allow to eliminate some alternatives since they will be less preferred than another one for any instantiation of the weights \citep{Stein1994}.
%Among others, Llamazares and Pe{\~{n}}a have considered  the problem of dealing with underspecified weights in positional scoring rules.

In this paper we focus on positional scoring rules, that are a particularly common method used to aggregate rankings, and we assume that both the agent preferences and the social choice rule are partially specified. We develop methods for computing the minimax-optimal
alternative using positional scoring rules and we provide incremental elicitation methods to acquire relevant preference information. We then discuss several heuristics that determine queries, either to an agent or to the chair, that quickly reduce minimax regret. While previous works have considered either partial information about the agent preferences or a partially specified aggregation method, we do not know of any work considering both sources of uncertainty at the same time.

%In this paper we consider that both the agent preferences and the social choice rule are partially specified and we develop methods for computing the minimax-optimal alternative under these assumptions.
%changed from: We develop methods for computing the minimax-optimal alternative for scoring rules with partial agent preference information and partial information about the weights of the scoring rule.

%We also address the problem of elicitation, providing incremental elicitation methods to acquire relevant preference information. We then discuss several heuristics that attempt to simultaneously reduce the preference and voting rule uncertainty by determining queries, either to an agent or to the chair, that quickly reduce minimax regret.
%In particular, our query strategies focus simultaneously on reduction of relevant preference and voting rule uncertainty.

\paragraph{Experimental conclusions}
Our approach permits to compare experimentally strategies that question the chair, the agents, or both. We describe informally here a few of the conclusions stemming from these experiments, given a problem involving $m$ alternatives and $n$ agents. First, a good quality recommendation can be reached by asking what can be considered a reasonable number of questions to the chair and agents. For example, in a problem of size $m = 5$, $n = 10$, $55$ questions in total are required to reach a good quality recommendation, whcih can be considered reasonable as these questions are to be dispatched over $10$ agents and the chair.
Second, the quality of the recommendation increases faster than linearly with the number of questions, after only a few questions (the first few questions yield no increase in quality).
Third, we experimentally find out that asking questions to the agents only performs similarly than asking questions both to the agents and the chair, although questions to the chair do have a positive value in terms of quality of the recommendation.
%the value of information of one question to the chair is similar to the value of information of one question to a agent (in the ranges we considered), up to a certain proportion of questions asked to the chair. 
Knowing this permits to choose the questioning strategy wisely depending on the estimated cost of each kind of question, which depends on the application. 

The paper is organized as follows. \Cref{sec:related} reviews the works related to our objective.
In Section \ref{sec:background} we provide the necessary background and in Section \ref{sec:mmr} we introduce the minimax criterion, that selects a winner that minimizes the worst possible regret.
Then, in Section \ref{sec:elicit} we provide an interactive elicitation protocol based on minimal regret;  in Section \ref{sec:experiments} we present the empirical validation of our approach with simulations; and Section \ref{sec:conclusions} provides some final thoughts.

\section{Related Work}
\label{sec:related}
In the context of social choice, several authors have been interested in obtaining information about winners with reduced assumptions about the knowledge of the agent preferences (assuming that the voting rule is known). One early work is  by \Citet{Conitzer2005} who studied the complexity of communication, using different voting rules, determining lower and upper bounds; they showed that, in the worst case, the agents should send their complete set of preference for most of the rules they investigated. 

\Citet{Konczak05} were the first to consider the problem of computing, for various voting rules, possible and necessary winners in the case of partial knowledge of agent preferences and showed that the problem is hard in the general case.
This work was then extended by \citet{Xia2008} that showed that, while the identification of a necessary co-winner in scoring rules is polynomial,  the determination of possible co-winners is NP-hard; they also  proposed polynomial-time algorithms when using maximin and Bucklin \citep{Xia2008}.
In other papers it was showed that the general case remains computationally hard even when restricting to single-peaked preferences \citep{Walsh2007};  sufficient conditions that ensure tractability were then found \citep{Pini2007}.

%They also showed that, even though this computation is hard in the general case, their method is polynomial in case of positional scoring rule. 
%Later on, for the necessary winner problem, other authors have proposed polynomial-time algorithms when using maximin and Bucklin \citep{Xia2008}; showed that the general case remains computationally hard even when restricting to single-peaked preferences \citep{Walsh2007}; proved sufficient conditions that ensure tractability \citep{Pini2007}.

Thereafter, noticing that in many practical situations there would be too many possible winners but no necessary winners, the importance of proposing efficient elicitation strategies has been identified. 
Several works addressed the problem of  elicitation of agent preferences \citep{Naamani-Dery2015,Kalech2011,Lu2011,Pini2009,Benabbou2016,Dey2016,Dey2016_2} using a variety of approaches (minimax regret, Bayesian methods, etc.), and others \citep{Walsh2009,Conitzer2009}  analyzed when to stop the elicitation process.

While most of the focus in the research community has been in dealing with incomplete agent preferences, 
%Although more interest has been shown in the incompleteness of agent preferences, 
another line of research has dealt with scenarios where it is instead the voting rule that is partially defined, but the preferences are given. 
A classic paper is the one by \citet{Stein1994} that, considering scoring rules, identified dominance relations between alternatives; these relations allow to determine pairs of alternatives where the first is at least as good as the second no matter which weights are chosen among a generic class of weights (decreasing weighs or convex decreasing weights).

More recently \citet{Llamazares2013} and \citet{Viappiani2018} (among others \commentOC{Which others?}) considered the problem of dealing with unspecified weights in positional scoring rules; the first \citep{Llamazares2013} proposed a method based on Data Envelope Analysis, while the second \citep{Viappiani2018} proposed to aggregate the uncertainty in weights using criteria used in decision-making under uncertainty (such as minimax regret). 
The work of \citet{Cailloux2014} deals with the problem of eliciting a generic voting rule.

Finally we also mention works related to the manipulability of voting rules %have been proposed in this context 
\citep{Elkind2012,Dey2018,Dey2018_2,Conitzer2011}, and work concerning strategic behaviors, when agents learn incrementally about other agents preferences \citep{Endriss2016,Lev2019,Annemieke2012}.

\section{Social choice with partial information}
\label{sec:background}
We now introduce some basic concepts.
We consider a set $A$ of $m$ alternatives (products, restaurants, movies, public projects, job candidates, etc.) and a set $\set{1, …, n}$ of agents. Each agent $j$ comes from an infinite set $\N$ of potential agents and is associated to her “real” preference order ${\pref_j}  \in \linors$ which is a linear order (a connected, transitive, asymmetric relation) over the alternatives.
Following the social choice nomenclature, we call {\em profile} the association of a preference to each agent, considering a subset of agents from the set $\N$, and denote a profile by $(\pref_1,\ldots,\pref_n)$.
A profile is equivalently represented by $\profile=(v_1,\ldots,v_n)$ where $v_j(i) \in \set{1, \ldots, m}$ denotes the rank (position) of alternative $i$ in the preference order $\pref_j$. 

Let $V$ be the set of possible preference profiles (the union, for any integer $n$, of the $n$-fold cartesian product of the linear orders over the alternatives).
A social choice function $f : V \rightarrow \powersetz{A}$ associates a profile with a set of winners, where $\powersetz{A}$ represents the set of subsets of $A$ except for the emptyset (sets are used for tied winners).
Among the many possible social choice functions, we consider {\em positional scoring rules (PSR)}, which attach weights to positions according to the vector $(w_1, \ldots, w_m)$ (also called the scoring vector).
An alternative obtains a score that depends on the rank obtained in each of the preference orders:
\begin{align}
	\label{eq:srule}
	s(x; \profile, \w) = \sum_{j=1}^{n} w_{v_j(x)}
	= \sum_{r=1}^{m} \alpha^{x}_r w_r 
\end{align}
where $\alpha^{x}_r$ is the number of times that alternative $x$ was ranked in the $r$-th position.
The winners are the alternatives with the highest score.

In this work we assume fixed, but unknown to us, a profile $\profile^*$ representing the preferences $\pref_j^*$ of the agents, and a weight vector $\w^*$, representing the preferences of the chair, and we want to reason about partial preference information concerning those objects.
At a given time, our knowledge of agent $j$'s preference is encoded by a partial order over the alternatives, thus a transitive and asymmetric binary relation, denoted by $\ppref_j$. 
In this work we assume that preference information is truthful, i.e. $a \ppref_j b ⇒ a \pref_j^* b$.
%We use $\prefinc$ to denote incomparability, that is $a \prefinc_{j} b$ iff $a \nppref_j b \wedge b \nppref_j a$.
An incomplete profile $\pprofile = (\ppref_1, \ldots, \ppref_n)$ maps each agent to a partial preference.

A completion of $\ppref_j$ is any linear order $\pref$ that extends $\ppref_j$ and we indicate with $C(\ppref_j) = \set{{\succ} \in \linors \suchthat {\ppref_j} \subseteq {\succ}}$ the set of possible completions of $\ppref_i$.
Then $C(\pprofile)=C(\ppref_1)\times … \times C(\ppref_n)$ is the set of complete profiles extending $p$. Note that $\profile^* \in C(\pprofile)$.

We also assume that the weights of the scoring rule are only partially specified.
Therefore, the vector $(w_1,\ldots,w_m)$  is not known but we are given a set of constraints restraining the possible values that weights can take.
We consider a decreasing sequence of weights:
\begin{align}
	1=w_{1} ≥ w_{2} ≥ \ldots ≥ w_{m}=0. \label{eq:monotone}
\end{align}
This is a natural assumption, as it is better to be ranked first than second, second than third, etc. 
Without loss of generality, we consider that $w_1=1$ and $w_m=0$. 

The weights of a scoring rule can model different preferences of the chair. 
For instance, the weights can control the inclination to favor ``extreme'' alternatives (often at either the top or the bottom of the input rankings) at the expenses of ``moderate'' alternatives (that are more consistently in the middle part of the input rankings).

An important class of scoring rule is the one composed of weights that represent a convex sequence \citep{Stein1994,Llamazares2016}, meaning that the difference between the weight of the first position and the weight of the second position is at least as great as the difference between the weights of the second and third positions, etc.
\begin{equation} 
	\label{eq:convexity}
	\forall r \in \{1,\ldots,m-2\}: w_r - w_{r+1} \geq w_{r+1}-w_{r+2}.
\end{equation}
The constraint above is often used when aggregating rankings in sport competitions.
We use $\W$ to denote the set of convex weight vectors.

In general it can be difficult to set the weights in an appropriate way.
We assume that in addition of basic requirements (monotonicity and convexity), the chair (the person or the organization that is supervising the voting process) is able to specify additional preferences about how the social choice function should behave.
In this work we assume that the preferences of the chair are encoded with linear constraints about the vector $\w$, relating the value of the weights of different positions, and the set of these constraints is denoted by $\Co_W$. Moreover, we use $\pw \subseteq \W$ to denote the set of weight vectors compatible with the preferences expressed by the chair about the scoring vector.

Of course it may be difficult for real decision makers to state preferences about the voting rule in such an abstract way.
We will show in \cref{sec:elicit} that the additional preferences we use can be elicited by 
%asking questions about concrete profiles, for instance, by 
showing a complete profile of a small synthetic election and asking who should be elected in this case.

\section{Robust winner determination}
\label{sec:mmr}
In this paper, we consider a setting where both the agent preferences and the preferences of the chair about the voting rule are incomplete.
As reviewed in \cref{sec:related}, some authors have considered possible and necessary winners assuming a partial profile or assuming an incompletely specified scoring rule, but in typical settings, there are no necessary winner and too many possible winners.
In practice it is often useful to declare a winner given partial information.

As a decision criterion to determine a winner, we propose to use minimax regret. 
Minimax regret \citep{Savage1954} is a decision criterion that has been used for robust optimization under data uncertainty \citep{Kouvelis1997} as well as in decision-making with uncertain utility values \citep{Salo2001,Boutilier2006}.
\citet{Lu2011} have adopted minimax regret for winner determination in social choice where
the preferences of agents that are only partially known, while the social choice function is predetermined and known.

In this work, we consider the simultaneous presence of uncertainty in agent preferences and in weights.
Using {\em maximum regret} to quantify the worst-case error, the alternatives that minimize this error are selected as tied winners, providing us with a form of robust optimization.
Intuitively, the quality of a proposed alternative $a$ is how far $a$ is from the optimal one in the worst case, given the current knowledge.

The maximum regret is considered by assuming that an adversary can both 1) extend the partial profile $\pprofile$ into a complete profile, and 2) instantiate the weights choosing among any weight vector in $\pw$, where $\pprofile$ and $\pw$ represent our knowledge so far.
We formalize the notion of minimax regret in multiple steps.
First of all, $\Regret(x, \profile, \w)$ is the “regret” of selecting $x$ as a winner instead of choosing the optimal alternative under $\profile$ and $\w$:
\begin{align} 
	\Regret(x, \profile, \w) = \max_{y \in A} s(y; \profile,\w) - s(x; \profile, \w).
\end{align}
The pairwise maximum regret $\PMR(x,y;\pprofile,W)$ of $x$ relative to $y$ given the partial profile $\pprofile$ and the set of weights $W$
is the worst-case loss of choosing $x$ instead of $y$ under all possible realizations of the full profile {\em and} all possible instantiations of the weights:
\begin{align}
	\PMR(x,y; \pprofile, W) & = \max_{\w \in W} \max_{\profile \in C(\pprofile)} s(y; \profile,\w) - s(x; \profile,\w).
\end{align}

The maximum regret $\MaxR(x;\pprofile,W)$ is the worst-case loss of $x$. That is the loss occurred as the result of an adversarial selection of the complete profile $\profile \in C(\pprofile)$ and of the scoring vector $\w \in W$ that together maximize the loss between $x$ and the true winner under $\profile$ and $\w$:
\begin{align}
	\MaxR(x; \pprofile, W) & = \max_{y \in A} \PMR(x,y; \pprofile, W)\\
	& = \max_{\w \in W} \max_{\profile \in C(\profile)} \Regret(x, \profile, \w).
\end{align}

Finally,  $\MMR(\pprofile,W)$ is the value of minimax regret under $\pprofile$ and $W$, obtained when recommending a minimax optimal alternative $x^*_{\pprofile, W} \in A^*_{\pprofile, W}$:

\begin{align}
	\MMR(\pprofile,W) & = \min_{x \in A} \MaxR(x;\pprofile,W); \\
	A^*_{\pprofile, W} & = \argmin_{x \in A} \MaxR(x;\pprofile,W).
\end{align}
By picking as consensus choice
an alternative associated with minimax regret, we can provide a recommendation that gives worst-case guarantees, giving some robustness in face of uncertainty (due to both not knowing the agent preferences and the weights used in the aggregation). 
In cases of ties in minimax regret, we can either decide to return all minimax alternatives $A^*_{\pprofile, W}$ as winners or to pick just one of them using some tie-breaking strategy.

Observe that if $\MMR(\pprofile, W)\!=\!0$, then any $x^{*}_{\pprofile,W} \in A^*_{\pprofile, W}$ is a necessary co-winner; this means that for any valid completion of the profile and any feasible $w \in W$, $x^{*}_{\pprofile,W}$ obtains the highest score.

We note that our notion of regret gives some cardinal meaning to the scores: instead of just being used to select winners under the corresponding PSR, their differences are considered as representing the regret of the chair.

% Add some general remarks about using minimax regret

\paragraph{Computation of minimax regret}
In order to compute pairwise maximum regret, and therefore minimax regret, we decompose the $\PMR$ into the contributions associated to each agent by adapting the reasoning from \citet{Lu2011}.
The setting is however more challenging due to the presence of uncertainty in the weights.

Recall that, in the computation of $s(x; \profile, \w)$, $w_{v_j(x)}$ represents the score that $x$ obtains in the ranking $v_j$ (see \cref{eq:srule}).
Since scoring rules are additively decomposable, we can consider separately the contribution of each agent to the total score. Thus, we can write the actual regret of choosing $x$ instead of $y$ as
\[
s(y; \profile,\w) - s(x; \profile, \w) = \sum_{j=1}^n w_{v_j(y)} - w_{v_j(x)},
\]
and we can rewrite $\PMR$ as follows:

\begin{align}
	& \PMR(x,y; \pprofile, W) = \max_{\w \in W} \max_{\profile \in C(\pprofile)} [ s(y; \profile,\w) - s(x; \profile,\w) ] \\
	& =  \max_{\w \in W} \sum_{j=1}^{n} \max_{v_j \in C(\succ_j^p)} [w_{v_j(y)} - w_{v_j(x)}]. \\
\end{align}
Note that in general the inner max depends on the weights chosen by the outer max.

We are interested in computing $\PMR(x, y; \pprofile, W)$. This represents the “worst” difference of score, thus the difference of score between $y$ and $x$ under the worst case preferences compatible with $\pprofile$ and $W$, where the worst case is the one that maximizes this difference of score.
We consider now a procedure for completing a partial profile that was first proposed by \citet{Lu2011} when considering %minimax regret with 
a fixed weight vector.
As we will show, this procedure can also be used when the weight vector is not completely known.

\begin{claim} \label{claim:completion}
	There exists a completion $\hat{\profile} \in C(\pprofile)$ of the profile $\profile$ such that $\PMR(x,y; \pprofile, W) = \max_{\w \in W} [ s(y; \hat{\profile}, \w) - s(x; \hat{\profile}, \w) ]$ and such that the linear order $\hat{v}_{j}$ of each agent $j$ satisfies:
	\begin{align} 
		\label{eq:complx}
		a \pref_j x &⇔ ¬(x \pprefeq_j a)\\
		\label{eq:comply}
		y \pref_j a &⇔ ¬(a \pprefeq_j y) ∧ ¬((x \pprefeq_j y) ∧ ¬(x \pprefeq_j a)).
	\end{align} 
\end{claim}
\begin{proof}[Sketch of proof]
	Consider our knowledge $\pprefeq_j$ about the preference of the agent $j$. 
	The adversary's goal is to make the score of $y$ as high as possible and the score of $x$ as low as possible. 
	To do this, he should complete $\ppref_j$ to $\pref_j$ by putting above $x$ as many alternatives as he can, that is, all the alternatives except those that are known to be worse than $x$ (those $a$ such that $x \pprefeq_j a$); and similarly, he should put below $y$ all the alternatives he can. Two conditions must be excluded for $a$ to go below $y$. The alternatives such that $a \pprefeq_j y$ can’t be put below $y$.
	Furthermore, the first objective must take priority over the second one: when an alternative should go above $x$ according to the first objective (because $¬(x \pprefeq_j a)$), and $x$ is known to be better than $y$ (thus $x \pprefeq_j y$), then $a$ should be put above $x$ (irrespective of whether $a \pprefeq_j y$), which will move both $x$ and $y$ one rank lower than if $a$ had been put below $y$. 
	This maximizes the adversary’s interests: because the weight vector is convex, the difference of scores will be lower when both alternatives are ranked lower (Equation \ref{eq:convexity}), and that difference of scores is in favor of $x$ when $x \ppref_j y$, thus to be minimized from the point of view of the adversary.
\end{proof}

Let ${\pprefeq_j}(x)$ designate the set of alternatives known to be considered by $j$ as less good than or equal to $x$, and ${\pprefinv_j}(y)$ be the set of alternatives known to be considered by $j$ as strictly better than $y$.
\begin{claim}
	The rank of $x$ in the PMR-maximizing linear orders of agent $j$ is $\hat{v}_{j}(x) = 1+\card{A}-\card{{\pprefeq_j}(x)}$, and the rank of $y$ is $\hat{v}_{j}(y)=1+\card{{\pprefinv_j}(y)}+\card{\beta}$, where $\card{\beta} = \card{A \setminus ({\pprefeq_j}(x) \cup {\pprefinv_j}(y))}$ if $(x \pprefeq_j y)$ and $\card{\beta} = 0$ otherwise.
\end{claim}
\begin{proof}
	The rank of $x$ is directly obtained from \cref{eq:complx}. The rank of $y$ is obtained by complementing \cref{eq:comply}, obtaining $a \prefeq_j y ⇔ (a \pprefeq_j y) ∨ ((x \pprefeq_j y) ∧ ¬(x \pprefeq_j a))$, and, observing that $a \pref_j y ⇔ a ≠ y ∧ a \prefeq_j y$, obtaining that $a \pref_j y$ if and only if
	\begin{equation}
		\label{eq:betteryinter}
		(a \neq y) ∧ [(a \pprefeq_j y) ∨ ((x \pprefeq_j y) ∧ ¬(x \pprefeq_j a))],
	\end{equation} 
	or equivalently, if and only if
	\begin{equation}
		\label{eq:bettery}
		%a \pref_j y ⇔ 
		(a \ppref_j y) ∨ ((x \pprefeq_j y) ∧ ¬(x \pprefeq_j a)).
	\end{equation} 
	Indeed, \eqref{eq:betteryinter} $⇒$ \eqref{eq:bettery}, and \eqref{eq:bettery} $⇒$ \eqref{eq:betteryinter} because $(x \pprefeq_j y) ∧ ¬(x \pprefeq_j a) ⇒ a ≠ y$ (as when $a = y$, $(x \pprefeq_j y)$ and $¬(x \pprefeq_j a)$ are opposite claims). Suffices now to rewrite \cref{eq:bettery} to let the two disjuncts designate disjoint sets:
	\begin{equation}
		\label{eq:betteryfinal}
		a \pref_j y ⇔ 
		(a \ppref_j y) ∨ ((x \pprefeq_j y) ∧ ¬(x \pprefeq_j a) ∧ ¬(a \ppref_j y)).
	\end{equation}
\end{proof}

The claim can also be understood by observing that in the case $(x \pprefeq_j y)$, $\beta$ is the number of alternatives incomparable with both $x$ and $y$.

\begin{claim}
	The $\PMR$ can be written as:
	\begin{align} 
		\PMR(x,y; \pprofile, W)  
		& = \max_{w \in W} \sum_{j=1}^n w_{\hat{v}_j(y)} - w_{\hat{v}_j(x)} = \\ 
		& = \max_{w \in W} \sum_{r=1}^m (\hat{\alpha}_{r}^{y} - \hat{\alpha}_{r}^{x}) w_i. 
	\end{align}
	where $\hat{\alpha}_{r}^{y}$ (resp. $\hat{\alpha}_{r}^{x}$)  is the number of times $y$ (resp. $x$) is at rank $r$ in the complete profile $\hat{\profile}$.
\end{claim}
The last claim shows that PMR is linear in the weights.
Recall that the preferences of the chair are encoded with linear constraints $\Co_{W}$.
The pairwise max regret $\PMR(x,y; \pprofile,W)$ can be obtained as the solution of the following linear program defined on the variables $w_1, …, w_m$, which represent the weights attached to different positions:

\begin{align}
	\max_{\w} & \sum_{r=1}^m (\hat{\alpha}_{r}^{y} - \hat{\alpha}_{r}^{x}) w_{r}\\
	\text{ s.t. } & \text{\cref{eq:monotone}} \text{ and } \text{\cref{eq:convexity}} \text{ and } \Co_W.
\end{align}
Note that given our choice $w_{1}=1$ and $w_{m}=0$, there are only $m-2$ variables 
(we leave $w_{1}$ and $w_{m}$ in the LP just for clarity of presentation).

The max regret $\MaxR(x; \pprofile, W)$ is determined by considering the pairwise regret of $x$ with all other alternatives in $A$.
Optimal alternatives w.r.t. minimax regret are the ones with least max regret. 
Observe that, whenever the $\PMR$ of an alternative $x$ (against some other alternative $y$) exceeds the best MR value found so far, we do not need to further evaluate $x$. 
This idea can be exploited further by adopting a minimax-search tree \citep{Braziunas2011}.

\section{Interactive Elicitation} 
\label{sec:elicit}
We propose an incremental elicitation method based on minimax regret.
At each step, the system may ask a question either to one of the agents about her preferences or to the chair about the voting rule. 
The goal is to obtain relevant information to reduce minimax regret as quickly as possible.
As termination condition of elicitation, we can check whether minimax regret is lower than a threshold; if we wish optimality, we can perform elicitation until minimax regret drops to zero.

The remainder of this section is structured as follows.
First of all, we discuss the different types of questions that can be asked to the agents and to the chair, and the way responses are handled.
Then, we describe different strategies to determine informative queries to ask next, with the goal of reducing $\MMR(\pprofile,W)$ quickly.

\paragraph{Question types}
We distinguish between questions asked to the agents and questions asked to the chair.
As {\em questions asked to the agents} it is natural to consider comparison queries asking to compare two alternatives.
%Another common type of queries are {\em top-k}, asking to each agent her $k$ most preferred alternatives.
The effect of a response to a question asked to an agent is the increase in our knowledge about the agent rankings, thus augmenting the partial profile $\pprofile$. 
If agent $j$ answers a comparison query stating that alternative $a$ is preferred to $b$, then the partial order $\ppref_j$ is augmented with $a \ppref_j b$ and by transitive closure.

A bit more discussion is needed about {\em questions asked to the chair}.
Such questions aim at refining our knowledge about the scoring rule; a response gives us a constraint on the weight vector $\w$.
In particular, we want to obtain constraints of the type
\begin{align}
	w_{r} - w_{r+1} \geq \lambda (w_{r+1} - w_{r+2})
\end{align}
for $r \in \{1,\ldots,m-2\}$, relating the difference between the importance of ranks $r$ and $r+1$ with the difference between ranks $r+1$ and $r+2$.

\paragraph{Building concrete questions for the chair}
As we assume that the weights constitute the utility components of the chair, it might be reasonable to assume that the chair is able to answer such abstract questions in our setting. However, it is important to make sure that a question can also, in principle, be asked in a more concrete way, in terms of winners of example profiles. This permits to test how the chair understands the question and to relate the preference of the chair to her choice behavior in the economic sense. 
Furthermore this will be necessary for an ordinal extension of our work where the scores would not be considered as cardinal utilities.
Thus, our task is to build a profile, given $\lambda$ and $r ≤ m-2$, in such a way that the set of (tied) winners picked by the chair reveals whether $w_{r} - w_{r+1} \geq \lambda (w_{r+1} - w_{r+2})$.
\begin{claim}
	Given a rational $\lambda = p/q$ and a rank $r$ between $1$ and $m - 2$, the profile named $P'$ in the ensuing description is such that, for any weight vector $\w \in \W$, $a \in f(P')$ iff $w_{r} - w_{r+1} ≥ \lambda (w_{r+1} - w_{r+2})$ and $b \in f(P')$ iff $w_{r} - w_{r+1} ≤ \lambda (w_{r+1} - w_{r+2})$, where $f$ is the PSR parameterized with $\w$.
\end{claim}
\begin{proof}
	Observe that the question may be defined equivalently as $q \cdot w_{r} + p \cdot w_{r+2} ≥ (p + q) \cdot w_{r+1}$, where $p, q$ are natural numbers. 
	As a first attempt to make this question concrete, we could define a profile $P$ containing $p+q$ agents in such a way that an alternative $a$ receives $q$ times the rank $r$ and $p$ times the rank $r+2$, and an alternative $b$ receives $p+q$ times the rank $r+1$. 
	Observing that the score of $a$ in that profile is exactly the left hand side of the question, and that the score of $b$ is the right hand side, intuition suggests that observing whether the chair picks $a$ or $b$ as winner will let us determine which side is greater; equality occurring when the chair declares $a$ and $b$ as tied for the victory. 
	However, we still need to complete the profile, thus, come up with other $m-2$ alternatives defined so that each agent in the resulting profile has placed exactly one alternative in each rank.
	And we must ensure, doing this, that these other alternatives are not better than $a$ or $b$: if the chair picks a different alternative $c$, it tells us that the chair prefers $c$ to both $a$ and $b$, 
	but it generally does not  answer the question we are interested in. 
	Taking $p = q = 1$, $m=4$, $r=2$, we see that such a completion may be impossible. Luckily, this problem can be worked around, at the price of increasing the number of agents. 
	
	First build a temporary profile $P$ of $p+q$ agents with $a$ and $b$ ranked as just described. Complete it with $m-2$ alternatives ranked in arbitrary orders so that the resulting profile has complete strict rankings for each of the $p+q$ agent. Now we will make the other alternatives as bad as desired by adding agents to $P$ that appreciate $a$ and $b$ more than the other alternatives, thus building a profile $P'$ in which, whatever the weights, $c$ may not have a better score than $a$. Observe that if we add $\delta$ agents (with $\delta$ a natural number) that put $a$ at first rank and $b$ at second rank, and $\delta$ agents that put $b$ at first rank and $a$ at second rank, we do not change the difference of the scores of $a$ and $b$, and thus, observing the chair choosing $a$ or $b$ still answers the question.
	% Choose arbitrarily a matching of the remaining $m-2$ alternatives to the ranks $3$ to $m$, say, $c \mapsto 3$, $d \mapsto 4$, and so on. The first $\delta$ supplementary agents, who rank $a$ first and $b$ second, rank the other alternatives as the just defined mapping suggests. The other $\delta$ supplementary agents, who rank $b$ first and $a$ second, use the opposite ranking for the remaining alternatives, say, $c \mapsto m$, $d \mapsto m-1$, and so on. 
	
	We can prove that, for $\delta = p+q$, and whatever the weight vector $\w \in \W$, no alternative, except possibly $b$, has a better score than $a$. 
	(In fact, as the construction will exhibit, they are all worst than both $a$ and $b$, but the weaker fact is enough for our claim.)
	Pick any alternative $c$ that is not $a$ or $b$. 
	%Let $c_t$ denote the number of times $c$ reaches rank $t$ or better in $P$. Note that $c_m = p+q$ and $c_t ≤ p + q, 1 ≤ t ≤ m$. Define $a_t$ similarly, for the alternative $a$. By construction, $a_r = q$, $a_{r+2} = q+p = a_m$. 
	Define $c'_t$ as the number of times $c$ gets rank $t$ or better in $P'$, and define $a'_t$ similarly. Now $a'_1 ≥ \delta, a'_t ≥ 2\delta \text{ for } 2 ≤ t < m, a'_{r+2} = q+p+2\delta = a'_m$. To obtain upper bounds for $c'_t$, assume $c$ is ranked first by the $p+q$ agents in $P$. By our construction, the  $2 \delta$ new agents of $P'$ give to $c$ the ranks $(3, m)$, or $(4, m-1)$, … Observe that the score of $c$ is maximal for attribution $(3, m)$, by convexity of the weights. Therefore, in the best case for $c$, $c'_1 ≤ p + q, c'_2 ≤ p+q, c'_t ≤ p+q+\delta \text{ for } 3 ≤ t < m, c'_m ≤ p+q+2\delta$. Observe that $a'_t ≥ c'_t, 1 ≤ t ≤ m$ (with strict inequality for $t=2$). Because weights are non increasing, this guarantees, with $\delta = p + q$, that the score of $a$ is not lower than the one of $c$.
\end{proof}
At this stage, we are satisfied that a procedure exists to transform our abstract questions to questions about winners of a PSR. Further studies would investigate, for example, what is lost in terms of elicitation efficiency when we are forced to restrain to realistic concrete questions, meaning, questions involving small profiles; or investigate the relationship between our understanding of scores as (cardinal) utility and (ordinal) scores as definition of a PSR.

\paragraph{Elicitation strategies}
We develop several elicitation strategies for simultaneous elicitation of agent preferences and of the scoring rule.
While it is of course possible to first fully elicit the agent preferences and afterwards elicit weights, we want to investigate experimentally approaches that are able to recommend winning alternatives before obtaining complete knowledge of the profile or the rule.
%Indeed, it can be beneficial to split efforts asking questions to the chair and to agents, depending on which is estimated to be more informative.
We define here the various strategies we tested experimentally. A strategy tells us, given the current partial knowledge $(\pprofile, W)$, which question should be asked next.

The \strat{Random} strategy gives a baseline for comparison and informs about the difficulty of an elicitation problem. 
This strategy first chooses equiprobably whether it will ask a question about weights or a question about a preference ordering. If it opted for a question about weights, it draws one rank in $1 ≤ r ≤ m-2$ equiprobably, takes the middle of the interval of values for $\lambda$ that are still possible considering our knowledge so far, and asks whether $w_r - w_{r+1} ≥ \lambda (w_{r+1} - w_{r+2})$. The intervals are initialized to $[1, n]$. If it opted for a question to agents, it draws equiprobably among the agents whose preference is not known entirely; draws an alternative $a$ equiprobably among those involved in some incomparabilities in $\ppref_j$; and draws an alternative $b$ equiprobably among those incomparable with $a$ in $\ppref_j$.

Let $(x^{*},\bar{y}, \bar{\profile}, \bar{\w})$ be the current solution of the minimax regret, where $\bar{y}, \bar{\profile}, \bar{\w}$ are, respectively, the choice of $y$, the completion of the profile and the selection of weights for which $x^{*}$ is the minimax optimal alternative. 
The \strat{Pessimistic} strategy considers a set of $n + m$ candidate questions: one per agent, and one per rank.
The candidate questions to the agents use the heuristic proposed by \citet{Lu2011}: for each agent $j$, if $x^*$ and $\bar{y}$ are incomparable in $\ppref_j$, the candidate question concerns the pair $(x^*, \bar{y})$, otherwise, the candidate question concerns the pair $(x^*, z)$ for some $z$ incomparable to $x^*$ (randomly chosen), or if none such $z$ exist, the pair $(\bar{y}, z)$ for some $z$ incomparable to $\bar{y}$, or, if both $x^*$ and $\bar{y}$ are comparable to every alternatives in $\ppref_j$, any incomparable pair is picked at random. 
To understand the intuition behind favoring these two alternatives, recall that, as shown in the proof of \cref{claim:completion}, in order to increase $\PMR(x^{*},\bar{y})$, an adversary should place as many alternatives as she can above $x^{*}$ if $x^* \ppref_j \bar{y}$, and between $\bar{y}$ and $x^{*}$ if $\bar{y} \ppref_j x^*$.  Asking questions involving $x^*$ or $\bar{y}$ thus has a particular chance of reducing usefully the possibilities of the adversary.
The candidate questions to the chair are determined as in the Random strategy.
The strategy then selects among these $n + m$ candidate questions one that leads to minimal regret in the worst case, considering both possible answers to the question, and with penalty terms depending on the kind of question. To define this comparison precisely, assume that a question $q_1$ has type $t_1$ (being “chair” or “agent”), and leads to the possible new knowledge states $(\pprofile_1, W_1)$ and $(\pprofile'_1, W'_1)$, depending on the answer. 
Define $R^{\max}_1 = \max\set{\MMR(\pprofile_1, W_1), \MMR(\pprofile'_1, W'_1)}$
and $R^{\min}_1 = \min\set{\MMR(\pprofile_1, W_1), \MMR(\pprofile'_1, W'_1)} p(t) + p'(t)$.
%If $(\pprofile_1, W_1) ≥ (\pprofile'_1, W'_1)$, define $\MMR^{\max} = \MMR(\pprofile_1, W_1)$ and $\MMR^{\min} = \MMR(\pprofile'_1, W'_1)$; otherwise, define $\MMR^{\max} = \MMR(\pprofile'_1, W'_1)$ and $\MMR^{\min} = \MMR(\pprofile_1, W_1)$.
% and and $(\pprofile^\min_2, W^\min_2)$. numbering them so that $\MMR(\pprofile_1, W_1) ≥ \MMR(\pprofile_2, W_2)$. Then the badness of the question in the worst case is:
The terms $p(t)$ and $p'(t)$ are real numbers depending on the kind of question. These parameters are to be determined experimentally.
Define similarly $t_2$, $R^{\max}_2$ and $R^{\min}_2$ for a question $q_2$.
Then, the Pessimistic strategy considers question $q_1$ more suitable than $q_2$ iff
\begin{align}
	R^{\max}_1 ≤ R^{\max}_2 \text{ or } [R^{\max}_1 = R^{\max}_2 \text{ and } R^{\min}_1 ≤ R^{\min}_2].
\end{align}
This comparison gives a way of picking questions among a set of possible questions, by picking one that minimizes this approximate measure of minimax regret {\em a posteriori}. 
We use this operator that is similar to a $\max$ operator because \citet{Cailloux2014} reported that pessimistic aggregation works much better than optimistic aggregation, although our experiments in this context suggested a weak impact of that choice on the performance of the strategy.
However, to avoid the absorption property of max, we adopt a criterion closer to $\leximax$ as an aggregator: if the maximal MMR of two questions are (nearly) equal, then it considers the penalized minimal MMR values, preferring the question with the lowest value.
(Technically, instead of applying a pure lexicographic aggregation, which is very sensitive to small errors due to floating-point computations, we apply a weighted sum between the maximal and the penalized minimal MMR values, with a predominant weight to the maximal one.)

The \strat{Extended pessimistic} strategy uses the same criterion as the pessimistic strategy, but extending it to a bigger set of $m^2 n + m$ candidate questions, the same as those considered by the Random strategy.
After having selected the candidate questions, we apply the same operator as for the Pessimistic strategy: we evaluate, for each of the possible questions, the minimax regret for the two possible answers and we choose the best according to (quasi-)$\leximax$.
We use this strategy to test whether Pessimistic performs well: depending on the quality of the heuristic of the Pessimistic strategy, it might perform nearly equally well, or perhaps even better, than Extended pessimistic, while being (much) faster. Extended pessimistic is applicable only to small problem instances: its complexity is in $O(n^2 m^5)$, because we consider $O(m^2)$ questions for each agent and need for each question to compute $\MMR$ twice, whose complexity is $O(nm^3)$.

The \strat{Two phases} strategy is developed in order to investigate the effect of varying the proportion of questions asked to agents and to the chair, and compare the performance of asking questions to the chair first or to the agents first. It is parameterized by $q_c$, a value indicating the number of questions that must be asked to the chair.
The \strat{Two phases - ca} variant first asks $q_c$ questions to the chair, then $k - q_c$ questions to the agents, using in both cases Pessimistic to select the specific questions; whereas the \strat{Two phases - ac} variant starts with $k - q_c$ questions to the agents, then questions the chair. 
%Note that when asking first only questions to the chair, if the obtained knowledge approximates well the scoring vector, then in the second part of the elicitation we fall into the more classical setting of incompleteness of preferences assuming a known voting rule. 
%And vice-versa when asking first questions to the agents, the second part of the elicitation is similar to the setting of incompletely specified scoring rule. 
%This strategy also permits to simulate the current state of the art when considering incomplete preferences. By asking enough questions to the chair we can reduce our problem into a well studied one of incompleteness of preferences knowing the voting rule. Similarly, if we invert to whom ask questions first, we fall instead in an already studied case of uncertainty of the voting rule with a known complete profile. 

Finally, the \strat{Elitist} strategy aims at uncovering as quickly as possible the top alternative of each agent. It asks $m - 1$ questions to each agent in turn. 
Given an agent to query $j$, it asks about an alternative currently undominated in $\ppref_j$ and any alternative that it has not yet asked about for that agent, or equivalently, any alternative that is currently incomparable to this alternative. This approach guarantees that the top alternative in the ranking of $j$ will be known after having asked $j$ exactly $m-1$ questions.
After having asked $n (m-1)$ questions to the agents, it questions the chair only, using the same approach as the Pessimistic strategy.
This strategy can be expected to perform well when the chair assigns an important weight to the first rank, as compared to the other ranks. It will be useful to further challenge the Pessimistic strategy, which is not specifically tailored to such a situation.

%\item {\em Volumetric} strategy: chooses an agent $i$ and a query that maximizes the number of new pairwise preferences revealed given the worst response.

\section{Empirical Evaluation} 
\label{sec:experiments}
\commentOC{
	\begin{enumerate}
		\item Comment on Table Pessimistic VS Elitist Geometric
		\item Comment on Table Pessimistic VS Elitist VS Random
		\item Waiting for the result of 2 more lines (p=15,30) in table two phases
		\item Reach regret: $n/10$. For this, we need $m^x n$ questions. (Correct formula)
		\item Check Pess VS XPess on other sizes.
		\item Prop qC in value of information
		\item Time comparison XPess VS Pess? (If time, check our estimate)
	\end{enumerate}
}
Our experiments are described by topic here below. We generally use the following protocol, deviations are indicated below for each experiment.

Having picked a problem size $(m, n)$, a number of questions $k$ and a strategy to test, we randomly generate an “oracle”, containing the true preferences of the agents (i.e. the linear orders) and the weights associated with the chair's scoring rule. 
The preferences are generated following an impartial culture assumption: the linear order of each agent is generated independently from the other ones and all have the same probability of appearing. The weights are generated by drawing $m-1$ differences randomly with uniform distribution, normalizing them, then ordering them, in order to ensure convexity.
The penalty parameters for the Pessimistic and Extended pessimistic strategies are set to $p(t) = 1.1$ and $p'(t) = 10^{-6}$ when $t = $“chair” and $p(t) = 1.0$ and $p'(t) = 0$ when $t = $“agent”.

We start with empty knowledge ($\pprofile = (\emptyset, \emptyset, …), W = \W$) about the preference orderings of the agents or the weight differences favored by the chair. We obtain the first question to be asked using the strategy under test, as described above. We use the oracle to answer the question and update our knowledge, which is thus used to obtain the next question. This is repeated until $k$ answers have been obtained, compute the resulting $\MMR$ values along the way for various values of $k$. We repeat this whole experiment a variable number of times indicated below, for a given $(m, n, k)$, and report the average resulting $\MMR$ and the standard deviation \textit{sd}.

\subsection{Relative evaluation of Pessimistic strategy}
\begin{table}
	\begin{center}
		\resizebox{\columnwidth}{!}{
		\begin{tabular}{S[table-figures-integer=3, table-number-alignment = right, table-figures-decimal=0]S[table-number-alignment = right]@{ ± }S[table-number-alignment = left]S[table-number-alignment = right]@{ ± }S[table-number-alignment = left]S[table-number-alignment = right]@{ ± }S[table-number-alignment = left]S[table-number-alignment = right]@{ ± }S[table-number-alignment = left]S[table-number-alignment = right]@{ ± }S[table-number-alignment = left]}
			\toprule
			{k} & {Rnd.} & {sd} & {Pes.} & {sd} & {Ex.\ pes.} & {sd} & {Eli.} & {sd} \\
			\midrule
			0 & 10.0 & 0.0 & 10.0 & 0.0 & 10.0 & 0.0 & 10.0 & 0.0\\
			20 & 9.06 & 0.49 & 5.85 & 0.87 & 6.17 & 0.78 & 6.32 & 0.61\\
			40 & 7.35 & 0.82 & 2.57 & 1.05 & 2.98 & 0.99 & 3.37 & 0.83\\
			60 & 5.74 & 1.03 & 0.67 & 0.67 & 0.77 &0.78 & 2.33 & 1.05\\
			80 & 4.10 & 1.15 & 0.03 & 0.12 & 0.05 & 0.18 & 2.28 & 1.05\\
			100 & 2.63 & 1.19 & 0.0 &  0.0 & 0.0 & 0.0& 2.28 & 1.05\\
			\bottomrule
		\end{tabular}
		}
	\end{center}
	\caption{Average MMR in problems of size $(5, 10)$ after $k$ questions and $200$ runs.}
	\label{tab:smallSize}
\end{table}

\begin{figure}
	\centering
	\includegraphics[width=.45\textwidth]{comparison.png}
	\caption{Plot in support of \cref{tab:smallSize}}
	\label{fig:smallSize}
\end{figure}

Our first experimental result concerns small size situations. We picked $m = 5, n = 10$ to illustrate our results (variations around this size yield similar conclusions). 
\Cref{tab:smallSize} and \cref{fig:smallSize} compares some of our strategies in this case.
We see that that Pessimistic performs better than Random and Extended pessimistic on a small size problem.
(These figures also display the performance of the Elitist strategy, to which we will come back.)

We observe that the Pessimistic strategy is able to reduce the regret almost completely after $30$ questions. We also see that Pessimistic performs slightly better than Extended pessimistic, showing that Pessimistic chooses candidate questions wisely; this is good news since (for $m = 5$ and $n = 10$), the latter strategy is much faster and takes only $15s$ for a complete elicitation session, while the former takes $80s$.

Asking Random questions does not permit to reach a low regret level even after having asked 100 questions, whereas a low regret level (MMR = 1) is reached by Pessimistic before having asked 60 questions.

We also see that Extended pessimistic performs slightly less well than Pessimistic. Although they come close, Extended pessimistic performs significantly less well, in the sense that the numbers are consistently in favor of Pessimistic if running the experiment again.

We next focus on the Pessimistic strategy, because Extended pessimistic is too slow to be ran repeatedly on much bigger problem sizes.
\begin{table}
	\begin{center}
			\begin{tabular}{S[table-figures-integer=3, table-number-alignment = right, table-figures-decimal=0]S[table-number-alignment = right]@{ ± }S[table-number-alignment = left]S[table-number-alignment = right]@{ ± }S[table-number-alignment = left]S[table-number-alignment = right]@{ ± }S[table-number-alignment = left]S[table-number-alignment = right]@{ ± }S[table-number-alignment = left]}
				\toprule
				{k} & {Rnd.} & {sd} & {Pes.} & {sd} & {Eli.} & {sd} \\
				\midrule
				0 &20.0 & 0.0 & 20.0 & 0.0&  20.0 & 0.0 \\
				50 & 19.82 & 0.04 &	15.59 &0.56	&17.38& 0.7\\
				100 & 19.12	& 0.26&	11.91&	1.26 &15.23& 0.73\\
				150 & 18.06	& 0.47&	8.92&	1.53 &13.38& 1.11\\
				200 & 16.89	& 0.62&6.24&2.02&11.76& 1.03\\
				250 & 15.63 & 0.84&	4.64&1.72&10.56& 0.89\\
				300 & 14.72	& 0.94&	3.01& 1.35 &10.48& 0.87\\
				350 & 13.41	& 1.15& 1.76& 0.95&10.48& 0.86\\
				400 & 11.98	& 0.68&	0.62&0.56 &10.48& 0.86\\
				\bottomrule
			\end{tabular}
	\end{center}
	\caption{Average MMR in problems of size $(10, 20)$ after $k$ questions and $10$ runs.}
	\label{tab:biggerSize}
\end{table}

\begin{table}
	\begin{center}
		
			\begin{tabular}{S[table-figures-integer=3, table-number-alignment = right, table-figures-decimal=0]S[table-number-alignment = right]@{ ± }S[table-number-alignment = left]S[table-number-alignment = right]@{ ± }S[table-number-alignment = left]S[table-number-alignment = right]@{ ± }S[table-number-alignment = left]}
				\toprule
				{k} & {Pes.} & {sd} & {Eli.} & {sd} \\
				\midrule
				0	&20.0	&0.0	&20.0	&0.0\\
				50	&15.96	&0.54	&17.26	&0.42\\
				100	&12.48	&0.93	&15.6	&0.43\\
				150	&9.58	&1.37	&13.94	&0.76\\
				200	&7.43	&1.25	&10.95	&1.12\\
				250	&5.26	&1.52	&6.6	&0.79\\
				300	&3.47	&1.32	&6.57	&0.79\\
				350	&1.75	&1.36	&6.57	&0.79\\
				400	&0.49	&0.68	&6.57	&0.79\\
				\bottomrule
			\end{tabular}
	\end{center}
	\caption{Average MMR in problems of size $(10, 20)$ and geometric weights after $k$ questions and $10$ runs.}
	\label{tab:geometricWeights}
\end{table}

\subsection{Absolute evaluation of Pessimistic strategy}
\label{sec:lowRegret}
\begin{table}
	\begin{center}
		\begin{tabular}{S[table-figures-integer=1, table-number-alignment = center, table-figures-decimal=0]S[table-figures-integer=1, table-number-alignment = center, table-figures-decimal=0]S[table-number-alignment = center]S[table-figures-integer=1, table-number-alignment = center, table-figures-decimal=0]S[table-figures-integer=1, table-number-alignment = center, table-figures-decimal=0]}
			\toprule
			{m} & {n} & {$\frac{n}{10}$} & {$k_{\MMR}$} & {$0.4 m^{1.6} n$} \\
			\midrule
			5 & 10 & 1.0 & 60 &  55 \\
			5 & 15 & 1.5 & 80 &  83 \\
			5 & 20 & 2.0 & 110 &  110 \\
			10 & 20 & 2.0 & 340 &  341 \\
			10 & 30 & 3.0 & 527 &  512 \\
			15 & 30 & 3.0 & 919 &  991 \\
			\bottomrule
		\end{tabular}
	\end{center}
	\caption{Number of questions needed by Pessimistic strategy to reach $\MMR=\frac{n}{10}$ (represented by $k_{\MMR}$), for various problem sizes.}
	\label{tab:lowRegret}
\end{table}
\begin{table}
	\begin{center}
		\begin{tabular}{S[table-figures-integer=2, table-number-alignment = center, table-figures-decimal=0]S[table-figures-integer=1, table-number-alignment = center, table-figures-decimal=2]}
			\toprule
			m & k \\
			\midrule
			5 & 6.6\\
			10 & 24.8\\
			15 & \\
			\bottomrule
		\end{tabular}
	\end{center}
	\caption{Number of questions needed by the Pessimistic strategy to elicit the complete preference of one agent.}
	\label{tab:fullProfile}
\end{table}

The previous section compares Pessimistic to other strategies. In this set of experiments, we wonder about the absolute quality of Pessimistic. For practical usability, it is important to know how many questions should be asked in order to reach a low regret level. Here, we consider a low level to be a tenth of the number of agents ($n/10$): such a regret is equivalent to the difference of score of an alternative $x$ that results from switching from a profile $P$ to a profile $P'$ where ten percent of the agents rank $x$ last instead of first.
The results for various sizes are displayed in \cref{tab:lowRegret}. 
By relating the sizes to the column $k_{\MMR}$, we obtain the function $0.4m^{1.6}n$ as an approximation formula for the number of questions required to reach a low regret level. We see that the number of questions required grows approximately linearly with the number of agents and grows less fast than the square of the number of alternatives, for the sizes we considered. In other words, on average, assuming no questions get asked to the chair, each agent will have to answer $0.4m^{1.6}$ questions in order to to reach a low regret level, if this experimental formula reveals correct. For $m = 5$, that is about $6$ questions per agent; for $m = 10$, that is about $17$ questions per agent; and for $m = 15$, that is about $33$ questions per agent. These numbers have to be contrasted with those displayed in \cref{tab:fullProfile}. We constrained the Pessimistic strategy to ask questions to the agent only, and ran it on a single agent problem, with various values for $m$, to evaluate the number of questions required to elicit the full preference of an agent.

\subsection{Decrease in MMR values}
\begin{figure}
	\centering
	\includegraphics[width=.45\textwidth]{linearity.png}
	\caption{Average $\MMR$ after $k$ questions with Pessimistic strategy after 10 runs for different problem sizes.}
	\label{fig:linearity}
\end{figure}

We next wonder how fast the regret decreases, measured as a function of the number of questions asked. This is important practically because it may be impossible or unwanted to ask enough questions to reach low level of regrets, as computed in \cref{sec:lowRegret}. Knowing the speed of decrease permits to know which trade-offs can be achieved between the quality of the resulting recommendation and the number of questions that participants have to answer. \Cref{fig:linearity} shows graphically the decrease in $\MMR$ according to the number of questions asked for various problem sizes. We see that this function is convex, apart from the very first few questions, indicating to users of our approach that asking, for example, half the number of questions required to reach a low regret value will still yield a significant gain in regret, namely, more than half the distance to the low regret value.

\subsection{Value of information comparison}
\begin{table}
	\begin{center}
		\begin{tabular}{S[table-figures-integer=3, table-number-alignment = right, table-figures-decimal=0]S[table-number-alignment = right]@{ ± }S[table-number-alignment = left]S[table-number-alignment = right]@{ ± }S[table-number-alignment = left]S[table-number-alignment = right]@{ ± }S[table-number-alignment = left]S[table-number-alignment = right]@{ ± }S[table-number-alignment = left]}
			\toprule
			{p} & {2 ph.\ ca} & {sd} & {2 ph.\ ac} & {sd} \\
			\midrule
			0 & 0.66 & 0.67 & 0.65 & 0.65  \\
			15 & 0.71 & 0.73 & 0.67	& 0.64 \\
			30 & 0.62 & 0.5 & 0.8 & 0.62 \\
			50 & 1.23 & 1.46 & 0.8 & 0.79 \\
			100 & 2.03 & 1.56 & 1.88 & 1.02  \\
			150 & 3.41 & 1.79 & 4.07 & 1.51 \\
			200 & 4.61	& 1.35  & 6.53 & 2.0  \\
			250 & 8.4 & 1.86 & 	7.69 & 1.24 \\
			300 &11.07 & 0.86& 11.87 & 1.25 \\
			350 &15.06 & 0.68&15.59 & 0.6 \\
			400 &20.0 & 0.0 & 20.0 & 20.0 \\
			\bottomrule
		\end{tabular}
	\end{center}
	\caption{Average $\MMR$ in problems of size $(10, 20)$ after $400$ questions and $10$ runs. Where $p$ represents the number of questions asked to the chair.}
	\label{tab:twoP400}
\end{table}

%In this work, we advocate obtaining an approximate winner without determining completely the voting rule, or the profile. \Cref{sec:lowRegret} evaluates the number of questions per agent required to reach a low regret value.
%In this section we want to explore to which extent it is possible to recommend without knowing the voting rule. One extreme, in our setting, consists in knowing only that the weights are convex, our basis hypothesis. The 

%contrast our approach to one that would obtain the full profile on the one hand, and determine the voting rule on the other hand. More precisely, we want to estimate the number of questions that would be required to elicit the complete profile. We focus on the profile because eliciting the weights to full precision would require infinitely many questions, with the questions we use. 
%Note that this may underestimate the number of questions required to determine completely the profile and the rule, as questions to the agents yield no information about the scoring vector.

%To estimate this number, we draw one linear order randomly (with equiprobability over all possible linear orders) and assume the following querying process: given a partial ordering representing our knowledge so far of an agent preferences, we pick a question equiprobably among the remaining edges, and add the edge in the appropriate direction to our partial ordering. We count the average number of questions required to reach a complete ordering.

The experiments so far let the strategy target its questions unrestrictively, leaving open the possibility of questioning either the chair or an agent at each step. It may be desired in practical implementations to restrict the strategy to question the agents only, or mostly, or to wonder what is lost in terms of regret by asking different proportions of questions to the chair and the agents. Such restrictions may be useful because of (partial) unavailability of the chair, or because the estimated cognitive costs may differ sensibly, as these types of questions differ in nature. \Cref{tab:twoP400} show the MMR value that is reached in a problem of size $m = 10, n = 20$ after $400$ questions, using the Two phases strategy, in the “ca” (chair then agents) and in the “ac” (agents then chair) variants. These numbers are to be compared with the MMR value reached after 400 questions with the unconstrained Pessimistic strategy (displayed in \cref{fig:linearity}), which is $0.62$.% ± $0.56$.

These numbers highlight that it is possible to obtain a good-quality recommendation while knowing only that the voting rule is a scoring rule and that the weights are convex, which is our basic hypothesis: when $p = 0$, no question is asked to the chair, and a good-quality recommendation is still reached after a relatively small number of questions. By contrast, the Pessimistic strategy, when unconstrained, asks X questions to the chair.

\section{Conclusions}  
\label{sec:conclusions}
%TODO balance columns appropriately before submitting
%\balance
In this paper we have considered a social choice setting with partial information about the agent's preferences and a partially specified voting rule.
In this setting, we have proposed the use of minimax regret both as a means of robust winner determination as well as a guide to the process of simultaneous elicitation of preferences and voting rule.
Our experimental results %on randomly generated and real world data sets 
suggest that regret-based elicitation is effective and allow to quickly reduce worst regret significantly. They also highlight the importance of not insisting to obtain a completely specified voting rule, as good quality (low regret) recommendations can be achieved short of knowing the weights precisely, in our setting.
%regret, but they also show that starting with zero knowledge became a pretentious approach really quickly when increasing the size of the problem. Future works will therefore include the test of these strategies on partial specified profiles, ideally on real datasets. An other important direction is the extension to voting rules beyond scoring rules.
%Regret-based elicitation allows to determine near-optimal winners using only few information about the agent preferences.

As part of the contribution of this work, we publish an open-source library that allows to reproduce our experiments, and more (url not displayed for anonymity reasons).

We mention some directions for future works.
Further development of elicitation strategies, considering alternative heuristics, is an important direction. 
Second, elicitation could be extended to voting rules beyond scoring rules. 
Finally, an important direction of extension aims at studying how to elicit preferences while restraining to concrete and easy questions.
% Acknowledgements: We thank the reviewers for comments helping to improve the paper. 

%%%%%%%%%%%%%%%%%%%%%%%%%%%%%%%%%%%%%%%%%%%%%%%%%%%%%%%%%%%%%%%%%%%%%%%%

%%% The acknowledgments section is defined using the "acks" environment
%%% (rather than an unnumbered section). The use of this environment 
%%% ensures the proper identification of the section in the article 
%%% metadata as well as the consistent spelling of the heading.

%\begin{acks}
%If you wish to include any acknowledgments in your paper (e.g., to 
%people or funding agencies), please do so using the `\texttt{acks}' 
%environment. Note that the text of your acknowledgments will be omitted
%if you compile your document with the `\texttt{anonymous}' option.
%\end{acks}

%%%%%%%%%%%%%%%%%%%%%%%%%%%%%%%%%%%%%%%%%%%%%%%%%%%%%%%%%%%%%%%%%%%%%%%%

%%% The next two lines define, first, the bibliography style to be 
%%% applied, and, second, the bibliography file to be used.
%\bibliographystyle{abbrvnat} 
%\bibliography{biblio}

\bibliographystyle{ACM-Reference-Format} 
%TODO remove this when submitting, and consider solving the vertical overflows
\vfuzz=5pt
\bibliography{biblio}
\end{document}

%%%%%%%%%%%%%%%%%%%%%%%%%%%%%%%%%%%%%%%%%%%%%%%%%%%%%%%%%%%%%%%%%%%%%%%%

\appendix
\import{preamble/}{appendix.tex}


%%%%%%%%%%%%%%%%%%%%%%%%%%%%%%%%%%%%%%%%%%%%%%%%%%%%%%%%%%%%%%%%%%%%%%%%



